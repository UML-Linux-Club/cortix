%% Generated by Sphinx.
\def\sphinxdocclass{report}
\documentclass[letterpaper,10pt,openany,oneside,english]{sphinxmanual}
\ifdefined\pdfpxdimen
   \let\sphinxpxdimen\pdfpxdimen\else\newdimen\sphinxpxdimen
\fi \sphinxpxdimen=.75bp\relax

\PassOptionsToPackage{warn}{textcomp}
\usepackage[utf8]{inputenc}
\ifdefined\DeclareUnicodeCharacter
 \ifdefined\DeclareUnicodeCharacterAsOptional
  \DeclareUnicodeCharacter{"00A0}{\nobreakspace}
  \DeclareUnicodeCharacter{"2500}{\sphinxunichar{2500}}
  \DeclareUnicodeCharacter{"2502}{\sphinxunichar{2502}}
  \DeclareUnicodeCharacter{"2514}{\sphinxunichar{2514}}
  \DeclareUnicodeCharacter{"251C}{\sphinxunichar{251C}}
  \DeclareUnicodeCharacter{"2572}{\textbackslash}
 \else
  \DeclareUnicodeCharacter{00A0}{\nobreakspace}
  \DeclareUnicodeCharacter{2500}{\sphinxunichar{2500}}
  \DeclareUnicodeCharacter{2502}{\sphinxunichar{2502}}
  \DeclareUnicodeCharacter{2514}{\sphinxunichar{2514}}
  \DeclareUnicodeCharacter{251C}{\sphinxunichar{251C}}
  \DeclareUnicodeCharacter{2572}{\textbackslash}
 \fi
\fi
\usepackage{cmap}
\usepackage[T1]{fontenc}
\usepackage{amsmath,amssymb,amstext}
\usepackage{babel}
\usepackage{times}
\usepackage[Bjarne]{fncychap}
\usepackage{sphinx}

\usepackage{geometry}

% Include hyperref last.
\usepackage{hyperref}
% Fix anchor placement for figures with captions.
\usepackage{hypcap}% it must be loaded after hyperref.
% Set up styles of URL: it should be placed after hyperref.
\urlstyle{same}

\addto\captionsenglish{\renewcommand{\figurename}{Fig.}}
\addto\captionsenglish{\renewcommand{\tablename}{Table}}
\addto\captionsenglish{\renewcommand{\literalblockname}{Listing}}

\addto\captionsenglish{\renewcommand{\literalblockcontinuedname}{continued from previous page}}
\addto\captionsenglish{\renewcommand{\literalblockcontinuesname}{continues on next page}}

\addto\extrasenglish{\def\pageautorefname{page}}

\setcounter{tocdepth}{1}


\usepackage{amsmath}


\title{Cortix Documentation}
\date{Aug 16, 2019}
\release{0.1.0}
\author{Valmor F. de Almeida, Taha Azzaoui, Gilberto E. Alas}
\newcommand{\sphinxlogo}{\vbox{}}
\renewcommand{\releasename}{Release}
\makeindex

\begin{document}

\maketitle
\sphinxtableofcontents
\phantomsection\label{\detokenize{contents::doc}}



\chapter{src}
\label{\detokenize{src_rst/modules:src}}\label{\detokenize{src_rst/modules::doc}}

\section{cortix\_main}
\label{\detokenize{src_rst/cortix_main:module-cortix_main}}\label{\detokenize{src_rst/cortix_main:cortix-main}}\label{\detokenize{src_rst/cortix_main::doc}}\index{cortix\_main (module)}\index{Cortix (class in cortix\_main)}

\begin{fulllineitems}
\phantomsection\label{\detokenize{src_rst/cortix_main:cortix_main.Cortix}}\pysiglinewithargsret{\sphinxbfcode{\sphinxupquote{class }}\sphinxcode{\sphinxupquote{cortix\_main.}}\sphinxbfcode{\sphinxupquote{Cortix}}}{\emph{use\_mpi=False}, \emph{splash=False}}{}
Bases: \sphinxhref{https://docs.python.org/3/library/functions.html\#object}{\sphinxcode{\sphinxupquote{object}}}

Cortix main class definition.

The typical Cortix run file workflow:
\begin{enumerate}
\item {} 
Create the \sphinxtitleref{Cortix} object

\item {} 
Create tne (nested) network of modules

\item {} 
Run and close \sphinxtitleref{Cortix}

\end{enumerate}
\index{\_\_del\_\_() (cortix\_main.Cortix method)}

\begin{fulllineitems}
\phantomsection\label{\detokenize{src_rst/cortix_main:cortix_main.Cortix.__del__}}\pysiglinewithargsret{\sphinxbfcode{\sphinxupquote{\_\_del\_\_}}}{}{}
Destructs a Cortix simulation object.

\begin{sphinxadmonition}{warning}{Warning:}
By the time the body of this function is executed, the machinery of
variables may have been deleted already. For example, \sphinxtitleref{logging} is no longer
there; do the least amount of work here.
\end{sphinxadmonition}

\end{fulllineitems}

\index{\_\_init\_\_() (cortix\_main.Cortix method)}

\begin{fulllineitems}
\phantomsection\label{\detokenize{src_rst/cortix_main:cortix_main.Cortix.__init__}}\pysiglinewithargsret{\sphinxbfcode{\sphinxupquote{\_\_init\_\_}}}{\emph{use\_mpi=False}, \emph{splash=False}}{}
Construct a Cortix simulation object.
\begin{quote}\begin{description}
\item[{Parameters}] \leavevmode\begin{itemize}
\item {} 
\sphinxstyleliteralstrong{\sphinxupquote{use\_mpi}} (\sphinxhref{https://docs.python.org/3/library/functions.html\#bool}{\sphinxstyleliteralemphasis{\sphinxupquote{bool}}}) \textendash{} True for MPI, False for multiprocessing.

\item {} 
\sphinxstyleliteralstrong{\sphinxupquote{splash}} (\sphinxhref{https://docs.python.org/3/library/functions.html\#bool}{\sphinxstyleliteralemphasis{\sphinxupquote{bool}}}) \textendash{} Show the Cortix splash image.

\end{itemize}

\end{description}\end{quote}
\index{network (cortix\_main.Cortix attribute)}

\begin{fulllineitems}
\phantomsection\label{\detokenize{src_rst/cortix_main:cortix_main.Cortix.network}}\pysigline{\sphinxbfcode{\sphinxupquote{network}}}
\sphinxstyleemphasis{Network} \textendash{} A network of modules and their connectivity.

\end{fulllineitems}

\index{use\_mpi (cortix\_main.Cortix attribute)}

\begin{fulllineitems}
\phantomsection\label{\detokenize{src_rst/cortix_main:cortix_main.Cortix.use_mpi}}\pysigline{\sphinxbfcode{\sphinxupquote{use\_mpi}}}
\sphinxstyleemphasis{bool} \textendash{} \sphinxtitleref{True} for MPI, \sphinxtitleref{False} for Multiprocessing.

\end{fulllineitems}

\index{use\_multiprocessing (cortix\_main.Cortix attribute)}

\begin{fulllineitems}
\phantomsection\label{\detokenize{src_rst/cortix_main:cortix_main.Cortix.use_multiprocessing}}\pysigline{\sphinxbfcode{\sphinxupquote{use\_multiprocessing}}}
\sphinxstyleemphasis{bool} \textendash{} \sphinxtitleref{False} for MPI, \sphinxtitleref{True} for Multiprocessing.

\end{fulllineitems}

\index{splash (cortix\_main.Cortix attribute)}

\begin{fulllineitems}
\phantomsection\label{\detokenize{src_rst/cortix_main:cortix_main.Cortix.splash}}\pysigline{\sphinxbfcode{\sphinxupquote{splash}}}
\sphinxstyleemphasis{bool} \textendash{} Show the Cortix splash image.

\end{fulllineitems}

\index{comm (cortix\_main.Cortix attribute)}

\begin{fulllineitems}
\phantomsection\label{\detokenize{src_rst/cortix_main:cortix_main.Cortix.comm}}\pysigline{\sphinxbfcode{\sphinxupquote{comm}}}
\sphinxstyleemphasis{mpi4py.MPI.Intracomm} \textendash{} MPI.COMM\_WORLD (if using MPI else None).

\end{fulllineitems}

\index{rank (cortix\_main.Cortix attribute)}

\begin{fulllineitems}
\phantomsection\label{\detokenize{src_rst/cortix_main:cortix_main.Cortix.rank}}\pysigline{\sphinxbfcode{\sphinxupquote{rank}}}
\sphinxstyleemphasis{int} \textendash{} The current MPI rank (if using MPI else None).

\end{fulllineitems}

\index{size (cortix\_main.Cortix attribute)}

\begin{fulllineitems}
\phantomsection\label{\detokenize{src_rst/cortix_main:cortix_main.Cortix.size}}\pysigline{\sphinxbfcode{\sphinxupquote{size}}}
\sphinxstyleemphasis{int} \textendash{} size of the group associated with MPI.COMM\_WORLD.

\end{fulllineitems}


\end{fulllineitems}

\index{close() (cortix\_main.Cortix method)}

\begin{fulllineitems}
\phantomsection\label{\detokenize{src_rst/cortix_main:cortix_main.Cortix.close}}\pysiglinewithargsret{\sphinxbfcode{\sphinxupquote{close}}}{}{}
Closes the cortix object properly before destruction.

User is strongly advised to call this method at the end of the run file otherwise
timings will not be recorded.

\end{fulllineitems}

\index{network (cortix\_main.Cortix attribute)}

\begin{fulllineitems}
\pysigline{\sphinxbfcode{\sphinxupquote{network}}}
\end{fulllineitems}

\index{run() (cortix\_main.Cortix method)}

\begin{fulllineitems}
\phantomsection\label{\detokenize{src_rst/cortix_main:cortix_main.Cortix.run}}\pysiglinewithargsret{\sphinxbfcode{\sphinxupquote{run}}}{}{}
Run the Cortix network simulation.

\end{fulllineitems}


\end{fulllineitems}



\section{module module}
\label{\detokenize{src_rst/module:module-module}}\label{\detokenize{src_rst/module:module-module}}\label{\detokenize{src_rst/module::doc}}\index{module (module)}\index{Module (class in module)}

\begin{fulllineitems}
\phantomsection\label{\detokenize{src_rst/module:module.Module}}\pysigline{\sphinxbfcode{\sphinxupquote{class }}\sphinxcode{\sphinxupquote{module.}}\sphinxbfcode{\sphinxupquote{Module}}}
Bases: \sphinxhref{https://docs.python.org/3/library/functions.html\#object}{\sphinxcode{\sphinxupquote{object}}}

Cortix module super class.

This class provides facilities for creating modules within the Cortix network.
Cortix will map one object of this class to either a Multiprocessing or MPI
process depending on the user’s configuration.

\begin{sphinxadmonition}{note}{Note:}
This class is to be inherited by every Cortix module.
In order to execute, modules \sphinxstyleemphasis{must} override the \sphinxtitleref{run} method, which will be
executed during the simulation.
\end{sphinxadmonition}
\index{\_\_init\_\_() (module.Module method)}

\begin{fulllineitems}
\phantomsection\label{\detokenize{src_rst/module:module.Module.__init__}}\pysiglinewithargsret{\sphinxbfcode{\sphinxupquote{\_\_init\_\_}}}{}{}
Module super class constructor.

\begin{sphinxadmonition}{note}{Note:}
This constructor must be called explicitly in the constructor of every
Cortix module like so:
\begin{quote}

super().\_\_init\_\_()
\end{quote}
\end{sphinxadmonition}
\index{name (module.Module attribute)}

\begin{fulllineitems}
\phantomsection\label{\detokenize{src_rst/module:module.Module.name}}\pysigline{\sphinxbfcode{\sphinxupquote{name}}}
\sphinxstyleemphasis{str} \textendash{} A name given to the instance. Default is the derived class name.

\end{fulllineitems}

\index{port\_names\_expected (module.Module attribute)}

\begin{fulllineitems}
\phantomsection\label{\detokenize{src_rst/module:module.Module.port_names_expected}}\pysigline{\sphinxbfcode{\sphinxupquote{port\_names\_expected}}}
\sphinxstyleemphasis{list(str), None} \textendash{} A list of names of ports expected in the module. This will be compared
to port names during runtime to check against the intended use of the
module.

\end{fulllineitems}

\index{state (module.Module attribute)}

\begin{fulllineitems}
\phantomsection\label{\detokenize{src_rst/module:module.Module.state}}\pysigline{\sphinxbfcode{\sphinxupquote{state}}}
\sphinxstyleemphasis{any} \textendash{} Any \sphinxtitleref{pickle-able} data structure to be passed in a \sphinxtitleref{multiprocessing.Queue}
to the parent process or to be gathered in the root MPI process.
Default is \sphinxtitleref{None}.

\end{fulllineitems}

\index{use\_mpi (module.Module attribute)}

\begin{fulllineitems}
\phantomsection\label{\detokenize{src_rst/module:module.Module.use_mpi}}\pysigline{\sphinxbfcode{\sphinxupquote{use\_mpi}}}
\sphinxstyleemphasis{bool} \textendash{} \sphinxtitleref{True} for MPI, \sphinxtitleref{False} for Multiprocessing

\end{fulllineitems}

\index{use\_multiprocessing (module.Module attribute)}

\begin{fulllineitems}
\phantomsection\label{\detokenize{src_rst/module:module.Module.use_multiprocessing}}\pysigline{\sphinxbfcode{\sphinxupquote{use\_multiprocessing}}}
\sphinxstyleemphasis{bool} \textendash{} \sphinxtitleref{False} for MPI, \sphinxtitleref{True} for Multiprocessing

\end{fulllineitems}

\index{ports (module.Module attribute)}

\begin{fulllineitems}
\phantomsection\label{\detokenize{src_rst/module:module.Module.ports}}\pysigline{\sphinxbfcode{\sphinxupquote{ports}}}
\sphinxstyleemphasis{list(Port)} \textendash{} A list of ports contained by the module

\end{fulllineitems}

\index{\_\_network (module.Module attribute)}

\begin{fulllineitems}
\phantomsection\label{\detokenize{src_rst/module:module.Module.__network}}\pysigline{\sphinxbfcode{\sphinxupquote{\_\_network}}}
\sphinxstyleemphasis{Network} \textendash{} An internal network inherited by the derived module for nested networks.

\end{fulllineitems}


\end{fulllineitems}

\index{get\_port() (module.Module method)}

\begin{fulllineitems}
\phantomsection\label{\detokenize{src_rst/module:module.Module.get_port}}\pysiglinewithargsret{\sphinxbfcode{\sphinxupquote{get\_port}}}{\emph{name}}{}
Get port by name; if it does not exist, create one.
\begin{quote}\begin{description}
\item[{Parameters}] \leavevmode
\sphinxstyleliteralstrong{\sphinxupquote{name}} (\sphinxhref{https://docs.python.org/3/library/stdtypes.html\#str}{\sphinxstyleliteralemphasis{\sphinxupquote{str}}}) \textendash{} The name of the port to get

\item[{Returns}] \leavevmode
\sphinxstylestrong{port} \textendash{} The port object with the corresponding name

\item[{Return type}] \leavevmode
{\hyperref[\detokenize{src_rst/port:port.Port}]{\sphinxcrossref{Port}}}

\end{description}\end{quote}

\end{fulllineitems}

\index{network (module.Module attribute)}

\begin{fulllineitems}
\phantomsection\label{\detokenize{src_rst/module:module.Module.network}}\pysigline{\sphinxbfcode{\sphinxupquote{network}}}
\end{fulllineitems}

\index{recv() (module.Module method)}

\begin{fulllineitems}
\phantomsection\label{\detokenize{src_rst/module:module.Module.recv}}\pysiglinewithargsret{\sphinxbfcode{\sphinxupquote{recv}}}{\emph{port}}{}
Receive data from a given port

\begin{sphinxadmonition}{warning}{Warning:}
This function will block until data is available
\end{sphinxadmonition}
\begin{quote}\begin{description}
\item[{Parameters}] \leavevmode
\sphinxstyleliteralstrong{\sphinxupquote{port}} ({\hyperref[\detokenize{src_rst/port:port.Port}]{\sphinxcrossref{\sphinxstyleliteralemphasis{\sphinxupquote{Port}}}}}\sphinxstyleliteralemphasis{\sphinxupquote{, }}\sphinxhref{https://docs.python.org/3/library/stdtypes.html\#str}{\sphinxstyleliteralemphasis{\sphinxupquote{str}}}) \textendash{} A Port object to send the data through, or its string name

\item[{Returns}] \leavevmode
\sphinxstylestrong{data} \textendash{} The data received through the port

\item[{Return type}] \leavevmode
any

\end{description}\end{quote}

\end{fulllineitems}

\index{run() (module.Module method)}

\begin{fulllineitems}
\phantomsection\label{\detokenize{src_rst/module:module.Module.run}}\pysiglinewithargsret{\sphinxbfcode{\sphinxupquote{run}}}{\emph{*args}}{}
Module run function

Run method with an option to pass data back to the parent process when running
in Python multiprocessing mode. If the user does not want to share data with
the parent process, this function can be overriden with \sphinxtitleref{run(self)}
or \sphinxtitleref{run(self, *args)} as long as \sphinxtitleref{self.state = None}.
If \sphinxtitleref{self.state} points to anything but \sphinxtitleref{None}, the user must use
{\color{red}\bfseries{}{}`}run(self, {\color{red}\bfseries{}*}args).

\begin{sphinxadmonition}{note}{Notes}

When in multiprocessing, \sphinxtitleref{*args} has two elements: \sphinxtitleref{comm\_idx} and \sphinxtitleref{comm\_state}.
To pass back the state of the module, the user should insert the provided
index \sphinxtitleref{comm\_idx} and the \sphinxtitleref{state} into the queue as follows:
\begin{quote}
\begin{description}
\item[{if self.use\_multiprocessing:}] \leavevmode\begin{description}
\item[{try:}] \leavevmode
pickle.dumps(self.state)

\item[{except pickle.PicklingError:}] \leavevmode
args{[}1{]}.put((arg{[}0{]},None))

\item[{else:}] \leavevmode
args{[}1{]}.put((arg{[}0{]},self.state))

\end{description}

\end{description}
\end{quote}

at the bottom of the user defined \sphinxtitleref{run()} function.
\end{sphinxadmonition}

\begin{sphinxadmonition}{warning}{Warning:}
This function must be overridden by all Cortix modules
\end{sphinxadmonition}
\begin{quote}\begin{description}
\item[{Parameters}] \leavevmode\begin{itemize}
\item {} 
\sphinxstyleliteralstrong{\sphinxupquote{arg}}\sphinxstyleliteralstrong{\sphinxupquote{{[}}}\sphinxstyleliteralstrong{\sphinxupquote{0}}\sphinxstyleliteralstrong{\sphinxupquote{{]}}} (\sphinxhref{https://docs.python.org/3/library/functions.html\#int}{\sphinxstyleliteralemphasis{\sphinxupquote{int}}}) \textendash{} Index of the state in the communication queue.

\item {} 
\sphinxstyleliteralstrong{\sphinxupquote{arg}}\sphinxstyleliteralstrong{\sphinxupquote{{[}}}\sphinxstyleliteralstrong{\sphinxupquote{1}}\sphinxstyleliteralstrong{\sphinxupquote{{]}}} (\sphinxhref{https://docs.python.org/3/library/multiprocessing.html\#multiprocessing.Queue}{\sphinxstyleliteralemphasis{\sphinxupquote{multiprocessing.Queue}}}) \textendash{} When using the Python \sphinxtitleref{multiprocessing} library \sphinxtitleref{state\_comm} must have
the module’s \sphinxtitleref{self.state} in it. That is,
\sphinxtitleref{state\_comm.put((idx\_comm,self.state))} must be the last command in the
method before \sphinxtitleref{return}. In addition, self.state must be \sphinxtitleref{pickle-able}.

\end{itemize}

\end{description}\end{quote}

\end{fulllineitems}

\index{send() (module.Module method)}

\begin{fulllineitems}
\phantomsection\label{\detokenize{src_rst/module:module.Module.send}}\pysiglinewithargsret{\sphinxbfcode{\sphinxupquote{send}}}{\emph{data}, \emph{port}}{}
Send data through a given port.
\begin{quote}\begin{description}
\item[{Parameters}] \leavevmode\begin{itemize}
\item {} 
\sphinxstyleliteralstrong{\sphinxupquote{data}} (\sphinxstyleliteralemphasis{\sphinxupquote{any}}) \textendash{} The data being sent out - must be pickleable

\item {} 
\sphinxstyleliteralstrong{\sphinxupquote{port}} ({\hyperref[\detokenize{src_rst/port:port.Port}]{\sphinxcrossref{\sphinxstyleliteralemphasis{\sphinxupquote{Port}}}}}\sphinxstyleliteralemphasis{\sphinxupquote{, }}\sphinxhref{https://docs.python.org/3/library/stdtypes.html\#str}{\sphinxstyleliteralemphasis{\sphinxupquote{str}}}) \textendash{} A Port object to send the data through, or its string name

\end{itemize}

\end{description}\end{quote}

\end{fulllineitems}


\end{fulllineitems}



\section{network module}
\label{\detokenize{src_rst/network:module-network}}\label{\detokenize{src_rst/network:network-module}}\label{\detokenize{src_rst/network::doc}}\index{network (module)}\index{Network (class in network)}

\begin{fulllineitems}
\phantomsection\label{\detokenize{src_rst/network:network.Network}}\pysigline{\sphinxbfcode{\sphinxupquote{class }}\sphinxcode{\sphinxupquote{network.}}\sphinxbfcode{\sphinxupquote{Network}}}
Bases: \sphinxhref{https://docs.python.org/3/library/functions.html\#object}{\sphinxcode{\sphinxupquote{object}}}

Cortix network.
\index{n\_networks (network.Network attribute)}

\begin{fulllineitems}
\phantomsection\label{\detokenize{src_rst/network:network.Network.n_networks}}\pysigline{\sphinxbfcode{\sphinxupquote{n\_networks}}}
\sphinxstyleemphasis{int} \textendash{} Number of instances of this class.

\end{fulllineitems}

\index{\_\_init\_\_() (network.Network method)}

\begin{fulllineitems}
\phantomsection\label{\detokenize{src_rst/network:network.Network.__init__}}\pysiglinewithargsret{\sphinxbfcode{\sphinxupquote{\_\_init\_\_}}}{}{}
Module super class constructor.
\index{max\_n\_modules\_for\_data\_copy\_on\_root (network.Network attribute)}

\begin{fulllineitems}
\phantomsection\label{\detokenize{src_rst/network:network.Network.max_n_modules_for_data_copy_on_root}}\pysigline{\sphinxbfcode{\sphinxupquote{max\_n\_modules\_for\_data\_copy\_on\_root}}}
\sphinxstyleemphasis{int} \textendash{} When using MPI the \sphinxtitleref{network} will copy the data from all modules on the
root process. This can generate an \sphinxtitleref{out of memory} condition. This variable
sets the maximum number of processes for which the data will be copied.
Default is 1000.

\end{fulllineitems}


\end{fulllineitems}

\index{add\_module() (network.Network method)}

\begin{fulllineitems}
\phantomsection\label{\detokenize{src_rst/network:network.Network.add_module}}\pysiglinewithargsret{\sphinxbfcode{\sphinxupquote{add\_module}}}{\emph{m}}{}
Alternative name to \sphinxtitleref{module()}.

\end{fulllineitems}

\index{connect() (network.Network method)}

\begin{fulllineitems}
\phantomsection\label{\detokenize{src_rst/network:network.Network.connect}}\pysiglinewithargsret{\sphinxbfcode{\sphinxupquote{connect}}}{\emph{module\_port\_a}, \emph{module\_port\_b}, \emph{info=None}}{}
Connect two modules using either their ports directly or inferred ports.

A connection always opens a channel for data communication in both ways.
That is, both sends and receives are allowed.

\begin{sphinxadmonition}{note}{Note:}
The simplest form of usage is with arguments: (\sphinxtitleref{module\_a}, \sphinxtitleref{module\_b}).
In this case, a \sphinxtitleref{port} with the name of \sphinxtitleref{module\_a} \sphinxstylestrong{must} exist in \sphinxtitleref{module\_b},
and vice-versa (port names as \sphinxtitleref{str} in lower case). In addition, the connect
must not be called again with these same two modules, else the underlying
connection will be overriden.

For more rigorous connection, the user is advised to fully specify the
module and the port in each list argument.
\end{sphinxadmonition}
\begin{quote}\begin{description}
\item[{Parameters}] \leavevmode\begin{itemize}
\item {} 
\sphinxstyleliteralstrong{\sphinxupquote{module\_port\_a}} (\sphinxhref{https://docs.python.org/3/library/stdtypes.html\#list}{\sphinxstyleliteralemphasis{\sphinxupquote{list}}}\sphinxstyleliteralemphasis{\sphinxupquote{(}}\sphinxstyleliteralemphasis{\sphinxupquote{{[}}}{\hyperref[\detokenize{src_rst/module:module.Module}]{\sphinxcrossref{\sphinxstyleliteralemphasis{\sphinxupquote{Module}}}}}\sphinxstyleliteralemphasis{\sphinxupquote{,}}{\hyperref[\detokenize{src_rst/port:port.Port}]{\sphinxcrossref{\sphinxstyleliteralemphasis{\sphinxupquote{Port}}}}}\sphinxstyleliteralemphasis{\sphinxupquote{{]}}}\sphinxstyleliteralemphasis{\sphinxupquote{) or }}\sphinxhref{https://docs.python.org/3/library/stdtypes.html\#list}{\sphinxstyleliteralemphasis{\sphinxupquote{list}}}\sphinxstyleliteralemphasis{\sphinxupquote{(}}\sphinxstyleliteralemphasis{\sphinxupquote{{[}}}{\hyperref[\detokenize{src_rst/module:module.Module}]{\sphinxcrossref{\sphinxstyleliteralemphasis{\sphinxupquote{Module}}}}}\sphinxstyleliteralemphasis{\sphinxupquote{,}}\sphinxhref{https://docs.python.org/3/library/stdtypes.html\#str}{\sphinxstyleliteralemphasis{\sphinxupquote{str}}}\sphinxstyleliteralemphasis{\sphinxupquote{{]}}}\sphinxstyleliteralemphasis{\sphinxupquote{) or }}{\hyperref[\detokenize{src_rst/module:module.Module}]{\sphinxcrossref{\sphinxstyleliteralemphasis{\sphinxupquote{Module}}}}}) \textendash{} First \sphinxtitleref{module}-\sphinxtitleref{port} to connect.

\item {} 
\sphinxstyleliteralstrong{\sphinxupquote{module\_port\_b}} (\sphinxhref{https://docs.python.org/3/library/stdtypes.html\#list}{\sphinxstyleliteralemphasis{\sphinxupquote{list}}}\sphinxstyleliteralemphasis{\sphinxupquote{(}}\sphinxstyleliteralemphasis{\sphinxupquote{{[}}}{\hyperref[\detokenize{src_rst/module:module.Module}]{\sphinxcrossref{\sphinxstyleliteralemphasis{\sphinxupquote{Module}}}}}\sphinxstyleliteralemphasis{\sphinxupquote{,}}{\hyperref[\detokenize{src_rst/port:port.Port}]{\sphinxcrossref{\sphinxstyleliteralemphasis{\sphinxupquote{Port}}}}}\sphinxstyleliteralemphasis{\sphinxupquote{{]}}}\sphinxstyleliteralemphasis{\sphinxupquote{) or }}\sphinxhref{https://docs.python.org/3/library/stdtypes.html\#list}{\sphinxstyleliteralemphasis{\sphinxupquote{list}}}\sphinxstyleliteralemphasis{\sphinxupquote{(}}\sphinxstyleliteralemphasis{\sphinxupquote{{[}}}{\hyperref[\detokenize{src_rst/module:module.Module}]{\sphinxcrossref{\sphinxstyleliteralemphasis{\sphinxupquote{Module}}}}}\sphinxstyleliteralemphasis{\sphinxupquote{,}}\sphinxhref{https://docs.python.org/3/library/stdtypes.html\#str}{\sphinxstyleliteralemphasis{\sphinxupquote{str}}}\sphinxstyleliteralemphasis{\sphinxupquote{{]}}}\sphinxstyleliteralemphasis{\sphinxupquote{) or }}{\hyperref[\detokenize{src_rst/module:module.Module}]{\sphinxcrossref{\sphinxstyleliteralemphasis{\sphinxupquote{Module}}}}}) \textendash{} Second \sphinxtitleref{module}-\sphinxtitleref{port} to connect.

\item {} 
\sphinxstyleliteralstrong{\sphinxupquote{info}} (\sphinxhref{https://docs.python.org/3/library/stdtypes.html\#str}{\sphinxstyleliteralemphasis{\sphinxupquote{str}}}) \textendash{} Information on the directionality of the information flow. This is for
graph visualization purposes only. The default value will use the order
in the argument list to define the direction. Default: None. If set
to \sphinxtitleref{bidiretional}, will create a double headed arrow in the graph figure.

\end{itemize}

\end{description}\end{quote}

\end{fulllineitems}

\index{draw() (network.Network method)}

\begin{fulllineitems}
\phantomsection\label{\detokenize{src_rst/network:network.Network.draw}}\pysiglinewithargsret{\sphinxbfcode{\sphinxupquote{draw}}}{\emph{graph\_attr=None}, \emph{node\_attr=None}, \emph{engine=None}}{}
\end{fulllineitems}

\index{module() (network.Network method)}

\begin{fulllineitems}
\phantomsection\label{\detokenize{src_rst/network:network.Network.module}}\pysiglinewithargsret{\sphinxbfcode{\sphinxupquote{module}}}{\emph{m}}{}
Add a module.

\end{fulllineitems}

\index{n\_networks (network.Network attribute)}

\begin{fulllineitems}
\pysigline{\sphinxbfcode{\sphinxupquote{n\_networks}}\sphinxbfcode{\sphinxupquote{ = 0}}}
\end{fulllineitems}

\index{run() (network.Network method)}

\begin{fulllineitems}
\phantomsection\label{\detokenize{src_rst/network:network.Network.run}}\pysiglinewithargsret{\sphinxbfcode{\sphinxupquote{run}}}{}{}
Run the network simulation.

This function concurrently executes the \sphinxtitleref{cortix.src.module.run} function
for each module in the network. Modules are run using either MPI or
Multiprocessing, depending on the user configuration.

\begin{sphinxadmonition}{note}{Note:}
When using multiprocessing, data from the modules state are copied to the master
process after the \sphinxtitleref{run()} method of the modules is finished.
\end{sphinxadmonition}

\end{fulllineitems}


\end{fulllineitems}



\section{port module}
\label{\detokenize{src_rst/port:module-port}}\label{\detokenize{src_rst/port:port-module}}\label{\detokenize{src_rst/port::doc}}\index{port (module)}\index{Port (class in port)}

\begin{fulllineitems}
\phantomsection\label{\detokenize{src_rst/port:port.Port}}\pysiglinewithargsret{\sphinxbfcode{\sphinxupquote{class }}\sphinxcode{\sphinxupquote{port.}}\sphinxbfcode{\sphinxupquote{Port}}}{\emph{name=None}, \emph{use\_mpi=False}}{}
Bases: \sphinxhref{https://docs.python.org/3/library/functions.html\#object}{\sphinxcode{\sphinxupquote{object}}}

Provides a method of communication between modules.

The Port class provides an interface for creating ports and connecting them to
other ports for the purpose of data transfer. Data exchange takes place by
send and/or receive calls on a given port. The concept of a port is that of a data
transfer “interaction.” This can be one- or two-way with sends and receives.
A port is connected to only one other port; as two ends of a pipe are connected.
\index{\_\_eq\_\_() (port.Port method)}

\begin{fulllineitems}
\phantomsection\label{\detokenize{src_rst/port:port.Port.__eq__}}\pysiglinewithargsret{\sphinxbfcode{\sphinxupquote{\_\_eq\_\_}}}{\emph{other}}{}
Check for port equality

\end{fulllineitems}

\index{\_\_init\_\_() (port.Port method)}

\begin{fulllineitems}
\phantomsection\label{\detokenize{src_rst/port:port.Port.__init__}}\pysiglinewithargsret{\sphinxbfcode{\sphinxupquote{\_\_init\_\_}}}{\emph{name=None}, \emph{use\_mpi=False}}{}
Constructs a Port object
\begin{quote}\begin{description}
\item[{Parameters}] \leavevmode\begin{itemize}
\item {} 
\sphinxstyleliteralstrong{\sphinxupquote{name}} (\sphinxhref{https://docs.python.org/3/library/stdtypes.html\#str}{\sphinxstyleliteralemphasis{\sphinxupquote{str}}}) \textendash{} The name of the Port object

\item {} 
\sphinxstyleliteralstrong{\sphinxupquote{use\_mpi}} (\sphinxhref{https://docs.python.org/3/library/functions.html\#bool}{\sphinxstyleliteralemphasis{\sphinxupquote{bool}}}) \textendash{} True for MPI, False for Multiprocessing

\end{itemize}

\end{description}\end{quote}
\index{id (port.Port attribute)}

\begin{fulllineitems}
\phantomsection\label{\detokenize{src_rst/port:port.Port.id}}\pysigline{\sphinxbfcode{\sphinxupquote{id}}}
\sphinxstyleemphasis{int}

\end{fulllineitems}

\index{name (port.Port attribute)}

\begin{fulllineitems}
\phantomsection\label{\detokenize{src_rst/port:port.Port.name}}\pysigline{\sphinxbfcode{\sphinxupquote{name}}}
\sphinxstyleemphasis{string}

\end{fulllineitems}

\index{use\_mpi (port.Port attribute)}

\begin{fulllineitems}
\phantomsection\label{\detokenize{src_rst/port:port.Port.use_mpi}}\pysigline{\sphinxbfcode{\sphinxupquote{use\_mpi}}}
\sphinxstyleemphasis{bool}

\end{fulllineitems}


\end{fulllineitems}

\index{\_\_repr\_\_() (port.Port method)}

\begin{fulllineitems}
\phantomsection\label{\detokenize{src_rst/port:port.Port.__repr__}}\pysiglinewithargsret{\sphinxbfcode{\sphinxupquote{\_\_repr\_\_}}}{}{}
Port name representation

\end{fulllineitems}

\index{connect() (port.Port method)}

\begin{fulllineitems}
\phantomsection\label{\detokenize{src_rst/port:port.Port.connect}}\pysiglinewithargsret{\sphinxbfcode{\sphinxupquote{connect}}}{\emph{port}}{}
Connect this port to another port

Ports must be connected for data to flow between them.
\begin{quote}\begin{description}
\item[{Parameters}] \leavevmode
\sphinxstyleliteralstrong{\sphinxupquote{port}} ({\hyperref[\detokenize{src_rst/port:port.Port}]{\sphinxcrossref{\sphinxstyleliteralemphasis{\sphinxupquote{Port}}}}}) \textendash{} A Port object to connect to

\end{description}\end{quote}

\end{fulllineitems}

\index{recv() (port.Port method)}

\begin{fulllineitems}
\phantomsection\label{\detokenize{src_rst/port:port.Port.recv}}\pysiglinewithargsret{\sphinxbfcode{\sphinxupquote{recv}}}{}{}
Receive data from the connected port.

\begin{sphinxadmonition}{warning}{Warning:}
This function will block if no data has been sent yet.
\end{sphinxadmonition}
\begin{quote}\begin{description}
\item[{Returns}] \leavevmode
\sphinxstylestrong{data}

\item[{Return type}] \leavevmode
any

\end{description}\end{quote}

\end{fulllineitems}

\index{send() (port.Port method)}

\begin{fulllineitems}
\phantomsection\label{\detokenize{src_rst/port:port.Port.send}}\pysiglinewithargsret{\sphinxbfcode{\sphinxupquote{send}}}{\emph{data}, \emph{tag=None}}{}
Send data to the connected port.

If the sending port is not connected do nothing.
\begin{quote}\begin{description}
\item[{Parameters}] \leavevmode\begin{itemize}
\item {} 
\sphinxstyleliteralstrong{\sphinxupquote{data}} (\sphinxstyleliteralemphasis{\sphinxupquote{any}}) \textendash{} This data must be pickleable

\item {} 
\sphinxstyleliteralstrong{\sphinxupquote{tag}} (\sphinxhref{https://docs.python.org/3/library/functions.html\#int}{\sphinxstyleliteralemphasis{\sphinxupquote{int}}}\sphinxstyleliteralemphasis{\sphinxupquote{, }}\sphinxstyleliteralemphasis{\sphinxupquote{optional}}) \textendash{} MPI tag used in sending data

\end{itemize}

\end{description}\end{quote}

\end{fulllineitems}


\end{fulllineitems}



\chapter{examples}
\label{\detokenize{examples_rst/modules:examples}}\label{\detokenize{examples_rst/modules::doc}}

\section{adjudication module}
\label{\detokenize{examples_rst/adjudication:module-adjudication}}\label{\detokenize{examples_rst/adjudication:adjudication-module}}\label{\detokenize{examples_rst/adjudication::doc}}\index{adjudication (module)}\index{Adjudication (class in adjudication)}

\begin{fulllineitems}
\phantomsection\label{\detokenize{examples_rst/adjudication:adjudication.Adjudication}}\pysiglinewithargsret{\sphinxbfcode{\sphinxupquote{class }}\sphinxcode{\sphinxupquote{adjudication.}}\sphinxbfcode{\sphinxupquote{Adjudication}}}{\emph{n\_groups=1}, \emph{pool\_size=0.0}}{}
Bases: \sphinxcode{\sphinxupquote{cortix.src.module.Module}}

Adjudication Cortix module used to model criminal group population in an
adjudication system.

\begin{sphinxadmonition}{note}{Notes}

These are the \sphinxtitleref{port} names available in this module to connect to respective
modules: \sphinxtitleref{probation}, \sphinxtitleref{jail}, \sphinxtitleref{arrested}, \sphinxtitleref{prison}, and \sphinxtitleref{community}.
See instance attribute \sphinxtitleref{port\_names\_expected}.
\end{sphinxadmonition}
\index{\_\_init\_\_() (adjudication.Adjudication method)}

\begin{fulllineitems}
\phantomsection\label{\detokenize{examples_rst/adjudication:adjudication.Adjudication.__init__}}\pysiglinewithargsret{\sphinxbfcode{\sphinxupquote{\_\_init\_\_}}}{\emph{n\_groups=1}, \emph{pool\_size=0.0}}{}~\begin{quote}\begin{description}
\item[{Parameters}] \leavevmode\begin{itemize}
\item {} 
\sphinxstyleliteralstrong{\sphinxupquote{n\_groups}} (\sphinxhref{https://docs.python.org/3/library/functions.html\#int}{\sphinxstyleliteralemphasis{\sphinxupquote{int}}}) \textendash{} Number of groups in the population.

\item {} 
\sphinxstyleliteralstrong{\sphinxupquote{pool\_size}} (\sphinxhref{https://docs.python.org/3/library/functions.html\#float}{\sphinxstyleliteralemphasis{\sphinxupquote{float}}}) \textendash{} Upperbound on the range of the existing population groups. A random value
from 0 to the upperbound value will be assigned to each group.

\end{itemize}

\end{description}\end{quote}

\end{fulllineitems}

\index{run() (adjudication.Adjudication method)}

\begin{fulllineitems}
\phantomsection\label{\detokenize{examples_rst/adjudication:adjudication.Adjudication.run}}\pysiglinewithargsret{\sphinxbfcode{\sphinxupquote{run}}}{\emph{*args}}{}
Module run function

Run method with an option to pass data back to the parent process when running
in Python multiprocessing mode. If the user does not want to share data with
the parent process, this function can be overriden with \sphinxtitleref{run(self)}
or \sphinxtitleref{run(self, *args)} as long as \sphinxtitleref{self.state = None}.
If \sphinxtitleref{self.state} points to anything but \sphinxtitleref{None}, the user must use
{\color{red}\bfseries{}{}`}run(self, {\color{red}\bfseries{}*}args).

\begin{sphinxadmonition}{note}{Notes}

When in multiprocessing, \sphinxtitleref{*args} has two elements: \sphinxtitleref{comm\_idx} and \sphinxtitleref{comm\_state}.
To pass back the state of the module, the user should insert the provided
index \sphinxtitleref{comm\_idx} and the \sphinxtitleref{state} into the queue as follows:
\begin{quote}
\begin{description}
\item[{if self.use\_multiprocessing:}] \leavevmode\begin{description}
\item[{try:}] \leavevmode
pickle.dumps(self.state)

\item[{except pickle.PicklingError:}] \leavevmode
args{[}1{]}.put((arg{[}0{]},None))

\item[{else:}] \leavevmode
args{[}1{]}.put((arg{[}0{]},self.state))

\end{description}

\end{description}
\end{quote}

at the bottom of the user defined \sphinxtitleref{run()} function.
\end{sphinxadmonition}

\begin{sphinxadmonition}{warning}{Warning:}
This function must be overridden by all Cortix modules
\end{sphinxadmonition}
\begin{quote}\begin{description}
\item[{Parameters}] \leavevmode\begin{itemize}
\item {} 
\sphinxstyleliteralstrong{\sphinxupquote{arg}}\sphinxstyleliteralstrong{\sphinxupquote{{[}}}\sphinxstyleliteralstrong{\sphinxupquote{0}}\sphinxstyleliteralstrong{\sphinxupquote{{]}}} (\sphinxhref{https://docs.python.org/3/library/functions.html\#int}{\sphinxstyleliteralemphasis{\sphinxupquote{int}}}) \textendash{} Index of the state in the communication queue.

\item {} 
\sphinxstyleliteralstrong{\sphinxupquote{arg}}\sphinxstyleliteralstrong{\sphinxupquote{{[}}}\sphinxstyleliteralstrong{\sphinxupquote{1}}\sphinxstyleliteralstrong{\sphinxupquote{{]}}} (\sphinxhref{https://docs.python.org/3/library/multiprocessing.html\#multiprocessing.Queue}{\sphinxstyleliteralemphasis{\sphinxupquote{multiprocessing.Queue}}}) \textendash{} When using the Python \sphinxtitleref{multiprocessing} library \sphinxtitleref{state\_comm} must have
the module’s \sphinxtitleref{self.state} in it. That is,
\sphinxtitleref{state\_comm.put((idx\_comm,self.state))} must be the last command in the
method before \sphinxtitleref{return}. In addition, self.state must be \sphinxtitleref{pickle-able}.

\end{itemize}

\end{description}\end{quote}

\end{fulllineitems}


\end{fulllineitems}



\section{arrested module}
\label{\detokenize{examples_rst/arrested:module-arrested}}\label{\detokenize{examples_rst/arrested:arrested-module}}\label{\detokenize{examples_rst/arrested::doc}}\index{arrested (module)}\index{Arrested (class in arrested)}

\begin{fulllineitems}
\phantomsection\label{\detokenize{examples_rst/arrested:arrested.Arrested}}\pysiglinewithargsret{\sphinxbfcode{\sphinxupquote{class }}\sphinxcode{\sphinxupquote{arrested.}}\sphinxbfcode{\sphinxupquote{Arrested}}}{\emph{n\_groups=1}, \emph{pool\_size=0.0}}{}
Bases: \sphinxcode{\sphinxupquote{cortix.src.module.Module}}

Arrested Cortix module used to model criminal group population in an arrested system.

\begin{sphinxadmonition}{note}{Notes}

These are the \sphinxtitleref{port} names available in this module to connect to respective
modules: \sphinxtitleref{probation}, \sphinxtitleref{adjudication}, \sphinxtitleref{jail}, and \sphinxtitleref{community}.
See instance attribute \sphinxtitleref{port\_names\_expected}.
\end{sphinxadmonition}
\index{\_\_init\_\_() (arrested.Arrested method)}

\begin{fulllineitems}
\phantomsection\label{\detokenize{examples_rst/arrested:arrested.Arrested.__init__}}\pysiglinewithargsret{\sphinxbfcode{\sphinxupquote{\_\_init\_\_}}}{\emph{n\_groups=1}, \emph{pool\_size=0.0}}{}~\begin{quote}\begin{description}
\item[{Parameters}] \leavevmode\begin{itemize}
\item {} 
\sphinxstyleliteralstrong{\sphinxupquote{n\_groups}} (\sphinxhref{https://docs.python.org/3/library/functions.html\#int}{\sphinxstyleliteralemphasis{\sphinxupquote{int}}}) \textendash{} Number of groups in the population.

\item {} 
\sphinxstyleliteralstrong{\sphinxupquote{pool\_size}} (\sphinxhref{https://docs.python.org/3/library/functions.html\#float}{\sphinxstyleliteralemphasis{\sphinxupquote{float}}}) \textendash{} Upperbound on the range of the existing population groups. A random value
from 0 to the upperbound value will be assigned to each group.

\end{itemize}

\end{description}\end{quote}

\end{fulllineitems}

\index{run() (arrested.Arrested method)}

\begin{fulllineitems}
\phantomsection\label{\detokenize{examples_rst/arrested:arrested.Arrested.run}}\pysiglinewithargsret{\sphinxbfcode{\sphinxupquote{run}}}{\emph{*args}}{}
Module run function

Run method with an option to pass data back to the parent process when running
in Python multiprocessing mode. If the user does not want to share data with
the parent process, this function can be overriden with \sphinxtitleref{run(self)}
or \sphinxtitleref{run(self, *args)} as long as \sphinxtitleref{self.state = None}.
If \sphinxtitleref{self.state} points to anything but \sphinxtitleref{None}, the user must use
{\color{red}\bfseries{}{}`}run(self, {\color{red}\bfseries{}*}args).

\begin{sphinxadmonition}{note}{Notes}

When in multiprocessing, \sphinxtitleref{*args} has two elements: \sphinxtitleref{comm\_idx} and \sphinxtitleref{comm\_state}.
To pass back the state of the module, the user should insert the provided
index \sphinxtitleref{comm\_idx} and the \sphinxtitleref{state} into the queue as follows:
\begin{quote}
\begin{description}
\item[{if self.use\_multiprocessing:}] \leavevmode\begin{description}
\item[{try:}] \leavevmode
pickle.dumps(self.state)

\item[{except pickle.PicklingError:}] \leavevmode
args{[}1{]}.put((arg{[}0{]},None))

\item[{else:}] \leavevmode
args{[}1{]}.put((arg{[}0{]},self.state))

\end{description}

\end{description}
\end{quote}

at the bottom of the user defined \sphinxtitleref{run()} function.
\end{sphinxadmonition}

\begin{sphinxadmonition}{warning}{Warning:}
This function must be overridden by all Cortix modules
\end{sphinxadmonition}
\begin{quote}\begin{description}
\item[{Parameters}] \leavevmode\begin{itemize}
\item {} 
\sphinxstyleliteralstrong{\sphinxupquote{arg}}\sphinxstyleliteralstrong{\sphinxupquote{{[}}}\sphinxstyleliteralstrong{\sphinxupquote{0}}\sphinxstyleliteralstrong{\sphinxupquote{{]}}} (\sphinxhref{https://docs.python.org/3/library/functions.html\#int}{\sphinxstyleliteralemphasis{\sphinxupquote{int}}}) \textendash{} Index of the state in the communication queue.

\item {} 
\sphinxstyleliteralstrong{\sphinxupquote{arg}}\sphinxstyleliteralstrong{\sphinxupquote{{[}}}\sphinxstyleliteralstrong{\sphinxupquote{1}}\sphinxstyleliteralstrong{\sphinxupquote{{]}}} (\sphinxhref{https://docs.python.org/3/library/multiprocessing.html\#multiprocessing.Queue}{\sphinxstyleliteralemphasis{\sphinxupquote{multiprocessing.Queue}}}) \textendash{} When using the Python \sphinxtitleref{multiprocessing} library \sphinxtitleref{state\_comm} must have
the module’s \sphinxtitleref{self.state} in it. That is,
\sphinxtitleref{state\_comm.put((idx\_comm,self.state))} must be the last command in the
method before \sphinxtitleref{return}. In addition, self.state must be \sphinxtitleref{pickle-able}.

\end{itemize}

\end{description}\end{quote}

\end{fulllineitems}


\end{fulllineitems}



\section{body module}
\label{\detokenize{examples_rst/body:module-body}}\label{\detokenize{examples_rst/body:body-module}}\label{\detokenize{examples_rst/body::doc}}\index{body (module)}\index{Body (class in body)}

\begin{fulllineitems}
\phantomsection\label{\detokenize{examples_rst/body:body.Body}}\pysiglinewithargsret{\sphinxbfcode{\sphinxupquote{class }}\sphinxcode{\sphinxupquote{body.}}\sphinxbfcode{\sphinxupquote{Body}}}{\emph{mass=0}, \emph{rad=0}, \emph{pos=(0}, \emph{0}, \emph{0)}, \emph{vel=(0}, \emph{0}, \emph{0)}}{}
Bases: \sphinxcode{\sphinxupquote{cortix.src.module.Module}}
\index{acceleration() (body.Body method)}

\begin{fulllineitems}
\phantomsection\label{\detokenize{examples_rst/body:body.Body.acceleration}}\pysiglinewithargsret{\sphinxbfcode{\sphinxupquote{acceleration}}}{\emph{body}}{}
\end{fulllineitems}

\index{run() (body.Body method)}

\begin{fulllineitems}
\phantomsection\label{\detokenize{examples_rst/body:body.Body.run}}\pysiglinewithargsret{\sphinxbfcode{\sphinxupquote{run}}}{}{}
Module run function

Run method with an option to pass data back to the parent process when running
in Python multiprocessing mode. If the user does not want to share data with
the parent process, this function can be overriden with \sphinxtitleref{run(self)}
or \sphinxtitleref{run(self, *args)} as long as \sphinxtitleref{self.state = None}.
If \sphinxtitleref{self.state} points to anything but \sphinxtitleref{None}, the user must use
{\color{red}\bfseries{}{}`}run(self, {\color{red}\bfseries{}*}args).

\begin{sphinxadmonition}{note}{Notes}

When in multiprocessing, \sphinxtitleref{*args} has two elements: \sphinxtitleref{comm\_idx} and \sphinxtitleref{comm\_state}.
To pass back the state of the module, the user should insert the provided
index \sphinxtitleref{comm\_idx} and the \sphinxtitleref{state} into the queue as follows:
\begin{quote}
\begin{description}
\item[{if self.use\_multiprocessing:}] \leavevmode\begin{description}
\item[{try:}] \leavevmode
pickle.dumps(self.state)

\item[{except pickle.PicklingError:}] \leavevmode
args{[}1{]}.put((arg{[}0{]},None))

\item[{else:}] \leavevmode
args{[}1{]}.put((arg{[}0{]},self.state))

\end{description}

\end{description}
\end{quote}

at the bottom of the user defined \sphinxtitleref{run()} function.
\end{sphinxadmonition}

\begin{sphinxadmonition}{warning}{Warning:}
This function must be overridden by all Cortix modules
\end{sphinxadmonition}
\begin{quote}\begin{description}
\item[{Parameters}] \leavevmode\begin{itemize}
\item {} 
\sphinxstyleliteralstrong{\sphinxupquote{arg}}\sphinxstyleliteralstrong{\sphinxupquote{{[}}}\sphinxstyleliteralstrong{\sphinxupquote{0}}\sphinxstyleliteralstrong{\sphinxupquote{{]}}} (\sphinxhref{https://docs.python.org/3/library/functions.html\#int}{\sphinxstyleliteralemphasis{\sphinxupquote{int}}}) \textendash{} Index of the state in the communication queue.

\item {} 
\sphinxstyleliteralstrong{\sphinxupquote{arg}}\sphinxstyleliteralstrong{\sphinxupquote{{[}}}\sphinxstyleliteralstrong{\sphinxupquote{1}}\sphinxstyleliteralstrong{\sphinxupquote{{]}}} (\sphinxhref{https://docs.python.org/3/library/multiprocessing.html\#multiprocessing.Queue}{\sphinxstyleliteralemphasis{\sphinxupquote{multiprocessing.Queue}}}) \textendash{} When using the Python \sphinxtitleref{multiprocessing} library \sphinxtitleref{state\_comm} must have
the module’s \sphinxtitleref{self.state} in it. That is,
\sphinxtitleref{state\_comm.put((idx\_comm,self.state))} must be the last command in the
method before \sphinxtitleref{return}. In addition, self.state must be \sphinxtitleref{pickle-able}.

\end{itemize}

\end{description}\end{quote}

\end{fulllineitems}


\end{fulllineitems}



\section{community module}
\label{\detokenize{examples_rst/community:module-community}}\label{\detokenize{examples_rst/community:community-module}}\label{\detokenize{examples_rst/community::doc}}\index{community (module)}\index{Community (class in community)}

\begin{fulllineitems}
\phantomsection\label{\detokenize{examples_rst/community:community.Community}}\pysiglinewithargsret{\sphinxbfcode{\sphinxupquote{class }}\sphinxcode{\sphinxupquote{community.}}\sphinxbfcode{\sphinxupquote{Community}}}{\emph{n\_groups=1}, \emph{non\_offender\_adult\_population=100}, \emph{offender\_pool\_size=0.0}, \emph{free\_offender\_pool\_size=0.0}}{}
Bases: \sphinxcode{\sphinxupquote{cortix.src.module.Module}}

Community Cortix module used to model criminal group population in a community system.
Community here is the system at large with all possible adult individuals included
in a society.

\begin{sphinxadmonition}{note}{Notes}

These are the \sphinxtitleref{port} names available in this module to connect to respective
modules: \sphinxtitleref{probation}, \sphinxtitleref{adjudication}, \sphinxtitleref{jail}, \sphinxtitleref{prison}, \sphinxtitleref{arrested}, and \sphinxtitleref{parole}.
See instance attribute \sphinxtitleref{port\_names\_expected}.
\end{sphinxadmonition}
\index{\_\_init\_\_() (community.Community method)}

\begin{fulllineitems}
\phantomsection\label{\detokenize{examples_rst/community:community.Community.__init__}}\pysiglinewithargsret{\sphinxbfcode{\sphinxupquote{\_\_init\_\_}}}{\emph{n\_groups=1}, \emph{non\_offender\_adult\_population=100}, \emph{offender\_pool\_size=0.0}, \emph{free\_offender\_pool\_size=0.0}}{}~\begin{quote}\begin{description}
\item[{Parameters}] \leavevmode\begin{itemize}
\item {} 
\sphinxstyleliteralstrong{\sphinxupquote{n\_groups}} (\sphinxhref{https://docs.python.org/3/library/functions.html\#int}{\sphinxstyleliteralemphasis{\sphinxupquote{int}}}) \textendash{} Number of groups in the population.

\item {} 
\sphinxstyleliteralstrong{\sphinxupquote{non\_offender\_adult\_population}} (\sphinxhref{https://docs.python.org/3/library/functions.html\#float}{\sphinxstyleliteralemphasis{\sphinxupquote{float}}}) \textendash{} Pool of individuals reaching the adult age (SI) unit. Default: 100.

\item {} 
\sphinxstyleliteralstrong{\sphinxupquote{offender\_pool\_size}} (\sphinxhref{https://docs.python.org/3/library/functions.html\#float}{\sphinxstyleliteralemphasis{\sphinxupquote{float}}}) \textendash{} Upperbound on the range of the existing population groups. A random value
from 0 to the upperbound value will be assigned to each group. This is
typically a small number, say a fraction of a percent.

\end{itemize}

\end{description}\end{quote}

\end{fulllineitems}

\index{run() (community.Community method)}

\begin{fulllineitems}
\phantomsection\label{\detokenize{examples_rst/community:community.Community.run}}\pysiglinewithargsret{\sphinxbfcode{\sphinxupquote{run}}}{\emph{*args}}{}
Module run function

Run method with an option to pass data back to the parent process when running
in Python multiprocessing mode. If the user does not want to share data with
the parent process, this function can be overriden with \sphinxtitleref{run(self)}
or \sphinxtitleref{run(self, *args)} as long as \sphinxtitleref{self.state = None}.
If \sphinxtitleref{self.state} points to anything but \sphinxtitleref{None}, the user must use
{\color{red}\bfseries{}{}`}run(self, {\color{red}\bfseries{}*}args).

\begin{sphinxadmonition}{note}{Notes}

When in multiprocessing, \sphinxtitleref{*args} has two elements: \sphinxtitleref{comm\_idx} and \sphinxtitleref{comm\_state}.
To pass back the state of the module, the user should insert the provided
index \sphinxtitleref{comm\_idx} and the \sphinxtitleref{state} into the queue as follows:
\begin{quote}
\begin{description}
\item[{if self.use\_multiprocessing:}] \leavevmode\begin{description}
\item[{try:}] \leavevmode
pickle.dumps(self.state)

\item[{except pickle.PicklingError:}] \leavevmode
args{[}1{]}.put((arg{[}0{]},None))

\item[{else:}] \leavevmode
args{[}1{]}.put((arg{[}0{]},self.state))

\end{description}

\end{description}
\end{quote}

at the bottom of the user defined \sphinxtitleref{run()} function.
\end{sphinxadmonition}

\begin{sphinxadmonition}{warning}{Warning:}
This function must be overridden by all Cortix modules
\end{sphinxadmonition}
\begin{quote}\begin{description}
\item[{Parameters}] \leavevmode\begin{itemize}
\item {} 
\sphinxstyleliteralstrong{\sphinxupquote{arg}}\sphinxstyleliteralstrong{\sphinxupquote{{[}}}\sphinxstyleliteralstrong{\sphinxupquote{0}}\sphinxstyleliteralstrong{\sphinxupquote{{]}}} (\sphinxhref{https://docs.python.org/3/library/functions.html\#int}{\sphinxstyleliteralemphasis{\sphinxupquote{int}}}) \textendash{} Index of the state in the communication queue.

\item {} 
\sphinxstyleliteralstrong{\sphinxupquote{arg}}\sphinxstyleliteralstrong{\sphinxupquote{{[}}}\sphinxstyleliteralstrong{\sphinxupquote{1}}\sphinxstyleliteralstrong{\sphinxupquote{{]}}} (\sphinxhref{https://docs.python.org/3/library/multiprocessing.html\#multiprocessing.Queue}{\sphinxstyleliteralemphasis{\sphinxupquote{multiprocessing.Queue}}}) \textendash{} When using the Python \sphinxtitleref{multiprocessing} library \sphinxtitleref{state\_comm} must have
the module’s \sphinxtitleref{self.state} in it. That is,
\sphinxtitleref{state\_comm.put((idx\_comm,self.state))} must be the last command in the
method before \sphinxtitleref{return}. In addition, self.state must be \sphinxtitleref{pickle-able}.

\end{itemize}

\end{description}\end{quote}

\end{fulllineitems}


\end{fulllineitems}



\section{dataplot module}
\label{\detokenize{examples_rst/dataplot:module-dataplot}}\label{\detokenize{examples_rst/dataplot:dataplot-module}}\label{\detokenize{examples_rst/dataplot::doc}}\index{dataplot (module)}\index{DataPlot (class in dataplot)}

\begin{fulllineitems}
\phantomsection\label{\detokenize{examples_rst/dataplot:dataplot.DataPlot}}\pysigline{\sphinxbfcode{\sphinxupquote{class }}\sphinxcode{\sphinxupquote{dataplot.}}\sphinxbfcode{\sphinxupquote{DataPlot}}}
Bases: \sphinxcode{\sphinxupquote{cortix.src.module.Module}}
\index{plot\_data() (dataplot.DataPlot method)}

\begin{fulllineitems}
\phantomsection\label{\detokenize{examples_rst/dataplot:dataplot.DataPlot.plot_data}}\pysiglinewithargsret{\sphinxbfcode{\sphinxupquote{plot\_data}}}{}{}
\end{fulllineitems}

\index{recv\_data() (dataplot.DataPlot method)}

\begin{fulllineitems}
\phantomsection\label{\detokenize{examples_rst/dataplot:dataplot.DataPlot.recv_data}}\pysiglinewithargsret{\sphinxbfcode{\sphinxupquote{recv\_data}}}{\emph{port}}{}
Keep listening on the port and receiving data.

\end{fulllineitems}

\index{run() (dataplot.DataPlot method)}

\begin{fulllineitems}
\phantomsection\label{\detokenize{examples_rst/dataplot:dataplot.DataPlot.run}}\pysiglinewithargsret{\sphinxbfcode{\sphinxupquote{run}}}{\emph{*args}}{}
Spawn a thread to handle each port connection.

\end{fulllineitems}


\end{fulllineitems}



\section{droplet module}
\label{\detokenize{examples_rst/droplet:module-droplet}}\label{\detokenize{examples_rst/droplet:droplet-module}}\label{\detokenize{examples_rst/droplet::doc}}\index{droplet (module)}\index{Droplet (class in droplet)}

\begin{fulllineitems}
\phantomsection\label{\detokenize{examples_rst/droplet:droplet.Droplet}}\pysigline{\sphinxbfcode{\sphinxupquote{class }}\sphinxcode{\sphinxupquote{droplet.}}\sphinxbfcode{\sphinxupquote{Droplet}}}
Bases: \sphinxcode{\sphinxupquote{cortix.src.module.Module}}

Droplet Cortix module used to model very simple fluid-particle interactions.

\begin{sphinxadmonition}{note}{Notes}

Port names used in this module: \sphinxtitleref{external-flow} exchanges data with any other
module that provides information about the flow outside the droplet,
\sphinxtitleref{visualization} sends data to a visualization module.
\end{sphinxadmonition}
\index{\_\_init\_\_() (droplet.Droplet method)}

\begin{fulllineitems}
\phantomsection\label{\detokenize{examples_rst/droplet:droplet.Droplet.__init__}}\pysiglinewithargsret{\sphinxbfcode{\sphinxupquote{\_\_init\_\_}}}{}{}~\index{initial\_time (droplet.Droplet attribute)}

\begin{fulllineitems}
\phantomsection\label{\detokenize{examples_rst/droplet:droplet.Droplet.initial_time}}\pysigline{\sphinxbfcode{\sphinxupquote{initial\_time}}}
\sphinxstyleemphasis{float}

\end{fulllineitems}

\index{end\_time (droplet.Droplet attribute)}

\begin{fulllineitems}
\phantomsection\label{\detokenize{examples_rst/droplet:droplet.Droplet.end_time}}\pysigline{\sphinxbfcode{\sphinxupquote{end\_time}}}
\sphinxstyleemphasis{float}

\end{fulllineitems}

\index{time\_step (droplet.Droplet attribute)}

\begin{fulllineitems}
\phantomsection\label{\detokenize{examples_rst/droplet:droplet.Droplet.time_step}}\pysigline{\sphinxbfcode{\sphinxupquote{time\_step}}}
\sphinxstyleemphasis{float}

\end{fulllineitems}

\index{show\_time (droplet.Droplet attribute)}

\begin{fulllineitems}
\phantomsection\label{\detokenize{examples_rst/droplet:droplet.Droplet.show_time}}\pysigline{\sphinxbfcode{\sphinxupquote{show\_time}}}
\sphinxstyleemphasis{tuple} \textendash{} Two-element tuple, \sphinxtitleref{(bool,float)}, \sphinxtitleref{True} will print to standard
output.

\end{fulllineitems}


\end{fulllineitems}

\index{run() (droplet.Droplet method)}

\begin{fulllineitems}
\phantomsection\label{\detokenize{examples_rst/droplet:droplet.Droplet.run}}\pysiglinewithargsret{\sphinxbfcode{\sphinxupquote{run}}}{\emph{*args}}{}
Module run function

Run method with an option to pass data back to the parent process when running
in Python multiprocessing mode. If the user does not want to share data with
the parent process, this function can be overriden with \sphinxtitleref{run(self)}
or \sphinxtitleref{run(self, *args)} as long as \sphinxtitleref{self.state = None}.
If \sphinxtitleref{self.state} points to anything but \sphinxtitleref{None}, the user must use
{\color{red}\bfseries{}{}`}run(self, {\color{red}\bfseries{}*}args).

\begin{sphinxadmonition}{note}{Notes}

When in multiprocessing, \sphinxtitleref{*args} has two elements: \sphinxtitleref{comm\_idx} and \sphinxtitleref{comm\_state}.
To pass back the state of the module, the user should insert the provided
index \sphinxtitleref{comm\_idx} and the \sphinxtitleref{state} into the queue as follows:
\begin{quote}
\begin{description}
\item[{if self.use\_multiprocessing:}] \leavevmode\begin{description}
\item[{try:}] \leavevmode
pickle.dumps(self.state)

\item[{except pickle.PicklingError:}] \leavevmode
args{[}1{]}.put((arg{[}0{]},None))

\item[{else:}] \leavevmode
args{[}1{]}.put((arg{[}0{]},self.state))

\end{description}

\end{description}
\end{quote}

at the bottom of the user defined \sphinxtitleref{run()} function.
\end{sphinxadmonition}

\begin{sphinxadmonition}{warning}{Warning:}
This function must be overridden by all Cortix modules
\end{sphinxadmonition}
\begin{quote}\begin{description}
\item[{Parameters}] \leavevmode\begin{itemize}
\item {} 
\sphinxstyleliteralstrong{\sphinxupquote{arg}}\sphinxstyleliteralstrong{\sphinxupquote{{[}}}\sphinxstyleliteralstrong{\sphinxupquote{0}}\sphinxstyleliteralstrong{\sphinxupquote{{]}}} (\sphinxhref{https://docs.python.org/3/library/functions.html\#int}{\sphinxstyleliteralemphasis{\sphinxupquote{int}}}) \textendash{} Index of the state in the communication queue.

\item {} 
\sphinxstyleliteralstrong{\sphinxupquote{arg}}\sphinxstyleliteralstrong{\sphinxupquote{{[}}}\sphinxstyleliteralstrong{\sphinxupquote{1}}\sphinxstyleliteralstrong{\sphinxupquote{{]}}} (\sphinxhref{https://docs.python.org/3/library/multiprocessing.html\#multiprocessing.Queue}{\sphinxstyleliteralemphasis{\sphinxupquote{multiprocessing.Queue}}}) \textendash{} When using the Python \sphinxtitleref{multiprocessing} library \sphinxtitleref{state\_comm} must have
the module’s \sphinxtitleref{self.state} in it. That is,
\sphinxtitleref{state\_comm.put((idx\_comm,self.state))} must be the last command in the
method before \sphinxtitleref{return}. In addition, self.state must be \sphinxtitleref{pickle-able}.

\end{itemize}

\end{description}\end{quote}

\end{fulllineitems}


\end{fulllineitems}



\section{dummy\_module module}
\label{\detokenize{examples_rst/dummy_module:module-dummy_module}}\label{\detokenize{examples_rst/dummy_module:dummy-module-module}}\label{\detokenize{examples_rst/dummy_module::doc}}\index{dummy\_module (module)}\index{DummyModule (class in dummy\_module)}

\begin{fulllineitems}
\phantomsection\label{\detokenize{examples_rst/dummy_module:dummy_module.DummyModule}}\pysigline{\sphinxbfcode{\sphinxupquote{class }}\sphinxcode{\sphinxupquote{dummy\_module.}}\sphinxbfcode{\sphinxupquote{DummyModule}}}
Bases: \sphinxcode{\sphinxupquote{cortix.src.module.Module}}
\index{run() (dummy\_module.DummyModule method)}

\begin{fulllineitems}
\phantomsection\label{\detokenize{examples_rst/dummy_module:dummy_module.DummyModule.run}}\pysiglinewithargsret{\sphinxbfcode{\sphinxupquote{run}}}{}{}
Module run function

Run method with an option to pass data back to the parent process when running
in Python multiprocessing mode. If the user does not want to share data with
the parent process, this function can be overriden with \sphinxtitleref{run(self)}
or \sphinxtitleref{run(self, *args)} as long as \sphinxtitleref{self.state = None}.
If \sphinxtitleref{self.state} points to anything but \sphinxtitleref{None}, the user must use
{\color{red}\bfseries{}{}`}run(self, {\color{red}\bfseries{}*}args).

\begin{sphinxadmonition}{note}{Notes}

When in multiprocessing, \sphinxtitleref{*args} has two elements: \sphinxtitleref{comm\_idx} and \sphinxtitleref{comm\_state}.
To pass back the state of the module, the user should insert the provided
index \sphinxtitleref{comm\_idx} and the \sphinxtitleref{state} into the queue as follows:
\begin{quote}
\begin{description}
\item[{if self.use\_multiprocessing:}] \leavevmode\begin{description}
\item[{try:}] \leavevmode
pickle.dumps(self.state)

\item[{except pickle.PicklingError:}] \leavevmode
args{[}1{]}.put((arg{[}0{]},None))

\item[{else:}] \leavevmode
args{[}1{]}.put((arg{[}0{]},self.state))

\end{description}

\end{description}
\end{quote}

at the bottom of the user defined \sphinxtitleref{run()} function.
\end{sphinxadmonition}

\begin{sphinxadmonition}{warning}{Warning:}
This function must be overridden by all Cortix modules
\end{sphinxadmonition}
\begin{quote}\begin{description}
\item[{Parameters}] \leavevmode\begin{itemize}
\item {} 
\sphinxstyleliteralstrong{\sphinxupquote{arg}}\sphinxstyleliteralstrong{\sphinxupquote{{[}}}\sphinxstyleliteralstrong{\sphinxupquote{0}}\sphinxstyleliteralstrong{\sphinxupquote{{]}}} (\sphinxhref{https://docs.python.org/3/library/functions.html\#int}{\sphinxstyleliteralemphasis{\sphinxupquote{int}}}) \textendash{} Index of the state in the communication queue.

\item {} 
\sphinxstyleliteralstrong{\sphinxupquote{arg}}\sphinxstyleliteralstrong{\sphinxupquote{{[}}}\sphinxstyleliteralstrong{\sphinxupquote{1}}\sphinxstyleliteralstrong{\sphinxupquote{{]}}} (\sphinxhref{https://docs.python.org/3/library/multiprocessing.html\#multiprocessing.Queue}{\sphinxstyleliteralemphasis{\sphinxupquote{multiprocessing.Queue}}}) \textendash{} When using the Python \sphinxtitleref{multiprocessing} library \sphinxtitleref{state\_comm} must have
the module’s \sphinxtitleref{self.state} in it. That is,
\sphinxtitleref{state\_comm.put((idx\_comm,self.state))} must be the last command in the
method before \sphinxtitleref{return}. In addition, self.state must be \sphinxtitleref{pickle-able}.

\end{itemize}

\end{description}\end{quote}

\end{fulllineitems}


\end{fulllineitems}

\index{DummyModule2 (class in dummy\_module)}

\begin{fulllineitems}
\phantomsection\label{\detokenize{examples_rst/dummy_module:dummy_module.DummyModule2}}\pysigline{\sphinxbfcode{\sphinxupquote{class }}\sphinxcode{\sphinxupquote{dummy\_module.}}\sphinxbfcode{\sphinxupquote{DummyModule2}}}
Bases: \sphinxcode{\sphinxupquote{cortix.src.module.Module}}
\index{run() (dummy\_module.DummyModule2 method)}

\begin{fulllineitems}
\phantomsection\label{\detokenize{examples_rst/dummy_module:dummy_module.DummyModule2.run}}\pysiglinewithargsret{\sphinxbfcode{\sphinxupquote{run}}}{}{}
Module run function

Run method with an option to pass data back to the parent process when running
in Python multiprocessing mode. If the user does not want to share data with
the parent process, this function can be overriden with \sphinxtitleref{run(self)}
or \sphinxtitleref{run(self, *args)} as long as \sphinxtitleref{self.state = None}.
If \sphinxtitleref{self.state} points to anything but \sphinxtitleref{None}, the user must use
{\color{red}\bfseries{}{}`}run(self, {\color{red}\bfseries{}*}args).

\begin{sphinxadmonition}{note}{Notes}

When in multiprocessing, \sphinxtitleref{*args} has two elements: \sphinxtitleref{comm\_idx} and \sphinxtitleref{comm\_state}.
To pass back the state of the module, the user should insert the provided
index \sphinxtitleref{comm\_idx} and the \sphinxtitleref{state} into the queue as follows:
\begin{quote}
\begin{description}
\item[{if self.use\_multiprocessing:}] \leavevmode\begin{description}
\item[{try:}] \leavevmode
pickle.dumps(self.state)

\item[{except pickle.PicklingError:}] \leavevmode
args{[}1{]}.put((arg{[}0{]},None))

\item[{else:}] \leavevmode
args{[}1{]}.put((arg{[}0{]},self.state))

\end{description}

\end{description}
\end{quote}

at the bottom of the user defined \sphinxtitleref{run()} function.
\end{sphinxadmonition}

\begin{sphinxadmonition}{warning}{Warning:}
This function must be overridden by all Cortix modules
\end{sphinxadmonition}
\begin{quote}\begin{description}
\item[{Parameters}] \leavevmode\begin{itemize}
\item {} 
\sphinxstyleliteralstrong{\sphinxupquote{arg}}\sphinxstyleliteralstrong{\sphinxupquote{{[}}}\sphinxstyleliteralstrong{\sphinxupquote{0}}\sphinxstyleliteralstrong{\sphinxupquote{{]}}} (\sphinxhref{https://docs.python.org/3/library/functions.html\#int}{\sphinxstyleliteralemphasis{\sphinxupquote{int}}}) \textendash{} Index of the state in the communication queue.

\item {} 
\sphinxstyleliteralstrong{\sphinxupquote{arg}}\sphinxstyleliteralstrong{\sphinxupquote{{[}}}\sphinxstyleliteralstrong{\sphinxupquote{1}}\sphinxstyleliteralstrong{\sphinxupquote{{]}}} (\sphinxhref{https://docs.python.org/3/library/multiprocessing.html\#multiprocessing.Queue}{\sphinxstyleliteralemphasis{\sphinxupquote{multiprocessing.Queue}}}) \textendash{} When using the Python \sphinxtitleref{multiprocessing} library \sphinxtitleref{state\_comm} must have
the module’s \sphinxtitleref{self.state} in it. That is,
\sphinxtitleref{state\_comm.put((idx\_comm,self.state))} must be the last command in the
method before \sphinxtitleref{return}. In addition, self.state must be \sphinxtitleref{pickle-able}.

\end{itemize}

\end{description}\end{quote}

\end{fulllineitems}


\end{fulllineitems}



\section{jail module}
\label{\detokenize{examples_rst/jail:module-jail}}\label{\detokenize{examples_rst/jail:jail-module}}\label{\detokenize{examples_rst/jail::doc}}\index{jail (module)}\index{Jail (class in jail)}

\begin{fulllineitems}
\phantomsection\label{\detokenize{examples_rst/jail:jail.Jail}}\pysiglinewithargsret{\sphinxbfcode{\sphinxupquote{class }}\sphinxcode{\sphinxupquote{jail.}}\sphinxbfcode{\sphinxupquote{Jail}}}{\emph{n\_groups=1}, \emph{pool\_size=0.0}}{}
Bases: \sphinxcode{\sphinxupquote{cortix.src.module.Module}}

Jail Cortix module used to model criminal group population in a jail.

\begin{sphinxadmonition}{note}{Notes}

These are the \sphinxtitleref{port} names available in this module to connect to respective
modules: \sphinxtitleref{probation}, \sphinxtitleref{adjudication}, \sphinxtitleref{arrested}, \sphinxtitleref{prison}, and \sphinxtitleref{community}.
See instance attribute \sphinxtitleref{port\_names\_expected}.
\end{sphinxadmonition}
\index{\_\_init\_\_() (jail.Jail method)}

\begin{fulllineitems}
\phantomsection\label{\detokenize{examples_rst/jail:jail.Jail.__init__}}\pysiglinewithargsret{\sphinxbfcode{\sphinxupquote{\_\_init\_\_}}}{\emph{n\_groups=1}, \emph{pool\_size=0.0}}{}~\begin{quote}\begin{description}
\item[{Parameters}] \leavevmode\begin{itemize}
\item {} 
\sphinxstyleliteralstrong{\sphinxupquote{n\_groups}} (\sphinxhref{https://docs.python.org/3/library/functions.html\#int}{\sphinxstyleliteralemphasis{\sphinxupquote{int}}}) \textendash{} Number of groups in the population.

\item {} 
\sphinxstyleliteralstrong{\sphinxupquote{pool\_size}} (\sphinxhref{https://docs.python.org/3/library/functions.html\#float}{\sphinxstyleliteralemphasis{\sphinxupquote{float}}}) \textendash{} Upperbound on the range of the existing population groups. A random value
from 0 to the upperbound value will be assigned to each group.

\end{itemize}

\end{description}\end{quote}

\end{fulllineitems}

\index{run() (jail.Jail method)}

\begin{fulllineitems}
\phantomsection\label{\detokenize{examples_rst/jail:jail.Jail.run}}\pysiglinewithargsret{\sphinxbfcode{\sphinxupquote{run}}}{\emph{*args}}{}
Module run function

Run method with an option to pass data back to the parent process when running
in Python multiprocessing mode. If the user does not want to share data with
the parent process, this function can be overriden with \sphinxtitleref{run(self)}
or \sphinxtitleref{run(self, *args)} as long as \sphinxtitleref{self.state = None}.
If \sphinxtitleref{self.state} points to anything but \sphinxtitleref{None}, the user must use
{\color{red}\bfseries{}{}`}run(self, {\color{red}\bfseries{}*}args).

\begin{sphinxadmonition}{note}{Notes}

When in multiprocessing, \sphinxtitleref{*args} has two elements: \sphinxtitleref{comm\_idx} and \sphinxtitleref{comm\_state}.
To pass back the state of the module, the user should insert the provided
index \sphinxtitleref{comm\_idx} and the \sphinxtitleref{state} into the queue as follows:
\begin{quote}
\begin{description}
\item[{if self.use\_multiprocessing:}] \leavevmode\begin{description}
\item[{try:}] \leavevmode
pickle.dumps(self.state)

\item[{except pickle.PicklingError:}] \leavevmode
args{[}1{]}.put((arg{[}0{]},None))

\item[{else:}] \leavevmode
args{[}1{]}.put((arg{[}0{]},self.state))

\end{description}

\end{description}
\end{quote}

at the bottom of the user defined \sphinxtitleref{run()} function.
\end{sphinxadmonition}

\begin{sphinxadmonition}{warning}{Warning:}
This function must be overridden by all Cortix modules
\end{sphinxadmonition}
\begin{quote}\begin{description}
\item[{Parameters}] \leavevmode\begin{itemize}
\item {} 
\sphinxstyleliteralstrong{\sphinxupquote{arg}}\sphinxstyleliteralstrong{\sphinxupquote{{[}}}\sphinxstyleliteralstrong{\sphinxupquote{0}}\sphinxstyleliteralstrong{\sphinxupquote{{]}}} (\sphinxhref{https://docs.python.org/3/library/functions.html\#int}{\sphinxstyleliteralemphasis{\sphinxupquote{int}}}) \textendash{} Index of the state in the communication queue.

\item {} 
\sphinxstyleliteralstrong{\sphinxupquote{arg}}\sphinxstyleliteralstrong{\sphinxupquote{{[}}}\sphinxstyleliteralstrong{\sphinxupquote{1}}\sphinxstyleliteralstrong{\sphinxupquote{{]}}} (\sphinxhref{https://docs.python.org/3/library/multiprocessing.html\#multiprocessing.Queue}{\sphinxstyleliteralemphasis{\sphinxupquote{multiprocessing.Queue}}}) \textendash{} When using the Python \sphinxtitleref{multiprocessing} library \sphinxtitleref{state\_comm} must have
the module’s \sphinxtitleref{self.state} in it. That is,
\sphinxtitleref{state\_comm.put((idx\_comm,self.state))} must be the last command in the
method before \sphinxtitleref{return}. In addition, self.state must be \sphinxtitleref{pickle-able}.

\end{itemize}

\end{description}\end{quote}

\end{fulllineitems}


\end{fulllineitems}



\section{parole module}
\label{\detokenize{examples_rst/parole:module-parole}}\label{\detokenize{examples_rst/parole:parole-module}}\label{\detokenize{examples_rst/parole::doc}}\index{parole (module)}\index{Parole (class in parole)}

\begin{fulllineitems}
\phantomsection\label{\detokenize{examples_rst/parole:parole.Parole}}\pysiglinewithargsret{\sphinxbfcode{\sphinxupquote{class }}\sphinxcode{\sphinxupquote{parole.}}\sphinxbfcode{\sphinxupquote{Parole}}}{\emph{n\_groups=1}, \emph{pool\_size=0.0}}{}
Bases: \sphinxcode{\sphinxupquote{cortix.src.module.Module}}

Parole Cortix module used to model criminal group population in a parole system.

\begin{sphinxadmonition}{note}{Notes}

These are the \sphinxtitleref{port} names available in this module to connect to respective
modules: \sphinxtitleref{prison} and \sphinxtitleref{community}.
See instance attribute \sphinxtitleref{port\_names\_expected}.
\end{sphinxadmonition}
\index{run() (parole.Parole method)}

\begin{fulllineitems}
\phantomsection\label{\detokenize{examples_rst/parole:parole.Parole.run}}\pysiglinewithargsret{\sphinxbfcode{\sphinxupquote{run}}}{\emph{*args}}{}
Module run function

Run method with an option to pass data back to the parent process when running
in Python multiprocessing mode. If the user does not want to share data with
the parent process, this function can be overriden with \sphinxtitleref{run(self)}
or \sphinxtitleref{run(self, *args)} as long as \sphinxtitleref{self.state = None}.
If \sphinxtitleref{self.state} points to anything but \sphinxtitleref{None}, the user must use
{\color{red}\bfseries{}{}`}run(self, {\color{red}\bfseries{}*}args).

\begin{sphinxadmonition}{note}{Notes}

When in multiprocessing, \sphinxtitleref{*args} has two elements: \sphinxtitleref{comm\_idx} and \sphinxtitleref{comm\_state}.
To pass back the state of the module, the user should insert the provided
index \sphinxtitleref{comm\_idx} and the \sphinxtitleref{state} into the queue as follows:
\begin{quote}
\begin{description}
\item[{if self.use\_multiprocessing:}] \leavevmode\begin{description}
\item[{try:}] \leavevmode
pickle.dumps(self.state)

\item[{except pickle.PicklingError:}] \leavevmode
args{[}1{]}.put((arg{[}0{]},None))

\item[{else:}] \leavevmode
args{[}1{]}.put((arg{[}0{]},self.state))

\end{description}

\end{description}
\end{quote}

at the bottom of the user defined \sphinxtitleref{run()} function.
\end{sphinxadmonition}

\begin{sphinxadmonition}{warning}{Warning:}
This function must be overridden by all Cortix modules
\end{sphinxadmonition}
\begin{quote}\begin{description}
\item[{Parameters}] \leavevmode\begin{itemize}
\item {} 
\sphinxstyleliteralstrong{\sphinxupquote{arg}}\sphinxstyleliteralstrong{\sphinxupquote{{[}}}\sphinxstyleliteralstrong{\sphinxupquote{0}}\sphinxstyleliteralstrong{\sphinxupquote{{]}}} (\sphinxhref{https://docs.python.org/3/library/functions.html\#int}{\sphinxstyleliteralemphasis{\sphinxupquote{int}}}) \textendash{} Index of the state in the communication queue.

\item {} 
\sphinxstyleliteralstrong{\sphinxupquote{arg}}\sphinxstyleliteralstrong{\sphinxupquote{{[}}}\sphinxstyleliteralstrong{\sphinxupquote{1}}\sphinxstyleliteralstrong{\sphinxupquote{{]}}} (\sphinxhref{https://docs.python.org/3/library/multiprocessing.html\#multiprocessing.Queue}{\sphinxstyleliteralemphasis{\sphinxupquote{multiprocessing.Queue}}}) \textendash{} When using the Python \sphinxtitleref{multiprocessing} library \sphinxtitleref{state\_comm} must have
the module’s \sphinxtitleref{self.state} in it. That is,
\sphinxtitleref{state\_comm.put((idx\_comm,self.state))} must be the last command in the
method before \sphinxtitleref{return}. In addition, self.state must be \sphinxtitleref{pickle-able}.

\end{itemize}

\end{description}\end{quote}

\end{fulllineitems}


\end{fulllineitems}



\section{plot\_data module}
\label{\detokenize{examples_rst/plot_data:module-plot_data}}\label{\detokenize{examples_rst/plot_data:plot-data-module}}\label{\detokenize{examples_rst/plot_data::doc}}\index{plot\_data (module)}\index{PlotData (class in plot\_data)}

\begin{fulllineitems}
\phantomsection\label{\detokenize{examples_rst/plot_data:plot_data.PlotData}}\pysigline{\sphinxbfcode{\sphinxupquote{class }}\sphinxcode{\sphinxupquote{plot\_data.}}\sphinxbfcode{\sphinxupquote{PlotData}}}
Bases: \sphinxcode{\sphinxupquote{cortix.src.module.Module}}
\index{run() (plot\_data.PlotData method)}

\begin{fulllineitems}
\phantomsection\label{\detokenize{examples_rst/plot_data:plot_data.PlotData.run}}\pysiglinewithargsret{\sphinxbfcode{\sphinxupquote{run}}}{}{}
Module run function

Run method with an option to pass data back to the parent process when running
in Python multiprocessing mode. If the user does not want to share data with
the parent process, this function can be overriden with \sphinxtitleref{run(self)}
or \sphinxtitleref{run(self, *args)} as long as \sphinxtitleref{self.state = None}.
If \sphinxtitleref{self.state} points to anything but \sphinxtitleref{None}, the user must use
{\color{red}\bfseries{}{}`}run(self, {\color{red}\bfseries{}*}args).

\begin{sphinxadmonition}{note}{Notes}

When in multiprocessing, \sphinxtitleref{*args} has two elements: \sphinxtitleref{comm\_idx} and \sphinxtitleref{comm\_state}.
To pass back the state of the module, the user should insert the provided
index \sphinxtitleref{comm\_idx} and the \sphinxtitleref{state} into the queue as follows:
\begin{quote}
\begin{description}
\item[{if self.use\_multiprocessing:}] \leavevmode\begin{description}
\item[{try:}] \leavevmode
pickle.dumps(self.state)

\item[{except pickle.PicklingError:}] \leavevmode
args{[}1{]}.put((arg{[}0{]},None))

\item[{else:}] \leavevmode
args{[}1{]}.put((arg{[}0{]},self.state))

\end{description}

\end{description}
\end{quote}

at the bottom of the user defined \sphinxtitleref{run()} function.
\end{sphinxadmonition}

\begin{sphinxadmonition}{warning}{Warning:}
This function must be overridden by all Cortix modules
\end{sphinxadmonition}
\begin{quote}\begin{description}
\item[{Parameters}] \leavevmode\begin{itemize}
\item {} 
\sphinxstyleliteralstrong{\sphinxupquote{arg}}\sphinxstyleliteralstrong{\sphinxupquote{{[}}}\sphinxstyleliteralstrong{\sphinxupquote{0}}\sphinxstyleliteralstrong{\sphinxupquote{{]}}} (\sphinxhref{https://docs.python.org/3/library/functions.html\#int}{\sphinxstyleliteralemphasis{\sphinxupquote{int}}}) \textendash{} Index of the state in the communication queue.

\item {} 
\sphinxstyleliteralstrong{\sphinxupquote{arg}}\sphinxstyleliteralstrong{\sphinxupquote{{[}}}\sphinxstyleliteralstrong{\sphinxupquote{1}}\sphinxstyleliteralstrong{\sphinxupquote{{]}}} (\sphinxhref{https://docs.python.org/3/library/multiprocessing.html\#multiprocessing.Queue}{\sphinxstyleliteralemphasis{\sphinxupquote{multiprocessing.Queue}}}) \textendash{} When using the Python \sphinxtitleref{multiprocessing} library \sphinxtitleref{state\_comm} must have
the module’s \sphinxtitleref{self.state} in it. That is,
\sphinxtitleref{state\_comm.put((idx\_comm,self.state))} must be the last command in the
method before \sphinxtitleref{return}. In addition, self.state must be \sphinxtitleref{pickle-able}.

\end{itemize}

\end{description}\end{quote}

\end{fulllineitems}


\end{fulllineitems}



\section{prison module}
\label{\detokenize{examples_rst/prison:module-prison}}\label{\detokenize{examples_rst/prison:prison-module}}\label{\detokenize{examples_rst/prison::doc}}\index{prison (module)}\index{Prison (class in prison)}

\begin{fulllineitems}
\phantomsection\label{\detokenize{examples_rst/prison:prison.Prison}}\pysiglinewithargsret{\sphinxbfcode{\sphinxupquote{class }}\sphinxcode{\sphinxupquote{prison.}}\sphinxbfcode{\sphinxupquote{Prison}}}{\emph{n\_groups=1}, \emph{pool\_size=0.0}}{}
Bases: \sphinxcode{\sphinxupquote{cortix.src.module.Module}}

Prison Cortix module used to model criminal group population in a prison.

\begin{sphinxadmonition}{note}{Notes}

These are the \sphinxtitleref{port} names available in this module to connect to respective
modules: \sphinxtitleref{parole}, \sphinxtitleref{adjudication}, \sphinxtitleref{jail}, and \sphinxtitleref{community}.
See instance attribute \sphinxtitleref{port\_names\_expected}.
\end{sphinxadmonition}
\index{\_\_init\_\_() (prison.Prison method)}

\begin{fulllineitems}
\phantomsection\label{\detokenize{examples_rst/prison:prison.Prison.__init__}}\pysiglinewithargsret{\sphinxbfcode{\sphinxupquote{\_\_init\_\_}}}{\emph{n\_groups=1}, \emph{pool\_size=0.0}}{}~\begin{quote}\begin{description}
\item[{Parameters}] \leavevmode\begin{itemize}
\item {} 
\sphinxstyleliteralstrong{\sphinxupquote{n\_groups}} (\sphinxhref{https://docs.python.org/3/library/functions.html\#int}{\sphinxstyleliteralemphasis{\sphinxupquote{int}}}) \textendash{} Number of groups in the population.

\item {} 
\sphinxstyleliteralstrong{\sphinxupquote{pool\_size}} (\sphinxhref{https://docs.python.org/3/library/functions.html\#float}{\sphinxstyleliteralemphasis{\sphinxupquote{float}}}) \textendash{} Upperbound on the range of the existing population groups. A random value
from 0 to the upperbound value will be assigned to each group.

\end{itemize}

\end{description}\end{quote}

\end{fulllineitems}

\index{run() (prison.Prison method)}

\begin{fulllineitems}
\phantomsection\label{\detokenize{examples_rst/prison:prison.Prison.run}}\pysiglinewithargsret{\sphinxbfcode{\sphinxupquote{run}}}{\emph{*args}}{}
Module run function

Run method with an option to pass data back to the parent process when running
in Python multiprocessing mode. If the user does not want to share data with
the parent process, this function can be overriden with \sphinxtitleref{run(self)}
or \sphinxtitleref{run(self, *args)} as long as \sphinxtitleref{self.state = None}.
If \sphinxtitleref{self.state} points to anything but \sphinxtitleref{None}, the user must use
{\color{red}\bfseries{}{}`}run(self, {\color{red}\bfseries{}*}args).

\begin{sphinxadmonition}{note}{Notes}

When in multiprocessing, \sphinxtitleref{*args} has two elements: \sphinxtitleref{comm\_idx} and \sphinxtitleref{comm\_state}.
To pass back the state of the module, the user should insert the provided
index \sphinxtitleref{comm\_idx} and the \sphinxtitleref{state} into the queue as follows:
\begin{quote}
\begin{description}
\item[{if self.use\_multiprocessing:}] \leavevmode\begin{description}
\item[{try:}] \leavevmode
pickle.dumps(self.state)

\item[{except pickle.PicklingError:}] \leavevmode
args{[}1{]}.put((arg{[}0{]},None))

\item[{else:}] \leavevmode
args{[}1{]}.put((arg{[}0{]},self.state))

\end{description}

\end{description}
\end{quote}

at the bottom of the user defined \sphinxtitleref{run()} function.
\end{sphinxadmonition}

\begin{sphinxadmonition}{warning}{Warning:}
This function must be overridden by all Cortix modules
\end{sphinxadmonition}
\begin{quote}\begin{description}
\item[{Parameters}] \leavevmode\begin{itemize}
\item {} 
\sphinxstyleliteralstrong{\sphinxupquote{arg}}\sphinxstyleliteralstrong{\sphinxupquote{{[}}}\sphinxstyleliteralstrong{\sphinxupquote{0}}\sphinxstyleliteralstrong{\sphinxupquote{{]}}} (\sphinxhref{https://docs.python.org/3/library/functions.html\#int}{\sphinxstyleliteralemphasis{\sphinxupquote{int}}}) \textendash{} Index of the state in the communication queue.

\item {} 
\sphinxstyleliteralstrong{\sphinxupquote{arg}}\sphinxstyleliteralstrong{\sphinxupquote{{[}}}\sphinxstyleliteralstrong{\sphinxupquote{1}}\sphinxstyleliteralstrong{\sphinxupquote{{]}}} (\sphinxhref{https://docs.python.org/3/library/multiprocessing.html\#multiprocessing.Queue}{\sphinxstyleliteralemphasis{\sphinxupquote{multiprocessing.Queue}}}) \textendash{} When using the Python \sphinxtitleref{multiprocessing} library \sphinxtitleref{state\_comm} must have
the module’s \sphinxtitleref{self.state} in it. That is,
\sphinxtitleref{state\_comm.put((idx\_comm,self.state))} must be the last command in the
method before \sphinxtitleref{return}. In addition, self.state must be \sphinxtitleref{pickle-able}.

\end{itemize}

\end{description}\end{quote}

\end{fulllineitems}


\end{fulllineitems}



\section{probation module}
\label{\detokenize{examples_rst/probation:module-probation}}\label{\detokenize{examples_rst/probation:probation-module}}\label{\detokenize{examples_rst/probation::doc}}\index{probation (module)}\index{Probation (class in probation)}

\begin{fulllineitems}
\phantomsection\label{\detokenize{examples_rst/probation:probation.Probation}}\pysiglinewithargsret{\sphinxbfcode{\sphinxupquote{class }}\sphinxcode{\sphinxupquote{probation.}}\sphinxbfcode{\sphinxupquote{Probation}}}{\emph{n\_groups=1}, \emph{pool\_size=0.0}}{}
Bases: \sphinxcode{\sphinxupquote{cortix.src.module.Module}}

Probation Cortix module used to model criminal group population in a probation.

\begin{sphinxadmonition}{note}{Notes}

These are the \sphinxtitleref{port} names available in this module to connect to respective
modules: \sphinxtitleref{adjudication}, \sphinxtitleref{jail}, \sphinxtitleref{arrested}, and \sphinxtitleref{community}.
See instance attribute \sphinxtitleref{port\_names\_expected}.
\end{sphinxadmonition}
\index{\_\_init\_\_() (probation.Probation method)}

\begin{fulllineitems}
\phantomsection\label{\detokenize{examples_rst/probation:probation.Probation.__init__}}\pysiglinewithargsret{\sphinxbfcode{\sphinxupquote{\_\_init\_\_}}}{\emph{n\_groups=1}, \emph{pool\_size=0.0}}{}~\begin{quote}\begin{description}
\item[{Parameters}] \leavevmode\begin{itemize}
\item {} 
\sphinxstyleliteralstrong{\sphinxupquote{n\_groups}} (\sphinxhref{https://docs.python.org/3/library/functions.html\#int}{\sphinxstyleliteralemphasis{\sphinxupquote{int}}}) \textendash{} Number of groups in the population.

\item {} 
\sphinxstyleliteralstrong{\sphinxupquote{pool\_size}} (\sphinxhref{https://docs.python.org/3/library/functions.html\#float}{\sphinxstyleliteralemphasis{\sphinxupquote{float}}}) \textendash{} Upperbound on the range of the existing population groups. A random value
from 0 to the upperbound value will be assigned to each group.

\end{itemize}

\end{description}\end{quote}

\end{fulllineitems}

\index{run() (probation.Probation method)}

\begin{fulllineitems}
\phantomsection\label{\detokenize{examples_rst/probation:probation.Probation.run}}\pysiglinewithargsret{\sphinxbfcode{\sphinxupquote{run}}}{\emph{*args}}{}
Module run function

Run method with an option to pass data back to the parent process when running
in Python multiprocessing mode. If the user does not want to share data with
the parent process, this function can be overriden with \sphinxtitleref{run(self)}
or \sphinxtitleref{run(self, *args)} as long as \sphinxtitleref{self.state = None}.
If \sphinxtitleref{self.state} points to anything but \sphinxtitleref{None}, the user must use
{\color{red}\bfseries{}{}`}run(self, {\color{red}\bfseries{}*}args).

\begin{sphinxadmonition}{note}{Notes}

When in multiprocessing, \sphinxtitleref{*args} has two elements: \sphinxtitleref{comm\_idx} and \sphinxtitleref{comm\_state}.
To pass back the state of the module, the user should insert the provided
index \sphinxtitleref{comm\_idx} and the \sphinxtitleref{state} into the queue as follows:
\begin{quote}
\begin{description}
\item[{if self.use\_multiprocessing:}] \leavevmode\begin{description}
\item[{try:}] \leavevmode
pickle.dumps(self.state)

\item[{except pickle.PicklingError:}] \leavevmode
args{[}1{]}.put((arg{[}0{]},None))

\item[{else:}] \leavevmode
args{[}1{]}.put((arg{[}0{]},self.state))

\end{description}

\end{description}
\end{quote}

at the bottom of the user defined \sphinxtitleref{run()} function.
\end{sphinxadmonition}

\begin{sphinxadmonition}{warning}{Warning:}
This function must be overridden by all Cortix modules
\end{sphinxadmonition}
\begin{quote}\begin{description}
\item[{Parameters}] \leavevmode\begin{itemize}
\item {} 
\sphinxstyleliteralstrong{\sphinxupquote{arg}}\sphinxstyleliteralstrong{\sphinxupquote{{[}}}\sphinxstyleliteralstrong{\sphinxupquote{0}}\sphinxstyleliteralstrong{\sphinxupquote{{]}}} (\sphinxhref{https://docs.python.org/3/library/functions.html\#int}{\sphinxstyleliteralemphasis{\sphinxupquote{int}}}) \textendash{} Index of the state in the communication queue.

\item {} 
\sphinxstyleliteralstrong{\sphinxupquote{arg}}\sphinxstyleliteralstrong{\sphinxupquote{{[}}}\sphinxstyleliteralstrong{\sphinxupquote{1}}\sphinxstyleliteralstrong{\sphinxupquote{{]}}} (\sphinxhref{https://docs.python.org/3/library/multiprocessing.html\#multiprocessing.Queue}{\sphinxstyleliteralemphasis{\sphinxupquote{multiprocessing.Queue}}}) \textendash{} When using the Python \sphinxtitleref{multiprocessing} library \sphinxtitleref{state\_comm} must have
the module’s \sphinxtitleref{self.state} in it. That is,
\sphinxtitleref{state\_comm.put((idx\_comm,self.state))} must be the last command in the
method before \sphinxtitleref{return}. In addition, self.state must be \sphinxtitleref{pickle-able}.

\end{itemize}

\end{description}\end{quote}

\end{fulllineitems}


\end{fulllineitems}



\section{run\_city\_justice module}
\label{\detokenize{examples_rst/run_city_justice:module-run_city_justice}}\label{\detokenize{examples_rst/run_city_justice:run-city-justice-module}}\label{\detokenize{examples_rst/run_city_justice::doc}}\index{run\_city\_justice (module)}
Crimninal justice network dynamics modeling.
\begin{description}
\item[{This example uses 7 modules:}] \leavevmode\begin{itemize}
\item {} 
Community

\item {} 
Arrested

\item {} 
Adjudication

\item {} 
Jail

\item {} 
Prison

\item {} 
Probation

\item {} 
Parole

\end{itemize}

\end{description}

and a population balance model is used to follow the offenders population
groups between modules.

To run this case using MPI you should compute the number of
processes as follows:
\begin{quote}

\sphinxtitleref{nprocs = 7 + 1 cortix}
\end{quote}

then issue the MPI run command as follows (replace \sphinxtitleref{nprocs} with a number):
\begin{quote}

\sphinxtitleref{mpiexec -n nprocs run\_justice.py}
\end{quote}

To run this case with the Python multiprocessing library, just run this file at the
command line as
\begin{quote}

\sphinxtitleref{run\_city\_justice.py}
\end{quote}
\index{main() (in module run\_city\_justice)}

\begin{fulllineitems}
\phantomsection\label{\detokenize{examples_rst/run_city_justice:run_city_justice.main}}\pysiglinewithargsret{\sphinxcode{\sphinxupquote{run\_city\_justice.}}\sphinxbfcode{\sphinxupquote{main}}}{}{}
Cortix run file for a criminal justice network.
\index{n\_groups (in module run\_city\_justice)}

\begin{fulllineitems}
\phantomsection\label{\detokenize{examples_rst/run_city_justice:run_city_justice.n_groups}}\pysigline{\sphinxcode{\sphinxupquote{run\_city\_justice.}}\sphinxbfcode{\sphinxupquote{n\_groups}}}
\sphinxstyleemphasis{int} \textendash{} Number of population groups being followed. This must be the same for all
modules.

\end{fulllineitems}

\index{end\_time (in module run\_city\_justice)}

\begin{fulllineitems}
\phantomsection\label{\detokenize{examples_rst/run_city_justice:run_city_justice.end_time}}\pysigline{\sphinxcode{\sphinxupquote{run\_city\_justice.}}\sphinxbfcode{\sphinxupquote{end\_time}}}
\sphinxstyleemphasis{float} \textendash{} End of the flow time in SI unit.

\end{fulllineitems}

\index{time\_step (in module run\_city\_justice)}

\begin{fulllineitems}
\phantomsection\label{\detokenize{examples_rst/run_city_justice:run_city_justice.time_step}}\pysigline{\sphinxcode{\sphinxupquote{run\_city\_justice.}}\sphinxbfcode{\sphinxupquote{time\_step}}}
\sphinxstyleemphasis{float} \textendash{} Size of the time step between port communications in SI unit.

\end{fulllineitems}

\index{use\_mpi (in module run\_city\_justice)}

\begin{fulllineitems}
\phantomsection\label{\detokenize{examples_rst/run_city_justice:run_city_justice.use_mpi}}\pysigline{\sphinxcode{\sphinxupquote{run\_city\_justice.}}\sphinxbfcode{\sphinxupquote{use\_mpi}}}
\sphinxstyleemphasis{bool} \textendash{} If set to \sphinxtitleref{True} use MPI otherwise use Python multiprocessing.

\end{fulllineitems}


\end{fulllineitems}



\section{run\_droplet\_swirl module}
\label{\detokenize{examples_rst/run_droplet_swirl:module-run_droplet_swirl}}\label{\detokenize{examples_rst/run_droplet_swirl:run-droplet-swirl-module}}\label{\detokenize{examples_rst/run_droplet_swirl::doc}}\index{run\_droplet\_swirl (module)}
This example uses two modules instantiated many times. It be executed with MPI
(if \sphinxtitleref{mpi4py} is available) or with the Python multiprocessing library. These choices
are made by variables listed below in the executable portion of this run file.

To run this case using MPI you should compute the number of
processes as follows:
\begin{quote}

\sphinxtitleref{nprocs = n\_droplets + 1 vortex + 1 cortix}
\end{quote}

then issue the MPI run command as follows (replace \sphinxtitleref{nprocs} with a number):
\begin{quote}

\sphinxtitleref{mpiexec -n nprocs run\_droplet.py}
\end{quote}

To run this case with the Python multiprocessing library, just run this file at the
command line as
\begin{quote}

\sphinxtitleref{run\_droplet.py}
\end{quote}
\index{main() (in module run\_droplet\_swirl)}

\begin{fulllineitems}
\phantomsection\label{\detokenize{examples_rst/run_droplet_swirl:run_droplet_swirl.main}}\pysiglinewithargsret{\sphinxcode{\sphinxupquote{run\_droplet\_swirl.}}\sphinxbfcode{\sphinxupquote{main}}}{}{}
Cortix run file for a \sphinxtitleref{Droplet}-\sphinxtitleref{Vortex} network.
\index{n\_droplets (in module run\_droplet\_swirl)}

\begin{fulllineitems}
\phantomsection\label{\detokenize{examples_rst/run_droplet_swirl:run_droplet_swirl.n_droplets}}\pysigline{\sphinxcode{\sphinxupquote{run\_droplet\_swirl.}}\sphinxbfcode{\sphinxupquote{n\_droplets}}}
\sphinxstyleemphasis{int} \textendash{} Number of droplets to use (one per process).

\end{fulllineitems}

\index{end\_time (in module run\_droplet\_swirl)}

\begin{fulllineitems}
\phantomsection\label{\detokenize{examples_rst/run_droplet_swirl:run_droplet_swirl.end_time}}\pysigline{\sphinxcode{\sphinxupquote{run\_droplet\_swirl.}}\sphinxbfcode{\sphinxupquote{end\_time}}}
\sphinxstyleemphasis{float} \textendash{} End of the flow time in SI unit.

\end{fulllineitems}

\index{time\_step (in module run\_droplet\_swirl)}

\begin{fulllineitems}
\phantomsection\label{\detokenize{examples_rst/run_droplet_swirl:run_droplet_swirl.time_step}}\pysigline{\sphinxcode{\sphinxupquote{run\_droplet\_swirl.}}\sphinxbfcode{\sphinxupquote{time\_step}}}
\sphinxstyleemphasis{float} \textendash{} Size of the time step between port communications in SI unit.

\end{fulllineitems}

\index{create\_plots (in module run\_droplet\_swirl)}

\begin{fulllineitems}
\phantomsection\label{\detokenize{examples_rst/run_droplet_swirl:run_droplet_swirl.create_plots}}\pysigline{\sphinxcode{\sphinxupquote{run\_droplet\_swirl.}}\sphinxbfcode{\sphinxupquote{create\_plots}}}
\sphinxstyleemphasis{bool} \textendash{} Create various plots and save to files. (all data collected in the
parent process; it may run out of memory).

\end{fulllineitems}

\index{plot\_vortex\_profile (in module run\_droplet\_swirl)}

\begin{fulllineitems}
\phantomsection\label{\detokenize{examples_rst/run_droplet_swirl:run_droplet_swirl.plot_vortex_profile}}\pysigline{\sphinxcode{\sphinxupquote{run\_droplet\_swirl.}}\sphinxbfcode{\sphinxupquote{plot\_vortex\_profile}}}
\sphinxstyleemphasis{bool} \textendash{} Whether to plot (to a file) the vortex function used.

\end{fulllineitems}

\index{use\_mpi (in module run\_droplet\_swirl)}

\begin{fulllineitems}
\phantomsection\label{\detokenize{examples_rst/run_droplet_swirl:run_droplet_swirl.use_mpi}}\pysigline{\sphinxcode{\sphinxupquote{run\_droplet\_swirl.}}\sphinxbfcode{\sphinxupquote{use\_mpi}}}
\sphinxstyleemphasis{bool} \textendash{} If set to \sphinxtitleref{True} use MPI otherwise use Python multiprocessing.

\end{fulllineitems}


\end{fulllineitems}



\section{run\_droplet\_test module}
\label{\detokenize{examples_rst/run_droplet_test:module-run_droplet_test}}\label{\detokenize{examples_rst/run_droplet_test:run-droplet-test-module}}\label{\detokenize{examples_rst/run_droplet_test::doc}}\index{run\_droplet\_test (module)}
This example uses three modules instantiated many times in two different networks.
Each network configuration uses a different amount of module instances and a different
network topology. This example can be executed with MPI (if mpi4py is available) or
with the Python multiprocessing library. These choices are made by variables listed
below in the executable portion of this run file.


\subsection{Single Plot}
\label{\detokenize{examples_rst/run_droplet_test:single-plot}}
The first network case is named “single plot”. Here one DataPlot module is connected
to all Droplet modules. To run this case using MPI you should compute the number of
processes as follows:
\begin{quote}

\sphinxtitleref{nprocs = n\_droplets + 1 vortex + 1 data\_plot + 1 cortix}
\end{quote}

then issue the MPI run command as follows (replace \sphinxtitleref{nprocs} with a number):
\begin{quote}

\sphinxtitleref{mpiexec -n nprocs run\_droplet.py}
\end{quote}

To run this case with the Python multiprocessing library, just run this file at the
command line as
\begin{quote}

\sphinxtitleref{run\_droplet.py}
\end{quote}


\subsection{Multiple Plot}
\label{\detokenize{examples_rst/run_droplet_test:multiple-plot}}
The second network case is named “multiple plot”. Here each Droplet is connected to an
instance of the DataPlot module, therefore many more nodes are added to the network
when compared to the first network case. To run this case using MPI compute
\begin{quote}

\sphinxtitleref{nprocs = 2*n\_droplets + 1 vortex + 1 cortix}
\end{quote}

then issue the MPI run command as follows (replace \sphinxtitleref{nprocs}:
\begin{quote}

\sphinxtitleref{mpiexec -n nprocs run\_droplet.py}
\end{quote}

To run this case with the Python multiprocessing library, just run this file at the
command line as
\begin{quote}

\sphinxtitleref{run\_droplet.py}
\end{quote}


\section{run\_planets module}
\label{\detokenize{examples_rst/run_planets:module-run_planets}}\label{\detokenize{examples_rst/run_planets:run-planets-module}}\label{\detokenize{examples_rst/run_planets::doc}}\index{run\_planets (module)}\index{main() (in module run\_planets)}

\begin{fulllineitems}
\phantomsection\label{\detokenize{examples_rst/run_planets:run_planets.main}}\pysiglinewithargsret{\sphinxcode{\sphinxupquote{run\_planets.}}\sphinxbfcode{\sphinxupquote{main}}}{}{}
\end{fulllineitems}



\section{run\_region\_justice module}
\label{\detokenize{examples_rst/run_region_justice:module-run_region_justice}}\label{\detokenize{examples_rst/run_region_justice:run-region-justice-module}}\label{\detokenize{examples_rst/run_region_justice::doc}}\index{run\_region\_justice (module)}
Crimninal justice network dynamics modeling.
\begin{description}
\item[{This example uses 7 modules:}] \leavevmode\begin{itemize}
\item {} 
Community

\item {} 
Arrested

\item {} 
Adjudication

\item {} 
Jail

\item {} 
Prison

\item {} 
Probation

\item {} 
Parole

\end{itemize}

\end{description}

and a population balance model is used to follow the offenders population
groups between modules.

To run this case using MPI you should compute the number of
processes as follows:
\begin{quote}

\sphinxtitleref{nprocs = 7 + 1 cortix}
\end{quote}

then issue the MPI run command as follows (replace \sphinxtitleref{nprocs} with a number):
\begin{quote}

\sphinxtitleref{mpiexec -n nprocs run\_justice.py}
\end{quote}

To run this case with the Python multiprocessing library, just run this file at the
command line as
\begin{quote}

\sphinxtitleref{run\_justice.py}
\end{quote}
\index{main() (in module run\_region\_justice)}

\begin{fulllineitems}
\phantomsection\label{\detokenize{examples_rst/run_region_justice:run_region_justice.main}}\pysiglinewithargsret{\sphinxcode{\sphinxupquote{run\_region\_justice.}}\sphinxbfcode{\sphinxupquote{main}}}{}{}
Cortix run file for a criminal justice network.
\index{n\_groups (in module run\_region\_justice)}

\begin{fulllineitems}
\phantomsection\label{\detokenize{examples_rst/run_region_justice:run_region_justice.n_groups}}\pysigline{\sphinxcode{\sphinxupquote{run\_region\_justice.}}\sphinxbfcode{\sphinxupquote{n\_groups}}}
\sphinxstyleemphasis{int} \textendash{} Number of population groups being followed. This must be the same for all
modules.

\end{fulllineitems}

\index{end\_time (in module run\_region\_justice)}

\begin{fulllineitems}
\phantomsection\label{\detokenize{examples_rst/run_region_justice:run_region_justice.end_time}}\pysigline{\sphinxcode{\sphinxupquote{run\_region\_justice.}}\sphinxbfcode{\sphinxupquote{end\_time}}}
\sphinxstyleemphasis{float} \textendash{} End of the flow time in SI unit.

\end{fulllineitems}

\index{time\_step (in module run\_region\_justice)}

\begin{fulllineitems}
\phantomsection\label{\detokenize{examples_rst/run_region_justice:run_region_justice.time_step}}\pysigline{\sphinxcode{\sphinxupquote{run\_region\_justice.}}\sphinxbfcode{\sphinxupquote{time\_step}}}
\sphinxstyleemphasis{float} \textendash{} Size of the time step between port communications in SI unit.

\end{fulllineitems}

\index{use\_mpi (in module run\_region\_justice)}

\begin{fulllineitems}
\phantomsection\label{\detokenize{examples_rst/run_region_justice:run_region_justice.use_mpi}}\pysigline{\sphinxcode{\sphinxupquote{run\_region\_justice.}}\sphinxbfcode{\sphinxupquote{use\_mpi}}}
\sphinxstyleemphasis{bool} \textendash{} If set to \sphinxtitleref{True} use MPI otherwise use Python multiprocessing.

\end{fulllineitems}


\end{fulllineitems}



\section{state module}
\label{\detokenize{examples_rst/state:module-state}}\label{\detokenize{examples_rst/state:state-module}}\label{\detokenize{examples_rst/state::doc}}\index{state (module)}\index{State (class in state)}

\begin{fulllineitems}
\phantomsection\label{\detokenize{examples_rst/state:state.State}}\pysiglinewithargsret{\sphinxbfcode{\sphinxupquote{class }}\sphinxcode{\sphinxupquote{state.}}\sphinxbfcode{\sphinxupquote{State}}}{\emph{name}, \emph{non\_offender\_adult\_population=100}}{}
Bases: \sphinxcode{\sphinxupquote{cortix.src.module.Module}}

State Cortix module used to model non-offender  group population transit from and to
a state. This assumes various ports of communication with other states,
and an internal port to the internal Community.

\begin{sphinxadmonition}{note}{Notes}

These are the \sphinxtitleref{port} names available in this module to connect to other \sphinxtitleref{State}
modules: \sphinxtitleref{inflow:id}, \sphinxtitleref{outflow:id}.
In addition this module takes an internal network to model the free-offenders
community of people. The port used for this connection is \sphinxtitleref{community}.
See instance attribute \sphinxtitleref{port\_names\_expected}.
\end{sphinxadmonition}
\index{\_\_init\_\_() (state.State method)}

\begin{fulllineitems}
\phantomsection\label{\detokenize{examples_rst/state:state.State.__init__}}\pysiglinewithargsret{\sphinxbfcode{\sphinxupquote{\_\_init\_\_}}}{\emph{name}, \emph{non\_offender\_adult\_population=100}}{}~\begin{quote}\begin{description}
\item[{Parameters}] \leavevmode
\sphinxstyleliteralstrong{\sphinxupquote{non\_offender\_adult\_population}} (\sphinxhref{https://docs.python.org/3/library/functions.html\#float}{\sphinxstyleliteralemphasis{\sphinxupquote{float}}}) \textendash{} Individuals reaching the adult age (SI) unit. Default: 100.

\end{description}\end{quote}

\end{fulllineitems}

\index{run() (state.State method)}

\begin{fulllineitems}
\phantomsection\label{\detokenize{examples_rst/state:state.State.run}}\pysiglinewithargsret{\sphinxbfcode{\sphinxupquote{run}}}{\emph{*args}}{}
Module run function

Run method with an option to pass data back to the parent process when running
in Python multiprocessing mode. If the user does not want to share data with
the parent process, this function can be overriden with \sphinxtitleref{run(self)}
or \sphinxtitleref{run(self, *args)} as long as \sphinxtitleref{self.state = None}.
If \sphinxtitleref{self.state} points to anything but \sphinxtitleref{None}, the user must use
{\color{red}\bfseries{}{}`}run(self, {\color{red}\bfseries{}*}args).

\begin{sphinxadmonition}{note}{Notes}

When in multiprocessing, \sphinxtitleref{*args} has two elements: \sphinxtitleref{comm\_idx} and \sphinxtitleref{comm\_state}.
To pass back the state of the module, the user should insert the provided
index \sphinxtitleref{comm\_idx} and the \sphinxtitleref{state} into the queue as follows:
\begin{quote}
\begin{description}
\item[{if self.use\_multiprocessing:}] \leavevmode\begin{description}
\item[{try:}] \leavevmode
pickle.dumps(self.state)

\item[{except pickle.PicklingError:}] \leavevmode
args{[}1{]}.put((arg{[}0{]},None))

\item[{else:}] \leavevmode
args{[}1{]}.put((arg{[}0{]},self.state))

\end{description}

\end{description}
\end{quote}

at the bottom of the user defined \sphinxtitleref{run()} function.
\end{sphinxadmonition}

\begin{sphinxadmonition}{warning}{Warning:}
This function must be overridden by all Cortix modules
\end{sphinxadmonition}
\begin{quote}\begin{description}
\item[{Parameters}] \leavevmode\begin{itemize}
\item {} 
\sphinxstyleliteralstrong{\sphinxupquote{arg}}\sphinxstyleliteralstrong{\sphinxupquote{{[}}}\sphinxstyleliteralstrong{\sphinxupquote{0}}\sphinxstyleliteralstrong{\sphinxupquote{{]}}} (\sphinxhref{https://docs.python.org/3/library/functions.html\#int}{\sphinxstyleliteralemphasis{\sphinxupquote{int}}}) \textendash{} Index of the state in the communication queue.

\item {} 
\sphinxstyleliteralstrong{\sphinxupquote{arg}}\sphinxstyleliteralstrong{\sphinxupquote{{[}}}\sphinxstyleliteralstrong{\sphinxupquote{1}}\sphinxstyleliteralstrong{\sphinxupquote{{]}}} (\sphinxhref{https://docs.python.org/3/library/multiprocessing.html\#multiprocessing.Queue}{\sphinxstyleliteralemphasis{\sphinxupquote{multiprocessing.Queue}}}) \textendash{} When using the Python \sphinxtitleref{multiprocessing} library \sphinxtitleref{state\_comm} must have
the module’s \sphinxtitleref{self.state} in it. That is,
\sphinxtitleref{state\_comm.put((idx\_comm,self.state))} must be the last command in the
method before \sphinxtitleref{return}. In addition, self.state must be \sphinxtitleref{pickle-able}.

\end{itemize}

\end{description}\end{quote}

\end{fulllineitems}


\end{fulllineitems}



\section{vortex module}
\label{\detokenize{examples_rst/vortex:module-vortex}}\label{\detokenize{examples_rst/vortex:vortex-module}}\label{\detokenize{examples_rst/vortex::doc}}\index{vortex (module)}\index{Vortex (class in vortex)}

\begin{fulllineitems}
\phantomsection\label{\detokenize{examples_rst/vortex:vortex.Vortex}}\pysigline{\sphinxbfcode{\sphinxupquote{class }}\sphinxcode{\sphinxupquote{vortex.}}\sphinxbfcode{\sphinxupquote{Vortex}}}
Bases: \sphinxcode{\sphinxupquote{cortix.src.module.Module}}

Vortex module used to model fluid flow using Cortix.

\begin{sphinxadmonition}{note}{Notes}

Any \sphinxtitleref{port} name and any number of ports are allowed.
\end{sphinxadmonition}
\index{\_\_init\_\_() (vortex.Vortex method)}

\begin{fulllineitems}
\phantomsection\label{\detokenize{examples_rst/vortex:vortex.Vortex.__init__}}\pysiglinewithargsret{\sphinxbfcode{\sphinxupquote{\_\_init\_\_}}}{}{}~\index{initial\_time (vortex.Vortex attribute)}

\begin{fulllineitems}
\phantomsection\label{\detokenize{examples_rst/vortex:vortex.Vortex.initial_time}}\pysigline{\sphinxbfcode{\sphinxupquote{initial\_time}}}
\sphinxstyleemphasis{float}

\end{fulllineitems}

\index{end\_time (vortex.Vortex attribute)}

\begin{fulllineitems}
\phantomsection\label{\detokenize{examples_rst/vortex:vortex.Vortex.end_time}}\pysigline{\sphinxbfcode{\sphinxupquote{end\_time}}}
\sphinxstyleemphasis{float}

\end{fulllineitems}

\index{time\_step (vortex.Vortex attribute)}

\begin{fulllineitems}
\phantomsection\label{\detokenize{examples_rst/vortex:vortex.Vortex.time_step}}\pysigline{\sphinxbfcode{\sphinxupquote{time\_step}}}
\sphinxstyleemphasis{float}

\end{fulllineitems}

\index{show\_time (vortex.Vortex attribute)}

\begin{fulllineitems}
\phantomsection\label{\detokenize{examples_rst/vortex:vortex.Vortex.show_time}}\pysigline{\sphinxbfcode{\sphinxupquote{show\_time}}}
\sphinxstyleemphasis{tuple} \textendash{} Two-element tuple, \sphinxtitleref{(bool,float)}, \sphinxtitleref{True} will print to standard
output.

\end{fulllineitems}


\end{fulllineitems}

\index{compute\_velocity() (vortex.Vortex method)}

\begin{fulllineitems}
\phantomsection\label{\detokenize{examples_rst/vortex:vortex.Vortex.compute_velocity}}\pysiglinewithargsret{\sphinxbfcode{\sphinxupquote{compute\_velocity}}}{\emph{time}, \emph{position}}{}
Compute the vortex velocity at the given external
position using a vortex flow model
\begin{quote}\begin{description}
\item[{Parameters}] \leavevmode\begin{itemize}
\item {} 
\sphinxstyleliteralstrong{\sphinxupquote{time}} (\sphinxhref{https://docs.python.org/3/library/functions.html\#float}{\sphinxstyleliteralemphasis{\sphinxupquote{float}}}) \textendash{} Time in SI unit.

\item {} 
\sphinxstyleliteralstrong{\sphinxupquote{position}} (\sphinxstyleliteralemphasis{\sphinxupquote{numpy.ndarray}}\sphinxstyleliteralemphasis{\sphinxupquote{(}}\sphinxstyleliteralemphasis{\sphinxupquote{3}}\sphinxstyleliteralemphasis{\sphinxupquote{)}}) \textendash{} Spatial position in SI unit.

\end{itemize}

\item[{Returns}] \leavevmode
\sphinxstylestrong{vortex\_velocity}

\item[{Return type}] \leavevmode
numpy.ndarray(3)

\end{description}\end{quote}

\end{fulllineitems}

\index{plot\_velocity() (vortex.Vortex method)}

\begin{fulllineitems}
\phantomsection\label{\detokenize{examples_rst/vortex:vortex.Vortex.plot_velocity}}\pysiglinewithargsret{\sphinxbfcode{\sphinxupquote{plot\_velocity}}}{\emph{time=None}}{}
Plot the vortex velocity as a function of height.

\end{fulllineitems}

\index{run() (vortex.Vortex method)}

\begin{fulllineitems}
\phantomsection\label{\detokenize{examples_rst/vortex:vortex.Vortex.run}}\pysiglinewithargsret{\sphinxbfcode{\sphinxupquote{run}}}{\emph{*args}}{}
Module run function

Run method with an option to pass data back to the parent process when running
in Python multiprocessing mode. If the user does not want to share data with
the parent process, this function can be overriden with \sphinxtitleref{run(self)}
or \sphinxtitleref{run(self, *args)} as long as \sphinxtitleref{self.state = None}.
If \sphinxtitleref{self.state} points to anything but \sphinxtitleref{None}, the user must use
{\color{red}\bfseries{}{}`}run(self, {\color{red}\bfseries{}*}args).

\begin{sphinxadmonition}{note}{Notes}

When in multiprocessing, \sphinxtitleref{*args} has two elements: \sphinxtitleref{comm\_idx} and \sphinxtitleref{comm\_state}.
To pass back the state of the module, the user should insert the provided
index \sphinxtitleref{comm\_idx} and the \sphinxtitleref{state} into the queue as follows:
\begin{quote}
\begin{description}
\item[{if self.use\_multiprocessing:}] \leavevmode\begin{description}
\item[{try:}] \leavevmode
pickle.dumps(self.state)

\item[{except pickle.PicklingError:}] \leavevmode
args{[}1{]}.put((arg{[}0{]},None))

\item[{else:}] \leavevmode
args{[}1{]}.put((arg{[}0{]},self.state))

\end{description}

\end{description}
\end{quote}

at the bottom of the user defined \sphinxtitleref{run()} function.
\end{sphinxadmonition}

\begin{sphinxadmonition}{warning}{Warning:}
This function must be overridden by all Cortix modules
\end{sphinxadmonition}
\begin{quote}\begin{description}
\item[{Parameters}] \leavevmode\begin{itemize}
\item {} 
\sphinxstyleliteralstrong{\sphinxupquote{arg}}\sphinxstyleliteralstrong{\sphinxupquote{{[}}}\sphinxstyleliteralstrong{\sphinxupquote{0}}\sphinxstyleliteralstrong{\sphinxupquote{{]}}} (\sphinxhref{https://docs.python.org/3/library/functions.html\#int}{\sphinxstyleliteralemphasis{\sphinxupquote{int}}}) \textendash{} Index of the state in the communication queue.

\item {} 
\sphinxstyleliteralstrong{\sphinxupquote{arg}}\sphinxstyleliteralstrong{\sphinxupquote{{[}}}\sphinxstyleliteralstrong{\sphinxupquote{1}}\sphinxstyleliteralstrong{\sphinxupquote{{]}}} (\sphinxhref{https://docs.python.org/3/library/multiprocessing.html\#multiprocessing.Queue}{\sphinxstyleliteralemphasis{\sphinxupquote{multiprocessing.Queue}}}) \textendash{} When using the Python \sphinxtitleref{multiprocessing} library \sphinxtitleref{state\_comm} must have
the module’s \sphinxtitleref{self.state} in it. That is,
\sphinxtitleref{state\_comm.put((idx\_comm,self.state))} must be the last command in the
method before \sphinxtitleref{return}. In addition, self.state must be \sphinxtitleref{pickle-able}.

\end{itemize}

\end{description}\end{quote}

\end{fulllineitems}


\end{fulllineitems}



\chapter{support}
\label{\detokenize{support_rst/modules:support}}\label{\detokenize{support_rst/modules::doc}}

\section{actor}
\label{\detokenize{support_rst/actor:module-actor}}\label{\detokenize{support_rst/actor:actor}}\label{\detokenize{support_rst/actor::doc}}\index{actor (module)}
This is a simple way to hide the name of species of interest in a simulation.
The user would modify and copy this class into the Cortix module of interest
and keep it private.
Author: Valmor de Almeida \sphinxhref{mailto:dealmeidav@ornl.gov}{dealmeidav@ornl.gov}; vfda
Sat Aug 15 13:41:12 EDT 2015
\index{Actor (class in actor)}

\begin{fulllineitems}
\phantomsection\label{\detokenize{support_rst/actor:actor.Actor}}\pysiglinewithargsret{\sphinxbfcode{\sphinxupquote{class }}\sphinxcode{\sphinxupquote{actor.}}\sphinxbfcode{\sphinxupquote{Actor}}}{\emph{name}}{}
Bases: \sphinxhref{https://docs.python.org/3/library/functions.html\#object}{\sphinxcode{\sphinxupquote{object}}}

See atoms list in Specie.
\index{atoms (actor.Actor attribute)}

\begin{fulllineitems}
\phantomsection\label{\detokenize{support_rst/actor:actor.Actor.atoms}}\pysigline{\sphinxbfcode{\sphinxupquote{atoms}}}
Returns the specific nuclides found in the specified chemical.
\begin{quote}\begin{description}
\item[{Returns}] \leavevmode
\sphinxstylestrong{atoms}

\item[{Return type}] \leavevmode
\sphinxhref{https://docs.python.org/3/library/stdtypes.html\#list}{list}(\sphinxhref{https://docs.python.org/3/library/stdtypes.html\#str}{str})

\end{description}\end{quote}

\end{fulllineitems}

\index{formula (actor.Actor attribute)}

\begin{fulllineitems}
\phantomsection\label{\detokenize{support_rst/actor:actor.Actor.formula}}\pysigline{\sphinxbfcode{\sphinxupquote{formula}}}
Returns the formula of the chemical in question.
\begin{quote}\begin{description}
\item[{Returns}] \leavevmode
\sphinxstylestrong{formula}

\item[{Return type}] \leavevmode
\sphinxhref{https://docs.python.org/3/library/stdtypes.html\#str}{str}

\end{description}\end{quote}

\end{fulllineitems}


\end{fulllineitems}



\section{fuel\_bucket}
\label{\detokenize{support_rst/fuel_bucket:module-fuel_bucket}}\label{\detokenize{support_rst/fuel_bucket:fuel-bucket}}\label{\detokenize{support_rst/fuel_bucket::doc}}\index{fuel\_bucket (module)}
This FuelBucket class is a container for usage with other plant-level process modules.
It is meant to represent a fuel bucket of a metal fuel reactor.
———-
ATTENTION:
———-
This container uses Phase() for phases (cladding and fuel). Therefore user is
responsible to make the “history” of the phases consistent. See Phase() info.

Author: Valmor de Almeida \sphinxhref{mailto:dealmeidav@ornl.gov}{dealmeidav@ornl.gov}; vfda
\index{FuelBucket (class in fuel\_bucket)}

\begin{fulllineitems}
\phantomsection\label{\detokenize{support_rst/fuel_bucket:fuel_bucket.FuelBucket}}\pysiglinewithargsret{\sphinxbfcode{\sphinxupquote{class }}\sphinxcode{\sphinxupquote{fuel\_bucket.}}\sphinxbfcode{\sphinxupquote{FuelBucket}}}{\emph{specs=Empty DataFrame Columns: {[}{]} Index: {[}{]}}}{}
Bases: \sphinxhref{https://docs.python.org/3/library/functions.html\#object}{\sphinxcode{\sphinxupquote{object}}}
\index{\_\_repr\_\_() (fuel\_bucket.FuelBucket method)}

\begin{fulllineitems}
\phantomsection\label{\detokenize{support_rst/fuel_bucket:fuel_bucket.FuelBucket.__repr__}}\pysiglinewithargsret{\sphinxbfcode{\sphinxupquote{\_\_repr\_\_}}}{}{}
Converts to string.

\end{fulllineitems}

\index{\_\_str\_\_() (fuel\_bucket.FuelBucket method)}

\begin{fulllineitems}
\phantomsection\label{\detokenize{support_rst/fuel_bucket:fuel_bucket.FuelBucket.__str__}}\pysiglinewithargsret{\sphinxbfcode{\sphinxupquote{\_\_str\_\_}}}{}{}
Converts to string.

\end{fulllineitems}

\index{cladding\_end\_thickness (fuel\_bucket.FuelBucket attribute)}

\begin{fulllineitems}
\phantomsection\label{\detokenize{support_rst/fuel_bucket:fuel_bucket.FuelBucket.cladding_end_thickness}}\pysigline{\sphinxbfcode{\sphinxupquote{cladding\_end\_thickness}}}
Gets the thickness of the hemispherical cladding end caps that are
placed on the top and bottom of the fuel slug, in cm.
\begin{quote}\begin{description}
\item[{Returns}] \leavevmode
\sphinxstylestrong{cladding\_end\_thickness}

\item[{Return type}] \leavevmode
\sphinxhref{https://docs.python.org/3/library/functions.html\#float}{float}

\end{description}\end{quote}

\end{fulllineitems}

\index{cladding\_mass (fuel\_bucket.FuelBucket attribute)}

\begin{fulllineitems}
\phantomsection\label{\detokenize{support_rst/fuel_bucket:fuel_bucket.FuelBucket.cladding_mass}}\pysigline{\sphinxbfcode{\sphinxupquote{cladding\_mass}}}
Returns the total mass of cladding material in the bucket, in grams.
\begin{quote}\begin{description}
\item[{Returns}] \leavevmode
\sphinxstylestrong{cladding\_mass}

\item[{Return type}] \leavevmode
\sphinxhref{https://docs.python.org/3/library/functions.html\#float}{float}

\end{description}\end{quote}

\end{fulllineitems}

\index{cladding\_phase (fuel\_bucket.FuelBucket attribute)}

\begin{fulllineitems}
\phantomsection\label{\detokenize{support_rst/fuel_bucket:fuel_bucket.FuelBucket.cladding_phase}}\pysigline{\sphinxbfcode{\sphinxupquote{cladding\_phase}}}
Returns the phase history of the cladding.
\begin{quote}\begin{description}
\item[{Returns}] \leavevmode
\sphinxstylestrong{cladding\_phase}

\item[{Return type}] \leavevmode
dataFrame

\end{description}\end{quote}

\end{fulllineitems}

\index{cladding\_volume (fuel\_bucket.FuelBucket attribute)}

\begin{fulllineitems}
\phantomsection\label{\detokenize{support_rst/fuel_bucket:fuel_bucket.FuelBucket.cladding_volume}}\pysigline{\sphinxbfcode{\sphinxupquote{cladding\_volume}}}
Returns the total volume of cladding in the bucket, in cm\textasciicircum{}3.
\begin{quote}\begin{description}
\item[{Returns}] \leavevmode
\sphinxstylestrong{cladding\_volume}

\item[{Return type}] \leavevmode
\sphinxhref{https://docs.python.org/3/library/functions.html\#float}{float}

\end{description}\end{quote}

\end{fulllineitems}

\index{cladding\_wall\_thickness (fuel\_bucket.FuelBucket attribute)}

\begin{fulllineitems}
\phantomsection\label{\detokenize{support_rst/fuel_bucket:fuel_bucket.FuelBucket.cladding_wall_thickness}}\pysigline{\sphinxbfcode{\sphinxupquote{cladding\_wall\_thickness}}}
Returns the thickness of the cladding wall which is on the outside of
every fuel slug, and in between both sections of fuel, in cm.
\begin{quote}\begin{description}
\item[{Returns}] \leavevmode
\sphinxstylestrong{cladding\_wall\_thickness}

\item[{Return type}] \leavevmode
\sphinxhref{https://docs.python.org/3/library/functions.html\#float}{float}

\end{description}\end{quote}

\end{fulllineitems}

\index{fresh\_u235\_mass (fuel\_bucket.FuelBucket attribute)}

\begin{fulllineitems}
\phantomsection\label{\detokenize{support_rst/fuel_bucket:fuel_bucket.FuelBucket.fresh_u235_mass}}\pysigline{\sphinxbfcode{\sphinxupquote{fresh\_u235\_mass}}}
Returns the total amount of uranium-235 in the bucket, in grams.
\begin{quote}\begin{description}
\item[{Returns}] \leavevmode
\sphinxstylestrong{fresh\_u235\_mass}

\item[{Return type}] \leavevmode
\sphinxhref{https://docs.python.org/3/library/functions.html\#float}{float}

\end{description}\end{quote}

\end{fulllineitems}

\index{fresh\_u238\_mass (fuel\_bucket.FuelBucket attribute)}

\begin{fulllineitems}
\phantomsection\label{\detokenize{support_rst/fuel_bucket:fuel_bucket.FuelBucket.fresh_u238_mass}}\pysigline{\sphinxbfcode{\sphinxupquote{fresh\_u238\_mass}}}
Returns the total amount of uranium-238 in the bucket, in grams.
\begin{quote}\begin{description}
\item[{Returns}] \leavevmode
\sphinxstylestrong{fresh\_u238\_mass}

\item[{Return type}] \leavevmode
\sphinxhref{https://docs.python.org/3/library/functions.html\#float}{float}

\end{description}\end{quote}

\end{fulllineitems}

\index{fresh\_u\_mass (fuel\_bucket.FuelBucket attribute)}

\begin{fulllineitems}
\phantomsection\label{\detokenize{support_rst/fuel_bucket:fuel_bucket.FuelBucket.fresh_u_mass}}\pysigline{\sphinxbfcode{\sphinxupquote{fresh\_u\_mass}}}
Returns the total amount of uranium in the bucket, in grams.
\begin{quote}\begin{description}
\item[{Returns}] \leavevmode
\sphinxstylestrong{fresh\_u\_mass}

\item[{Return type}] \leavevmode
\sphinxhref{https://docs.python.org/3/library/functions.html\#float}{float}

\end{description}\end{quote}

\end{fulllineitems}

\index{fuel\_enrichment (fuel\_bucket.FuelBucket attribute)}

\begin{fulllineitems}
\phantomsection\label{\detokenize{support_rst/fuel_bucket:fuel_bucket.FuelBucket.fuel_enrichment}}\pysigline{\sphinxbfcode{\sphinxupquote{fuel\_enrichment}}}
Returns the enrichment of the fuel slugs in the bucket, in \%.
\begin{quote}\begin{description}
\item[{Returns}] \leavevmode
\sphinxstylestrong{fuel\_enrichment}

\item[{Return type}] \leavevmode
\sphinxhref{https://docs.python.org/3/library/functions.html\#float}{float}

\end{description}\end{quote}

\end{fulllineitems}

\index{fuel\_mass (fuel\_bucket.FuelBucket attribute)}

\begin{fulllineitems}
\phantomsection\label{\detokenize{support_rst/fuel_bucket:fuel_bucket.FuelBucket.fuel_mass}}\pysigline{\sphinxbfcode{\sphinxupquote{fuel\_mass}}}
Returns the total mass of fuel in the solid phase in the bucket.
\begin{quote}\begin{description}
\item[{Returns}] \leavevmode
\sphinxstylestrong{fuel\_mass}

\item[{Return type}] \leavevmode
\sphinxhref{https://docs.python.org/3/library/functions.html\#float}{float}

\end{description}\end{quote}

\end{fulllineitems}

\index{fuel\_mass\_unit (fuel\_bucket.FuelBucket attribute)}

\begin{fulllineitems}
\phantomsection\label{\detokenize{support_rst/fuel_bucket:fuel_bucket.FuelBucket.fuel_mass_unit}}\pysigline{\sphinxbfcode{\sphinxupquote{fuel\_mass\_unit}}}
Returns the unit that is used to measure the mass of fuel in the
bucket.
\begin{quote}\begin{description}
\item[{Returns}] \leavevmode
\sphinxstylestrong{fuel\_mass\_unit}

\item[{Return type}] \leavevmode
\sphinxhref{https://docs.python.org/3/library/stdtypes.html\#str}{str}

\end{description}\end{quote}

\end{fulllineitems}

\index{fuel\_phase (fuel\_bucket.FuelBucket attribute)}

\begin{fulllineitems}
\phantomsection\label{\detokenize{support_rst/fuel_bucket:fuel_bucket.FuelBucket.fuel_phase}}\pysigline{\sphinxbfcode{\sphinxupquote{fuel\_phase}}}
Returns the phase history of the fuel.
\begin{quote}\begin{description}
\item[{Returns}] \leavevmode
\sphinxstylestrong{fuel\_phase}

\item[{Return type}] \leavevmode
pandas.core.frame.DataFrame

\end{description}\end{quote}

\end{fulllineitems}

\index{fuel\_radioactivity (fuel\_bucket.FuelBucket attribute)}

\begin{fulllineitems}
\phantomsection\label{\detokenize{support_rst/fuel_bucket:fuel_bucket.FuelBucket.fuel_radioactivity}}\pysigline{\sphinxbfcode{\sphinxupquote{fuel\_radioactivity}}}
Returns the total radioactivity of the solid phase fuel, in units of
curies.
\begin{quote}\begin{description}
\item[{Returns}] \leavevmode
\sphinxstylestrong{fuel\_radioactivity}

\item[{Return type}] \leavevmode
\sphinxhref{https://docs.python.org/3/library/functions.html\#float}{float}

\end{description}\end{quote}

\end{fulllineitems}

\index{fuel\_volume (fuel\_bucket.FuelBucket attribute)}

\begin{fulllineitems}
\phantomsection\label{\detokenize{support_rst/fuel_bucket:fuel_bucket.FuelBucket.fuel_volume}}\pysigline{\sphinxbfcode{\sphinxupquote{fuel\_volume}}}
Returns the total volume of fuel in the entire bucket, in cm\textasciicircum{}3.
\begin{quote}\begin{description}
\item[{Returns}] \leavevmode
\sphinxstylestrong{fuel\_volume}

\item[{Return type}] \leavevmode
\sphinxhref{https://docs.python.org/3/library/functions.html\#float}{float}

\end{description}\end{quote}

\end{fulllineitems}

\index{gamma\_pwr (fuel\_bucket.FuelBucket attribute)}

\begin{fulllineitems}
\phantomsection\label{\detokenize{support_rst/fuel_bucket:fuel_bucket.FuelBucket.gamma_pwr}}\pysigline{\sphinxbfcode{\sphinxupquote{gamma\_pwr}}}
Returns the amount of gamma radiation given off by the fuel bucket,
in units of watts.
\begin{quote}\begin{description}
\item[{Returns}] \leavevmode
\sphinxstylestrong{gamma\_pwr}

\item[{Return type}] \leavevmode
\sphinxhref{https://docs.python.org/3/library/functions.html\#float}{float}

\end{description}\end{quote}

\end{fulllineitems}

\index{get\_cladding\_end\_thickness() (fuel\_bucket.FuelBucket method)}

\begin{fulllineitems}
\phantomsection\label{\detokenize{support_rst/fuel_bucket:fuel_bucket.FuelBucket.get_cladding_end_thickness}}\pysiglinewithargsret{\sphinxbfcode{\sphinxupquote{get\_cladding\_end\_thickness}}}{}{}
Gets the thickness of the hemispherical cladding end caps that are
placed on the top and bottom of the fuel slug, in cm.
\begin{quote}\begin{description}
\item[{Returns}] \leavevmode
\sphinxstylestrong{cladding\_end\_thickness}

\item[{Return type}] \leavevmode
\sphinxhref{https://docs.python.org/3/library/functions.html\#float}{float}

\end{description}\end{quote}

\end{fulllineitems}

\index{get\_cladding\_mass() (fuel\_bucket.FuelBucket method)}

\begin{fulllineitems}
\phantomsection\label{\detokenize{support_rst/fuel_bucket:fuel_bucket.FuelBucket.get_cladding_mass}}\pysiglinewithargsret{\sphinxbfcode{\sphinxupquote{get\_cladding\_mass}}}{}{}
Returns the total mass of cladding material in the bucket, in grams.
\begin{quote}\begin{description}
\item[{Returns}] \leavevmode
\sphinxstylestrong{cladding\_mass}

\item[{Return type}] \leavevmode
\sphinxhref{https://docs.python.org/3/library/functions.html\#float}{float}

\end{description}\end{quote}

\end{fulllineitems}

\index{get\_cladding\_phase() (fuel\_bucket.FuelBucket method)}

\begin{fulllineitems}
\phantomsection\label{\detokenize{support_rst/fuel_bucket:fuel_bucket.FuelBucket.get_cladding_phase}}\pysiglinewithargsret{\sphinxbfcode{\sphinxupquote{get\_cladding\_phase}}}{}{}
Returns the phase history of the cladding.
\begin{quote}\begin{description}
\item[{Returns}] \leavevmode
\sphinxstylestrong{cladding\_phase}

\item[{Return type}] \leavevmode
dataFrame

\end{description}\end{quote}

\end{fulllineitems}

\index{get\_cladding\_volume() (fuel\_bucket.FuelBucket method)}

\begin{fulllineitems}
\phantomsection\label{\detokenize{support_rst/fuel_bucket:fuel_bucket.FuelBucket.get_cladding_volume}}\pysiglinewithargsret{\sphinxbfcode{\sphinxupquote{get\_cladding\_volume}}}{}{}
Returns the total volume of cladding in the bucket, in cm\textasciicircum{}3.
\begin{quote}\begin{description}
\item[{Returns}] \leavevmode
\sphinxstylestrong{cladding\_volume}

\item[{Return type}] \leavevmode
\sphinxhref{https://docs.python.org/3/library/functions.html\#float}{float}

\end{description}\end{quote}

\end{fulllineitems}

\index{get\_cladding\_wall\_thickness() (fuel\_bucket.FuelBucket method)}

\begin{fulllineitems}
\phantomsection\label{\detokenize{support_rst/fuel_bucket:fuel_bucket.FuelBucket.get_cladding_wall_thickness}}\pysiglinewithargsret{\sphinxbfcode{\sphinxupquote{get\_cladding\_wall\_thickness}}}{}{}
Returns the thickness of the cladding wall which is on the outside of
every fuel slug, and in between both sections of fuel, in cm.
\begin{quote}\begin{description}
\item[{Returns}] \leavevmode
\sphinxstylestrong{cladding\_wall\_thickness}

\item[{Return type}] \leavevmode
\sphinxhref{https://docs.python.org/3/library/functions.html\#float}{float}

\end{description}\end{quote}

\end{fulllineitems}

\index{get\_fresh\_u235\_mass() (fuel\_bucket.FuelBucket method)}

\begin{fulllineitems}
\phantomsection\label{\detokenize{support_rst/fuel_bucket:fuel_bucket.FuelBucket.get_fresh_u235_mass}}\pysiglinewithargsret{\sphinxbfcode{\sphinxupquote{get\_fresh\_u235\_mass}}}{}{}
Returns the total amount of uranium-235 in the bucket, in grams.
\begin{quote}\begin{description}
\item[{Returns}] \leavevmode
\sphinxstylestrong{fresh\_u235\_mass}

\item[{Return type}] \leavevmode
\sphinxhref{https://docs.python.org/3/library/functions.html\#float}{float}

\end{description}\end{quote}

\end{fulllineitems}

\index{get\_fresh\_u238\_mass() (fuel\_bucket.FuelBucket method)}

\begin{fulllineitems}
\phantomsection\label{\detokenize{support_rst/fuel_bucket:fuel_bucket.FuelBucket.get_fresh_u238_mass}}\pysiglinewithargsret{\sphinxbfcode{\sphinxupquote{get\_fresh\_u238\_mass}}}{}{}
Returns the total amount of uranium-238 in the bucket, in grams.
\begin{quote}\begin{description}
\item[{Returns}] \leavevmode
\sphinxstylestrong{fresh\_u238\_mass}

\item[{Return type}] \leavevmode
\sphinxhref{https://docs.python.org/3/library/functions.html\#float}{float}

\end{description}\end{quote}

\end{fulllineitems}

\index{get\_fresh\_u\_mass() (fuel\_bucket.FuelBucket method)}

\begin{fulllineitems}
\phantomsection\label{\detokenize{support_rst/fuel_bucket:fuel_bucket.FuelBucket.get_fresh_u_mass}}\pysiglinewithargsret{\sphinxbfcode{\sphinxupquote{get\_fresh\_u\_mass}}}{}{}
Returns the total amount of uranium in the bucket, in grams.
\begin{quote}\begin{description}
\item[{Returns}] \leavevmode
\sphinxstylestrong{fresh\_u\_mass}

\item[{Return type}] \leavevmode
\sphinxhref{https://docs.python.org/3/library/functions.html\#float}{float}

\end{description}\end{quote}

\end{fulllineitems}

\index{get\_fuel\_enrichment() (fuel\_bucket.FuelBucket method)}

\begin{fulllineitems}
\phantomsection\label{\detokenize{support_rst/fuel_bucket:fuel_bucket.FuelBucket.get_fuel_enrichment}}\pysiglinewithargsret{\sphinxbfcode{\sphinxupquote{get\_fuel\_enrichment}}}{}{}
Returns the enrichment of the fuel slugs in the bucket, in \%.
\begin{quote}\begin{description}
\item[{Returns}] \leavevmode
\sphinxstylestrong{fuel\_enrichment}

\item[{Return type}] \leavevmode
\sphinxhref{https://docs.python.org/3/library/functions.html\#float}{float}

\end{description}\end{quote}

\end{fulllineitems}

\index{get\_fuel\_mass() (fuel\_bucket.FuelBucket method)}

\begin{fulllineitems}
\phantomsection\label{\detokenize{support_rst/fuel_bucket:fuel_bucket.FuelBucket.get_fuel_mass}}\pysiglinewithargsret{\sphinxbfcode{\sphinxupquote{get\_fuel\_mass}}}{}{}
Returns the total mass of fuel in the solid phase in the bucket.
\begin{quote}\begin{description}
\item[{Returns}] \leavevmode
\sphinxstylestrong{fuel\_mass}

\item[{Return type}] \leavevmode
\sphinxhref{https://docs.python.org/3/library/functions.html\#float}{float}

\end{description}\end{quote}

\end{fulllineitems}

\index{get\_fuel\_mass\_unit() (fuel\_bucket.FuelBucket method)}

\begin{fulllineitems}
\phantomsection\label{\detokenize{support_rst/fuel_bucket:fuel_bucket.FuelBucket.get_fuel_mass_unit}}\pysiglinewithargsret{\sphinxbfcode{\sphinxupquote{get\_fuel\_mass\_unit}}}{}{}
Returns the unit that is used to measure the mass of fuel in the
bucket.
\begin{quote}\begin{description}
\item[{Returns}] \leavevmode
\sphinxstylestrong{fuel\_mass\_unit}

\item[{Return type}] \leavevmode
\sphinxhref{https://docs.python.org/3/library/stdtypes.html\#str}{str}

\end{description}\end{quote}

\end{fulllineitems}

\index{get\_fuel\_phase() (fuel\_bucket.FuelBucket method)}

\begin{fulllineitems}
\phantomsection\label{\detokenize{support_rst/fuel_bucket:fuel_bucket.FuelBucket.get_fuel_phase}}\pysiglinewithargsret{\sphinxbfcode{\sphinxupquote{get\_fuel\_phase}}}{}{}
Returns the phase history of the fuel.
\begin{quote}\begin{description}
\item[{Returns}] \leavevmode
\sphinxstylestrong{fuel\_phase}

\item[{Return type}] \leavevmode
pandas.core.frame.DataFrame

\end{description}\end{quote}

\end{fulllineitems}

\index{get\_fuel\_radioactivity() (fuel\_bucket.FuelBucket method)}

\begin{fulllineitems}
\phantomsection\label{\detokenize{support_rst/fuel_bucket:fuel_bucket.FuelBucket.get_fuel_radioactivity}}\pysiglinewithargsret{\sphinxbfcode{\sphinxupquote{get\_fuel\_radioactivity}}}{}{}
Returns the total radioactivity of the solid phase fuel, in units of
curies.
\begin{quote}\begin{description}
\item[{Returns}] \leavevmode
\sphinxstylestrong{fuel\_radioactivity}

\item[{Return type}] \leavevmode
\sphinxhref{https://docs.python.org/3/library/functions.html\#float}{float}

\end{description}\end{quote}

\end{fulllineitems}

\index{get\_fuel\_volume() (fuel\_bucket.FuelBucket method)}

\begin{fulllineitems}
\phantomsection\label{\detokenize{support_rst/fuel_bucket:fuel_bucket.FuelBucket.get_fuel_volume}}\pysiglinewithargsret{\sphinxbfcode{\sphinxupquote{get\_fuel\_volume}}}{}{}
Returns the total volume of fuel in the entire bucket, in cm\textasciicircum{}3.
\begin{quote}\begin{description}
\item[{Returns}] \leavevmode
\sphinxstylestrong{fuel\_volume}

\item[{Return type}] \leavevmode
\sphinxhref{https://docs.python.org/3/library/functions.html\#float}{float}

\end{description}\end{quote}

\end{fulllineitems}

\index{get\_gamma\_pwr() (fuel\_bucket.FuelBucket method)}

\begin{fulllineitems}
\phantomsection\label{\detokenize{support_rst/fuel_bucket:fuel_bucket.FuelBucket.get_gamma_pwr}}\pysiglinewithargsret{\sphinxbfcode{\sphinxupquote{get\_gamma\_pwr}}}{}{}
Returns the amount of gamma radiation given off by the fuel bucket,
in units of watts.
\begin{quote}\begin{description}
\item[{Returns}] \leavevmode
\sphinxstylestrong{gamma\_pwr}

\item[{Return type}] \leavevmode
\sphinxhref{https://docs.python.org/3/library/functions.html\#float}{float}

\end{description}\end{quote}

\end{fulllineitems}

\index{get\_heat\_pwr() (fuel\_bucket.FuelBucket method)}

\begin{fulllineitems}
\phantomsection\label{\detokenize{support_rst/fuel_bucket:fuel_bucket.FuelBucket.get_heat_pwr}}\pysiglinewithargsret{\sphinxbfcode{\sphinxupquote{get\_heat\_pwr}}}{}{}
Returns the total amount of heat generated by the bucket, in units of
watts.
\begin{quote}\begin{description}
\item[{Returns}] \leavevmode
\sphinxstylestrong{heat\_pwr}

\item[{Return type}] \leavevmode
\sphinxhref{https://docs.python.org/3/library/functions.html\#float}{float}

\end{description}\end{quote}

\end{fulllineitems}

\index{get\_inner\_slug\_id() (fuel\_bucket.FuelBucket method)}

\begin{fulllineitems}
\phantomsection\label{\detokenize{support_rst/fuel_bucket:fuel_bucket.FuelBucket.get_inner_slug_id}}\pysiglinewithargsret{\sphinxbfcode{\sphinxupquote{get\_inner\_slug\_id}}}{}{}
Returns the inner diameter of the inner section of fuel, in cm.
\begin{quote}\begin{description}
\item[{Returns}] \leavevmode
\sphinxstylestrong{inner\_slug\_id}

\item[{Return type}] \leavevmode
\sphinxhref{https://docs.python.org/3/library/functions.html\#float}{float}

\end{description}\end{quote}

\end{fulllineitems}

\index{get\_inner\_slug\_od() (fuel\_bucket.FuelBucket method)}

\begin{fulllineitems}
\phantomsection\label{\detokenize{support_rst/fuel_bucket:fuel_bucket.FuelBucket.get_inner_slug_od}}\pysiglinewithargsret{\sphinxbfcode{\sphinxupquote{get\_inner\_slug\_od}}}{}{}
Returns the outer  diameter of the inner section of fuel, in cm.
\begin{quote}\begin{description}
\item[{Returns}] \leavevmode
\sphinxstylestrong{inner\_slug\_od}

\item[{Return type}] \leavevmode
\sphinxhref{https://docs.python.org/3/library/functions.html\#float}{float}

\end{description}\end{quote}

\end{fulllineitems}

\index{get\_n\_slugs() (fuel\_bucket.FuelBucket method)}

\begin{fulllineitems}
\phantomsection\label{\detokenize{support_rst/fuel_bucket:fuel_bucket.FuelBucket.get_n_slugs}}\pysiglinewithargsret{\sphinxbfcode{\sphinxupquote{get\_n\_slugs}}}{}{}
Returns the number of fuel slugs in the bucket.
\begin{quote}\begin{description}
\item[{Returns}] \leavevmode
\sphinxstylestrong{n\_slugs}

\item[{Return type}] \leavevmode
\sphinxhref{https://docs.python.org/3/library/functions.html\#int}{int}

\end{description}\end{quote}

\end{fulllineitems}

\index{get\_name() (fuel\_bucket.FuelBucket method)}

\begin{fulllineitems}
\phantomsection\label{\detokenize{support_rst/fuel_bucket:fuel_bucket.FuelBucket.get_name}}\pysiglinewithargsret{\sphinxbfcode{\sphinxupquote{get\_name}}}{}{}
Returns the name of the fuel bucket.
\begin{quote}\begin{description}
\item[{Returns}] \leavevmode
\sphinxstylestrong{name}

\item[{Return type}] \leavevmode
\sphinxhref{https://docs.python.org/3/library/stdtypes.html\#str}{str}

\end{description}\end{quote}

\end{fulllineitems}

\index{get\_outer\_slug\_id() (fuel\_bucket.FuelBucket method)}

\begin{fulllineitems}
\phantomsection\label{\detokenize{support_rst/fuel_bucket:fuel_bucket.FuelBucket.get_outer_slug_id}}\pysiglinewithargsret{\sphinxbfcode{\sphinxupquote{get\_outer\_slug\_id}}}{}{}
Returns the inner diameter of the outer section of fuel, in cm.
\begin{quote}\begin{description}
\item[{Returns}] \leavevmode
\sphinxstylestrong{outer\_slug\_id}

\item[{Return type}] \leavevmode
\sphinxhref{https://docs.python.org/3/library/functions.html\#float}{float}

\end{description}\end{quote}

\end{fulllineitems}

\index{get\_outer\_slug\_od() (fuel\_bucket.FuelBucket method)}

\begin{fulllineitems}
\phantomsection\label{\detokenize{support_rst/fuel_bucket:fuel_bucket.FuelBucket.get_outer_slug_od}}\pysiglinewithargsret{\sphinxbfcode{\sphinxupquote{get\_outer\_slug\_od}}}{}{}
Returns the outer diameter of the outer section of fuel, in cm. A fuel
slug consists of an outer section of fuel and an inner section of fuel,
with cladding on the outside of the slug and between the inner and
outer sections of fuel.
\begin{quote}\begin{description}
\item[{Returns}] \leavevmode
\sphinxstylestrong{outer\_slug\_od}

\item[{Return type}] \leavevmode
\sphinxhref{https://docs.python.org/3/library/functions.html\#float}{float}

\end{description}\end{quote}

\end{fulllineitems}

\index{get\_radioactivity() (fuel\_bucket.FuelBucket method)}

\begin{fulllineitems}
\phantomsection\label{\detokenize{support_rst/fuel_bucket:fuel_bucket.FuelBucket.get_radioactivity}}\pysiglinewithargsret{\sphinxbfcode{\sphinxupquote{get\_radioactivity}}}{}{}
Returns the radioactivity of the fuel bucket, in units of curies.
\begin{quote}\begin{description}
\item[{Returns}] \leavevmode
\sphinxstylestrong{radioactivity}

\item[{Return type}] \leavevmode
\sphinxhref{https://docs.python.org/3/library/functions.html\#float}{float}

\end{description}\end{quote}

\end{fulllineitems}

\index{get\_slug\_cladding\_volume() (fuel\_bucket.FuelBucket method)}

\begin{fulllineitems}
\phantomsection\label{\detokenize{support_rst/fuel_bucket:fuel_bucket.FuelBucket.get_slug_cladding_volume}}\pysiglinewithargsret{\sphinxbfcode{\sphinxupquote{get\_slug\_cladding\_volume}}}{}{}
Returns the volume of cladding present in a single fuel slug, in cm\textasciicircum{}3.
\begin{quote}\begin{description}
\item[{Returns}] \leavevmode
\sphinxstylestrong{slug\_cladding\_volume}

\item[{Return type}] \leavevmode
\sphinxhref{https://docs.python.org/3/library/functions.html\#float}{float}

\end{description}\end{quote}

\end{fulllineitems}

\index{get\_slug\_fuel\_volume() (fuel\_bucket.FuelBucket method)}

\begin{fulllineitems}
\phantomsection\label{\detokenize{support_rst/fuel_bucket:fuel_bucket.FuelBucket.get_slug_fuel_volume}}\pysiglinewithargsret{\sphinxbfcode{\sphinxupquote{get\_slug\_fuel\_volume}}}{}{}
Returns the volume of fuel present in a single fuel slug, in cm\textasciicircum{}3.
\begin{quote}\begin{description}
\item[{Returns}] \leavevmode
\sphinxstylestrong{slug\_fuel\_volume}

\item[{Return type}] \leavevmode
\sphinxhref{https://docs.python.org/3/library/functions.html\#float}{float}

\end{description}\end{quote}

\end{fulllineitems}

\index{get\_slug\_length() (fuel\_bucket.FuelBucket method)}

\begin{fulllineitems}
\phantomsection\label{\detokenize{support_rst/fuel_bucket:fuel_bucket.FuelBucket.get_slug_length}}\pysiglinewithargsret{\sphinxbfcode{\sphinxupquote{get\_slug\_length}}}{}{}
Returns the length of each slug in the fuel bucket.
\begin{quote}\begin{description}
\item[{Returns}] \leavevmode
\sphinxstylestrong{slug\_length}

\item[{Return type}] \leavevmode
\sphinxhref{https://docs.python.org/3/library/functions.html\#float}{float}

\end{description}\end{quote}

\end{fulllineitems}

\index{get\_slug\_type() (fuel\_bucket.FuelBucket method)}

\begin{fulllineitems}
\phantomsection\label{\detokenize{support_rst/fuel_bucket:fuel_bucket.FuelBucket.get_slug_type}}\pysiglinewithargsret{\sphinxbfcode{\sphinxupquote{get\_slug\_type}}}{}{}
Returns the type of slugs being stored in the bucket (inner slug or
outer slug).
\begin{quote}\begin{description}
\item[{Returns}] \leavevmode
\sphinxstylestrong{slug\_type}

\item[{Return type}] \leavevmode
\sphinxhref{https://docs.python.org/3/library/stdtypes.html\#str}{str}

\end{description}\end{quote}

\end{fulllineitems}

\index{heat\_pwr (fuel\_bucket.FuelBucket attribute)}

\begin{fulllineitems}
\phantomsection\label{\detokenize{support_rst/fuel_bucket:fuel_bucket.FuelBucket.heat_pwr}}\pysigline{\sphinxbfcode{\sphinxupquote{heat\_pwr}}}
Returns the total amount of heat generated by the bucket, in units of
watts.
\begin{quote}\begin{description}
\item[{Returns}] \leavevmode
\sphinxstylestrong{heat\_pwr}

\item[{Return type}] \leavevmode
\sphinxhref{https://docs.python.org/3/library/functions.html\#float}{float}

\end{description}\end{quote}

\end{fulllineitems}

\index{inner\_slug\_id (fuel\_bucket.FuelBucket attribute)}

\begin{fulllineitems}
\phantomsection\label{\detokenize{support_rst/fuel_bucket:fuel_bucket.FuelBucket.inner_slug_id}}\pysigline{\sphinxbfcode{\sphinxupquote{inner\_slug\_id}}}
Returns the inner diameter of the inner section of fuel, in cm.
\begin{quote}\begin{description}
\item[{Returns}] \leavevmode
\sphinxstylestrong{inner\_slug\_id}

\item[{Return type}] \leavevmode
\sphinxhref{https://docs.python.org/3/library/functions.html\#float}{float}

\end{description}\end{quote}

\end{fulllineitems}

\index{inner\_slug\_od (fuel\_bucket.FuelBucket attribute)}

\begin{fulllineitems}
\phantomsection\label{\detokenize{support_rst/fuel_bucket:fuel_bucket.FuelBucket.inner_slug_od}}\pysigline{\sphinxbfcode{\sphinxupquote{inner\_slug\_od}}}
Returns the outer  diameter of the inner section of fuel, in cm.
\begin{quote}\begin{description}
\item[{Returns}] \leavevmode
\sphinxstylestrong{inner\_slug\_od}

\item[{Return type}] \leavevmode
\sphinxhref{https://docs.python.org/3/library/functions.html\#float}{float}

\end{description}\end{quote}

\end{fulllineitems}

\index{n\_slugs (fuel\_bucket.FuelBucket attribute)}

\begin{fulllineitems}
\phantomsection\label{\detokenize{support_rst/fuel_bucket:fuel_bucket.FuelBucket.n_slugs}}\pysigline{\sphinxbfcode{\sphinxupquote{n\_slugs}}}
Returns the number of fuel slugs in the bucket.
\begin{quote}\begin{description}
\item[{Returns}] \leavevmode
\sphinxstylestrong{n\_slugs}

\item[{Return type}] \leavevmode
\sphinxhref{https://docs.python.org/3/library/functions.html\#int}{int}

\end{description}\end{quote}

\end{fulllineitems}

\index{name (fuel\_bucket.FuelBucket attribute)}

\begin{fulllineitems}
\phantomsection\label{\detokenize{support_rst/fuel_bucket:fuel_bucket.FuelBucket.name}}\pysigline{\sphinxbfcode{\sphinxupquote{name}}}
Returns the name of the fuel bucket.
\begin{quote}\begin{description}
\item[{Returns}] \leavevmode
\sphinxstylestrong{name}

\item[{Return type}] \leavevmode
\sphinxhref{https://docs.python.org/3/library/stdtypes.html\#str}{str}

\end{description}\end{quote}

\end{fulllineitems}

\index{outer\_slug\_id (fuel\_bucket.FuelBucket attribute)}

\begin{fulllineitems}
\phantomsection\label{\detokenize{support_rst/fuel_bucket:fuel_bucket.FuelBucket.outer_slug_id}}\pysigline{\sphinxbfcode{\sphinxupquote{outer\_slug\_id}}}
Returns the inner diameter of the outer section of fuel, in cm.
\begin{quote}\begin{description}
\item[{Returns}] \leavevmode
\sphinxstylestrong{outer\_slug\_id}

\item[{Return type}] \leavevmode
\sphinxhref{https://docs.python.org/3/library/functions.html\#float}{float}

\end{description}\end{quote}

\end{fulllineitems}

\index{outer\_slug\_od (fuel\_bucket.FuelBucket attribute)}

\begin{fulllineitems}
\phantomsection\label{\detokenize{support_rst/fuel_bucket:fuel_bucket.FuelBucket.outer_slug_od}}\pysigline{\sphinxbfcode{\sphinxupquote{outer\_slug\_od}}}
Returns the outer diameter of the outer section of fuel, in cm. A fuel
slug consists of an outer section of fuel and an inner section of fuel,
with cladding on the outside of the slug and between the inner and
outer sections of fuel.
\begin{quote}\begin{description}
\item[{Returns}] \leavevmode
\sphinxstylestrong{outer\_slug\_od}

\item[{Return type}] \leavevmode
\sphinxhref{https://docs.python.org/3/library/functions.html\#float}{float}

\end{description}\end{quote}

\end{fulllineitems}

\index{radioactivity (fuel\_bucket.FuelBucket attribute)}

\begin{fulllineitems}
\phantomsection\label{\detokenize{support_rst/fuel_bucket:fuel_bucket.FuelBucket.radioactivity}}\pysigline{\sphinxbfcode{\sphinxupquote{radioactivity}}}
Returns the radioactivity of the fuel bucket, in units of curies.
\begin{quote}\begin{description}
\item[{Returns}] \leavevmode
\sphinxstylestrong{radioactivity}

\item[{Return type}] \leavevmode
\sphinxhref{https://docs.python.org/3/library/functions.html\#float}{float}

\end{description}\end{quote}

\end{fulllineitems}

\index{set\_cladding\_phase() (fuel\_bucket.FuelBucket method)}

\begin{fulllineitems}
\phantomsection\label{\detokenize{support_rst/fuel_bucket:fuel_bucket.FuelBucket.set_cladding_phase}}\pysiglinewithargsret{\sphinxbfcode{\sphinxupquote{set\_cladding\_phase}}}{\emph{phase}}{}
Set’s the phase history to specific values.
\begin{quote}\begin{description}
\item[{Parameters}] \leavevmode
\sphinxstyleliteralstrong{\sphinxupquote{phase}} (\sphinxstyleliteralemphasis{\sphinxupquote{dataFrame}}) \textendash{} 

\end{description}\end{quote}

\end{fulllineitems}

\index{set\_fuel\_phase() (fuel\_bucket.FuelBucket method)}

\begin{fulllineitems}
\phantomsection\label{\detokenize{support_rst/fuel_bucket:fuel_bucket.FuelBucket.set_fuel_phase}}\pysiglinewithargsret{\sphinxbfcode{\sphinxupquote{set\_fuel\_phase}}}{\emph{phase}}{}
Sets the current fuel phase to a specified phase value.
\begin{quote}\begin{description}
\item[{Parameters}] \leavevmode
\sphinxstyleliteralstrong{\sphinxupquote{phase}} (\sphinxstyleliteralemphasis{\sphinxupquote{dataFrame}}) \textendash{} 

\end{description}\end{quote}

\end{fulllineitems}

\index{set\_slug\_length() (fuel\_bucket.FuelBucket method)}

\begin{fulllineitems}
\phantomsection\label{\detokenize{support_rst/fuel_bucket:fuel_bucket.FuelBucket.set_slug_length}}\pysiglinewithargsret{\sphinxbfcode{\sphinxupquote{set\_slug\_length}}}{\emph{x}}{}
Sets the length of all slugs in the bucket to x. Used for chopping.
\begin{quote}\begin{description}
\item[{Parameters}] \leavevmode
\sphinxstyleliteralstrong{\sphinxupquote{x}} (\sphinxhref{https://docs.python.org/3/library/functions.html\#float}{\sphinxstyleliteralemphasis{\sphinxupquote{float}}}) \textendash{} 

\end{description}\end{quote}

\end{fulllineitems}

\index{slug\_cladding\_volume (fuel\_bucket.FuelBucket attribute)}

\begin{fulllineitems}
\phantomsection\label{\detokenize{support_rst/fuel_bucket:fuel_bucket.FuelBucket.slug_cladding_volume}}\pysigline{\sphinxbfcode{\sphinxupquote{slug\_cladding\_volume}}}
Returns the volume of cladding present in a single fuel slug, in cm\textasciicircum{}3.
\begin{quote}\begin{description}
\item[{Returns}] \leavevmode
\sphinxstylestrong{slug\_cladding\_volume}

\item[{Return type}] \leavevmode
\sphinxhref{https://docs.python.org/3/library/functions.html\#float}{float}

\end{description}\end{quote}

\end{fulllineitems}

\index{slug\_fuel\_volume (fuel\_bucket.FuelBucket attribute)}

\begin{fulllineitems}
\phantomsection\label{\detokenize{support_rst/fuel_bucket:fuel_bucket.FuelBucket.slug_fuel_volume}}\pysigline{\sphinxbfcode{\sphinxupquote{slug\_fuel\_volume}}}
Returns the volume of fuel present in a single fuel slug, in cm\textasciicircum{}3.
\begin{quote}\begin{description}
\item[{Returns}] \leavevmode
\sphinxstylestrong{slug\_fuel\_volume}

\item[{Return type}] \leavevmode
\sphinxhref{https://docs.python.org/3/library/functions.html\#float}{float}

\end{description}\end{quote}

\end{fulllineitems}

\index{slug\_length (fuel\_bucket.FuelBucket attribute)}

\begin{fulllineitems}
\phantomsection\label{\detokenize{support_rst/fuel_bucket:fuel_bucket.FuelBucket.slug_length}}\pysigline{\sphinxbfcode{\sphinxupquote{slug\_length}}}
Returns the length of each slug in the fuel bucket.
\begin{quote}\begin{description}
\item[{Returns}] \leavevmode
\sphinxstylestrong{slug\_length}

\item[{Return type}] \leavevmode
\sphinxhref{https://docs.python.org/3/library/functions.html\#float}{float}

\end{description}\end{quote}

\end{fulllineitems}

\index{slug\_type (fuel\_bucket.FuelBucket attribute)}

\begin{fulllineitems}
\phantomsection\label{\detokenize{support_rst/fuel_bucket:fuel_bucket.FuelBucket.slug_type}}\pysigline{\sphinxbfcode{\sphinxupquote{slug\_type}}}
Returns the type of slugs being stored in the bucket (inner slug or
outer slug).
\begin{quote}\begin{description}
\item[{Returns}] \leavevmode
\sphinxstylestrong{slug\_type}

\item[{Return type}] \leavevmode
\sphinxhref{https://docs.python.org/3/library/stdtypes.html\#str}{str}

\end{description}\end{quote}

\end{fulllineitems}


\end{fulllineitems}



\section{fuel\_bundle}
\label{\detokenize{support_rst/fuel_bundle:module-fuel_bundle}}\label{\detokenize{support_rst/fuel_bundle:fuel-bundle}}\label{\detokenize{support_rst/fuel_bundle::doc}}\index{fuel\_bundle (module)}
This FuelBundle class is a container for usage with other plant-level process modules.
It is meant to represent a fuel bundle of an oxide fuel LWR reactor.
There are three main data structures:
\begin{enumerate}
\item {} 
fuel bundle specs

\item {} 
solid phase

\item {} 
gas phase

\end{enumerate}

The container user will have to provide all the data and from then on, this class
will help acess the data.
The printing methods reveal the contained data.

Author: Valmor de Almeida \sphinxhref{mailto:dealmeidav@ornl.gov}{dealmeidav@ornl.gov}; vfda
Sun Dec 27 15:06:55 EST 2015
\index{FuelBundle (class in fuel\_bundle)}

\begin{fulllineitems}
\phantomsection\label{\detokenize{support_rst/fuel_bundle:fuel_bundle.FuelBundle}}\pysiglinewithargsret{\sphinxbfcode{\sphinxupquote{class }}\sphinxcode{\sphinxupquote{fuel\_bundle.}}\sphinxbfcode{\sphinxupquote{FuelBundle}}}{\emph{specs=Empty DataFrame Columns: {[}{]} Index: {[}{]}}}{}
Bases: \sphinxhref{https://docs.python.org/3/library/functions.html\#object}{\sphinxcode{\sphinxupquote{object}}}
\index{fresh\_u235\_mass (fuel\_bundle.FuelBundle attribute)}

\begin{fulllineitems}
\phantomsection\label{\detokenize{support_rst/fuel_bundle:fuel_bundle.FuelBundle.fresh_u235_mass}}\pysigline{\sphinxbfcode{\sphinxupquote{fresh\_u235\_mass}}}
Returns the amount of uranium-235 in the bucket, in grams.
\begin{quote}\begin{description}
\item[{Returns}] \leavevmode
\sphinxstylestrong{fresh\_u235\_mass}

\item[{Return type}] \leavevmode
\sphinxhref{https://docs.python.org/3/library/functions.html\#float}{float}

\end{description}\end{quote}

\end{fulllineitems}

\index{fresh\_u238\_mass (fuel\_bundle.FuelBundle attribute)}

\begin{fulllineitems}
\phantomsection\label{\detokenize{support_rst/fuel_bundle:fuel_bundle.FuelBundle.fresh_u238_mass}}\pysigline{\sphinxbfcode{\sphinxupquote{fresh\_u238\_mass}}}
Returns the amount of uranium-238 in the bucket, in grams.
\begin{quote}\begin{description}
\item[{Returns}] \leavevmode
\sphinxstylestrong{fresh\_u238\_mass}

\item[{Return type}] \leavevmode
\sphinxhref{https://docs.python.org/3/library/functions.html\#float}{float}

\end{description}\end{quote}

\end{fulllineitems}

\index{fresh\_u\_mass (fuel\_bundle.FuelBundle attribute)}

\begin{fulllineitems}
\phantomsection\label{\detokenize{support_rst/fuel_bundle:fuel_bundle.FuelBundle.fresh_u_mass}}\pysigline{\sphinxbfcode{\sphinxupquote{fresh\_u\_mass}}}
Returns the amount of uranium in the bundle, in grams.
\begin{quote}\begin{description}
\item[{Returns}] \leavevmode
\sphinxstylestrong{fresh\_u\_mass}

\item[{Return type}] \leavevmode
\sphinxhref{https://docs.python.org/3/library/functions.html\#float}{float}

\end{description}\end{quote}

\end{fulllineitems}

\index{fuel\_enrichment (fuel\_bundle.FuelBundle attribute)}

\begin{fulllineitems}
\phantomsection\label{\detokenize{support_rst/fuel_bundle:fuel_bundle.FuelBundle.fuel_enrichment}}\pysigline{\sphinxbfcode{\sphinxupquote{fuel\_enrichment}}}
Returns the enrichment of the fuel pins in the bundle, in \%.
\begin{quote}\begin{description}
\item[{Returns}] \leavevmode
\sphinxstylestrong{fuel\_enrichment}

\item[{Return type}] \leavevmode
\sphinxhref{https://docs.python.org/3/library/functions.html\#float}{float}

\end{description}\end{quote}

\end{fulllineitems}

\index{fuel\_mass (fuel\_bundle.FuelBundle attribute)}

\begin{fulllineitems}
\phantomsection\label{\detokenize{support_rst/fuel_bundle:fuel_bundle.FuelBundle.fuel_mass}}\pysigline{\sphinxbfcode{\sphinxupquote{fuel\_mass}}}
Returns the total numerical value for  mass of fuel in the solid phase
in the bundle.
\begin{quote}\begin{description}
\item[{Returns}] \leavevmode
\sphinxstylestrong{fuel\_mass}

\item[{Return type}] \leavevmode
\sphinxhref{https://docs.python.org/3/library/functions.html\#float}{float}

\end{description}\end{quote}

\end{fulllineitems}

\index{fuel\_mass\_unit (fuel\_bundle.FuelBundle attribute)}

\begin{fulllineitems}
\phantomsection\label{\detokenize{support_rst/fuel_bundle:fuel_bundle.FuelBundle.fuel_mass_unit}}\pysigline{\sphinxbfcode{\sphinxupquote{fuel\_mass\_unit}}}
Returns the unit used to measure the mass of fuel in the bundle.
\begin{quote}\begin{description}
\item[{Returns}] \leavevmode
\sphinxstylestrong{fuel\_mass\_unit}

\item[{Return type}] \leavevmode
\sphinxhref{https://docs.python.org/3/library/stdtypes.html\#str}{str}

\end{description}\end{quote}

\end{fulllineitems}

\index{fuel\_pin\_length (fuel\_bundle.FuelBundle attribute)}

\begin{fulllineitems}
\phantomsection\label{\detokenize{support_rst/fuel_bundle:fuel_bundle.FuelBundle.fuel_pin_length}}\pysigline{\sphinxbfcode{\sphinxupquote{fuel\_pin\_length}}}
Returns the length of each fuel pin in the fuel bundle. A fuel pin is
a cylindircal section of uranium fuel that is surrounded by cladding.
\begin{quote}\begin{description}
\item[{Returns}] \leavevmode
\sphinxstylestrong{fuel\_pin\_length}

\item[{Return type}] \leavevmode
\sphinxhref{https://docs.python.org/3/library/functions.html\#float}{float}

\end{description}\end{quote}

\end{fulllineitems}

\index{fuel\_pin\_radius (fuel\_bundle.FuelBundle attribute)}

\begin{fulllineitems}
\phantomsection\label{\detokenize{support_rst/fuel_bundle:fuel_bundle.FuelBundle.fuel_pin_radius}}\pysigline{\sphinxbfcode{\sphinxupquote{fuel\_pin\_radius}}}
Returns the radius of the fuel pin, in cm.

\end{fulllineitems}

\index{fuel\_pin\_volume (fuel\_bundle.FuelBundle attribute)}

\begin{fulllineitems}
\phantomsection\label{\detokenize{support_rst/fuel_bundle:fuel_bundle.FuelBundle.fuel_pin_volume}}\pysigline{\sphinxbfcode{\sphinxupquote{fuel\_pin\_volume}}}
Returns the volume of fuel in each fuel pin, in cm\textasciicircum{}3.
\begin{quote}\begin{description}
\item[{Returns}] \leavevmode
\sphinxstylestrong{fuel\_pin\_volume}

\item[{Return type}] \leavevmode
\sphinxhref{https://docs.python.org/3/library/functions.html\#float}{float}

\end{description}\end{quote}

\end{fulllineitems}

\index{fuel\_radioactivity (fuel\_bundle.FuelBundle attribute)}

\begin{fulllineitems}
\phantomsection\label{\detokenize{support_rst/fuel_bundle:fuel_bundle.FuelBundle.fuel_radioactivity}}\pysigline{\sphinxbfcode{\sphinxupquote{fuel\_radioactivity}}}
Returns the total radioactivity of the fuel in the solid phase in the
fuel bundle.
\begin{quote}\begin{description}
\item[{Returns}] \leavevmode
\sphinxstylestrong{fuel\_radioactivity}

\item[{Return type}] \leavevmode
\sphinxhref{https://docs.python.org/3/library/functions.html\#float}{float}

\end{description}\end{quote}

\end{fulllineitems}

\index{fuel\_rod\_od (fuel\_bundle.FuelBundle attribute)}

\begin{fulllineitems}
\phantomsection\label{\detokenize{support_rst/fuel_bundle:fuel_bundle.FuelBundle.fuel_rod_od}}\pysigline{\sphinxbfcode{\sphinxupquote{fuel\_rod\_od}}}
Returns the outer diameter of the fuel rod, in cm.
A fuel rod consists of a fuel pin surrounded by cladding.
\begin{quote}\begin{description}
\item[{Returns}] \leavevmode
\sphinxstylestrong{fuel\_rod\_od}

\item[{Return type}] \leavevmode
\sphinxhref{https://docs.python.org/3/library/functions.html\#float}{float}

\end{description}\end{quote}

\end{fulllineitems}

\index{fuel\_volume (fuel\_bundle.FuelBundle attribute)}

\begin{fulllineitems}
\phantomsection\label{\detokenize{support_rst/fuel_bundle:fuel_bundle.FuelBundle.fuel_volume}}\pysigline{\sphinxbfcode{\sphinxupquote{fuel\_volume}}}
Returns the total volume of fuel in the bundle, in cm\textasciicircum{}3.
\begin{quote}\begin{description}
\item[{Returns}] \leavevmode
\sphinxstylestrong{fuel\_volume}

\item[{Return type}] \leavevmode
\sphinxhref{https://docs.python.org/3/library/functions.html\#float}{float}

\end{description}\end{quote}

\end{fulllineitems}

\index{gamma\_pwr (fuel\_bundle.FuelBundle attribute)}

\begin{fulllineitems}
\phantomsection\label{\detokenize{support_rst/fuel_bundle:fuel_bundle.FuelBundle.gamma_pwr}}\pysigline{\sphinxbfcode{\sphinxupquote{gamma\_pwr}}}
Returns the total amount of gamma radiation given by the fuel bundle,
in watts.
\begin{quote}\begin{description}
\item[{Returns}] \leavevmode
\sphinxstylestrong{gamma\_pwr}

\item[{Return type}] \leavevmode
\sphinxhref{https://docs.python.org/3/library/functions.html\#float}{float}

\end{description}\end{quote}

\end{fulllineitems}

\index{gas\_mass (fuel\_bundle.FuelBundle attribute)}

\begin{fulllineitems}
\phantomsection\label{\detokenize{support_rst/fuel_bundle:fuel_bundle.FuelBundle.gas_mass}}\pysigline{\sphinxbfcode{\sphinxupquote{gas\_mass}}}
Returns the total numerical value for mass of the fuel in the gas
phase.

\end{fulllineitems}

\index{gas\_phase (fuel\_bundle.FuelBundle attribute)}

\begin{fulllineitems}
\phantomsection\label{\detokenize{support_rst/fuel_bundle:fuel_bundle.FuelBundle.gas_phase}}\pysigline{\sphinxbfcode{\sphinxupquote{gas\_phase}}}
Returns the gas phase history of the fuel.
\begin{quote}\begin{description}
\item[{Returns}] \leavevmode
\sphinxstylestrong{gas\_phase}

\item[{Return type}] \leavevmode
dataFrame

\end{description}\end{quote}

\end{fulllineitems}

\index{gas\_radioactivity (fuel\_bundle.FuelBundle attribute)}

\begin{fulllineitems}
\phantomsection\label{\detokenize{support_rst/fuel_bundle:fuel_bundle.FuelBundle.gas_radioactivity}}\pysigline{\sphinxbfcode{\sphinxupquote{gas\_radioactivity}}}
Returns the total radioactivity of the fuel in the gas phase in the
fuel bundle, in curies.
\begin{quote}\begin{description}
\item[{Returns}] \leavevmode
\sphinxstylestrong{gas\_radioactivity}

\item[{Return type}] \leavevmode
\sphinxhref{https://docs.python.org/3/library/functions.html\#float}{float}

\end{description}\end{quote}

\end{fulllineitems}

\index{get\_fresh\_U235\_mass() (fuel\_bundle.FuelBundle method)}

\begin{fulllineitems}
\phantomsection\label{\detokenize{support_rst/fuel_bundle:fuel_bundle.FuelBundle.get_fresh_U235_mass}}\pysiglinewithargsret{\sphinxbfcode{\sphinxupquote{get\_fresh\_U235\_mass}}}{}{}
Returns the amount of uranium-235 in the bucket, in grams.
\begin{quote}\begin{description}
\item[{Returns}] \leavevmode
\sphinxstylestrong{fresh\_u235\_mass}

\item[{Return type}] \leavevmode
\sphinxhref{https://docs.python.org/3/library/functions.html\#float}{float}

\end{description}\end{quote}

\end{fulllineitems}

\index{get\_fresh\_u238\_mass() (fuel\_bundle.FuelBundle method)}

\begin{fulllineitems}
\phantomsection\label{\detokenize{support_rst/fuel_bundle:fuel_bundle.FuelBundle.get_fresh_u238_mass}}\pysiglinewithargsret{\sphinxbfcode{\sphinxupquote{get\_fresh\_u238\_mass}}}{}{}
Returns the amount of uranium-238 in the bucket, in grams.
\begin{quote}\begin{description}
\item[{Returns}] \leavevmode
\sphinxstylestrong{fresh\_u238\_mass}

\item[{Return type}] \leavevmode
\sphinxhref{https://docs.python.org/3/library/functions.html\#float}{float}

\end{description}\end{quote}

\end{fulllineitems}

\index{get\_fresh\_u\_mass() (fuel\_bundle.FuelBundle method)}

\begin{fulllineitems}
\phantomsection\label{\detokenize{support_rst/fuel_bundle:fuel_bundle.FuelBundle.get_fresh_u_mass}}\pysiglinewithargsret{\sphinxbfcode{\sphinxupquote{get\_fresh\_u\_mass}}}{}{}
Returns the amount of uranium in the bundle, in grams.
\begin{quote}\begin{description}
\item[{Returns}] \leavevmode
\sphinxstylestrong{fresh\_u\_mass}

\item[{Return type}] \leavevmode
\sphinxhref{https://docs.python.org/3/library/functions.html\#float}{float}

\end{description}\end{quote}

\end{fulllineitems}

\index{get\_fuel\_enrichment() (fuel\_bundle.FuelBundle method)}

\begin{fulllineitems}
\phantomsection\label{\detokenize{support_rst/fuel_bundle:fuel_bundle.FuelBundle.get_fuel_enrichment}}\pysiglinewithargsret{\sphinxbfcode{\sphinxupquote{get\_fuel\_enrichment}}}{}{}
Returns the enrichment of the fuel pins in the bundle, in \%.
\begin{quote}\begin{description}
\item[{Returns}] \leavevmode
\sphinxstylestrong{fuel\_enrichment}

\item[{Return type}] \leavevmode
\sphinxhref{https://docs.python.org/3/library/functions.html\#float}{float}

\end{description}\end{quote}

\end{fulllineitems}

\index{get\_fuel\_mass() (fuel\_bundle.FuelBundle method)}

\begin{fulllineitems}
\phantomsection\label{\detokenize{support_rst/fuel_bundle:fuel_bundle.FuelBundle.get_fuel_mass}}\pysiglinewithargsret{\sphinxbfcode{\sphinxupquote{get\_fuel\_mass}}}{}{}
Returns the total numerical value for  mass of fuel in the solid phase
in the bundle.
\begin{quote}\begin{description}
\item[{Returns}] \leavevmode
\sphinxstylestrong{fuel\_mass}

\item[{Return type}] \leavevmode
\sphinxhref{https://docs.python.org/3/library/functions.html\#float}{float}

\end{description}\end{quote}

\end{fulllineitems}

\index{get\_fuel\_mass\_unit() (fuel\_bundle.FuelBundle method)}

\begin{fulllineitems}
\phantomsection\label{\detokenize{support_rst/fuel_bundle:fuel_bundle.FuelBundle.get_fuel_mass_unit}}\pysiglinewithargsret{\sphinxbfcode{\sphinxupquote{get\_fuel\_mass\_unit}}}{}{}
Returns the unit used to measure the mass of fuel in the bundle.
\begin{quote}\begin{description}
\item[{Returns}] \leavevmode
\sphinxstylestrong{fuel\_mass\_unit}

\item[{Return type}] \leavevmode
\sphinxhref{https://docs.python.org/3/library/stdtypes.html\#str}{str}

\end{description}\end{quote}

\end{fulllineitems}

\index{get\_fuel\_pin\_length() (fuel\_bundle.FuelBundle method)}

\begin{fulllineitems}
\phantomsection\label{\detokenize{support_rst/fuel_bundle:fuel_bundle.FuelBundle.get_fuel_pin_length}}\pysiglinewithargsret{\sphinxbfcode{\sphinxupquote{get\_fuel\_pin\_length}}}{}{}
Returns the length of each fuel pin in the fuel bundle. A fuel pin is
a cylindircal section of uranium fuel that is surrounded by cladding.
\begin{quote}\begin{description}
\item[{Returns}] \leavevmode
\sphinxstylestrong{fuel\_pin\_length}

\item[{Return type}] \leavevmode
\sphinxhref{https://docs.python.org/3/library/functions.html\#float}{float}

\end{description}\end{quote}

\end{fulllineitems}

\index{get\_fuel\_pin\_radius() (fuel\_bundle.FuelBundle method)}

\begin{fulllineitems}
\phantomsection\label{\detokenize{support_rst/fuel_bundle:fuel_bundle.FuelBundle.get_fuel_pin_radius}}\pysiglinewithargsret{\sphinxbfcode{\sphinxupquote{get\_fuel\_pin\_radius}}}{}{}
Returns the radius of the fuel pin, in cm.

\end{fulllineitems}

\index{get\_fuel\_pin\_volume() (fuel\_bundle.FuelBundle method)}

\begin{fulllineitems}
\phantomsection\label{\detokenize{support_rst/fuel_bundle:fuel_bundle.FuelBundle.get_fuel_pin_volume}}\pysiglinewithargsret{\sphinxbfcode{\sphinxupquote{get\_fuel\_pin\_volume}}}{}{}
Returns the volume of fuel in each fuel pin, in cm\textasciicircum{}3.
\begin{quote}\begin{description}
\item[{Returns}] \leavevmode
\sphinxstylestrong{fuel\_pin\_volume}

\item[{Return type}] \leavevmode
\sphinxhref{https://docs.python.org/3/library/functions.html\#float}{float}

\end{description}\end{quote}

\end{fulllineitems}

\index{get\_fuel\_radioactivity() (fuel\_bundle.FuelBundle method)}

\begin{fulllineitems}
\phantomsection\label{\detokenize{support_rst/fuel_bundle:fuel_bundle.FuelBundle.get_fuel_radioactivity}}\pysiglinewithargsret{\sphinxbfcode{\sphinxupquote{get\_fuel\_radioactivity}}}{}{}
Returns the total radioactivity of the fuel in the solid phase in the
fuel bundle.
\begin{quote}\begin{description}
\item[{Returns}] \leavevmode
\sphinxstylestrong{fuel\_radioactivity}

\item[{Return type}] \leavevmode
\sphinxhref{https://docs.python.org/3/library/functions.html\#float}{float}

\end{description}\end{quote}

\end{fulllineitems}

\index{get\_fuel\_rod\_od() (fuel\_bundle.FuelBundle method)}

\begin{fulllineitems}
\phantomsection\label{\detokenize{support_rst/fuel_bundle:fuel_bundle.FuelBundle.get_fuel_rod_od}}\pysiglinewithargsret{\sphinxbfcode{\sphinxupquote{get\_fuel\_rod\_od}}}{}{}
Returns the outer diameter of the fuel rod, in cm.
A fuel rod consists of a fuel pin surrounded by cladding.
\begin{quote}\begin{description}
\item[{Returns}] \leavevmode
\sphinxstylestrong{fuel\_rod\_od}

\item[{Return type}] \leavevmode
\sphinxhref{https://docs.python.org/3/library/functions.html\#float}{float}

\end{description}\end{quote}

\end{fulllineitems}

\index{get\_fuel\_volume() (fuel\_bundle.FuelBundle method)}

\begin{fulllineitems}
\phantomsection\label{\detokenize{support_rst/fuel_bundle:fuel_bundle.FuelBundle.get_fuel_volume}}\pysiglinewithargsret{\sphinxbfcode{\sphinxupquote{get\_fuel\_volume}}}{}{}
Returns the total volume of fuel in the bundle, in cm\textasciicircum{}3.
\begin{quote}\begin{description}
\item[{Returns}] \leavevmode
\sphinxstylestrong{fuel\_volume}

\item[{Return type}] \leavevmode
\sphinxhref{https://docs.python.org/3/library/functions.html\#float}{float}

\end{description}\end{quote}

\end{fulllineitems}

\index{get\_gamma\_pwr() (fuel\_bundle.FuelBundle method)}

\begin{fulllineitems}
\phantomsection\label{\detokenize{support_rst/fuel_bundle:fuel_bundle.FuelBundle.get_gamma_pwr}}\pysiglinewithargsret{\sphinxbfcode{\sphinxupquote{get\_gamma\_pwr}}}{}{}
Returns the total amount of gamma radiation given by the fuel bundle,
in watts.
\begin{quote}\begin{description}
\item[{Returns}] \leavevmode
\sphinxstylestrong{gamma\_pwr}

\item[{Return type}] \leavevmode
\sphinxhref{https://docs.python.org/3/library/functions.html\#float}{float}

\end{description}\end{quote}

\end{fulllineitems}

\index{get\_gas\_mass() (fuel\_bundle.FuelBundle method)}

\begin{fulllineitems}
\phantomsection\label{\detokenize{support_rst/fuel_bundle:fuel_bundle.FuelBundle.get_gas_mass}}\pysiglinewithargsret{\sphinxbfcode{\sphinxupquote{get\_gas\_mass}}}{}{}
Returns the total numerical value for mass of the fuel in the gas
phase.

\end{fulllineitems}

\index{get\_gas\_phase() (fuel\_bundle.FuelBundle method)}

\begin{fulllineitems}
\phantomsection\label{\detokenize{support_rst/fuel_bundle:fuel_bundle.FuelBundle.get_gas_phase}}\pysiglinewithargsret{\sphinxbfcode{\sphinxupquote{get\_gas\_phase}}}{}{}
Returns the gas phase history of the fuel.
\begin{quote}\begin{description}
\item[{Returns}] \leavevmode
\sphinxstylestrong{gas\_phase}

\item[{Return type}] \leavevmode
dataFrame

\end{description}\end{quote}

\end{fulllineitems}

\index{get\_gas\_radioactivity() (fuel\_bundle.FuelBundle method)}

\begin{fulllineitems}
\phantomsection\label{\detokenize{support_rst/fuel_bundle:fuel_bundle.FuelBundle.get_gas_radioactivity}}\pysiglinewithargsret{\sphinxbfcode{\sphinxupquote{get\_gas\_radioactivity}}}{}{}
Returns the total radioactivity of the fuel in the gas phase in the
fuel bundle, in curies.
\begin{quote}\begin{description}
\item[{Returns}] \leavevmode
\sphinxstylestrong{gas\_radioactivity}

\item[{Return type}] \leavevmode
\sphinxhref{https://docs.python.org/3/library/functions.html\#float}{float}

\end{description}\end{quote}

\end{fulllineitems}

\index{get\_heat\_pwr() (fuel\_bundle.FuelBundle method)}

\begin{fulllineitems}
\phantomsection\label{\detokenize{support_rst/fuel_bundle:fuel_bundle.FuelBundle.get_heat_pwr}}\pysiglinewithargsret{\sphinxbfcode{\sphinxupquote{get\_heat\_pwr}}}{}{}
Returns the total amount of heat produced by the fuel bundle, in watts.
\begin{quote}\begin{description}
\item[{Returns}] \leavevmode
\sphinxstylestrong{heat\_pwr}

\item[{Return type}] \leavevmode
\sphinxhref{https://docs.python.org/3/library/functions.html\#float}{float}

\end{description}\end{quote}

\end{fulllineitems}

\index{get\_n\_fuel\_rods() (fuel\_bundle.FuelBundle method)}

\begin{fulllineitems}
\phantomsection\label{\detokenize{support_rst/fuel_bundle:fuel_bundle.FuelBundle.get_n_fuel_rods}}\pysiglinewithargsret{\sphinxbfcode{\sphinxupquote{get\_n\_fuel\_rods}}}{}{}
Returns the number of fuel rods in the bundle.
\begin{quote}\begin{description}
\item[{Returns}] \leavevmode
\sphinxstylestrong{n\_fuel\_rods}

\item[{Return type}] \leavevmode
\sphinxhref{https://docs.python.org/3/library/functions.html\#int}{int}

\end{description}\end{quote}

\end{fulllineitems}

\index{get\_name() (fuel\_bundle.FuelBundle method)}

\begin{fulllineitems}
\phantomsection\label{\detokenize{support_rst/fuel_bundle:fuel_bundle.FuelBundle.get_name}}\pysiglinewithargsret{\sphinxbfcode{\sphinxupquote{get\_name}}}{}{}
Returns the name of the fuel bundle.
\begin{quote}\begin{description}
\item[{Returns}] \leavevmode
\sphinxstylestrong{name}

\item[{Return type}] \leavevmode
\sphinxhref{https://docs.python.org/3/library/stdtypes.html\#str}{str}

\end{description}\end{quote}

\end{fulllineitems}

\index{get\_radioactivity() (fuel\_bundle.FuelBundle method)}

\begin{fulllineitems}
\phantomsection\label{\detokenize{support_rst/fuel_bundle:fuel_bundle.FuelBundle.get_radioactivity}}\pysiglinewithargsret{\sphinxbfcode{\sphinxupquote{get\_radioactivity}}}{}{}
Returns the total radioactivity of the fuel bundle, in curies.
\begin{quote}\begin{description}
\item[{Returns}] \leavevmode
\sphinxstylestrong{raduioactivity}

\item[{Return type}] \leavevmode
\sphinxhref{https://docs.python.org/3/library/functions.html\#float}{float}

\end{description}\end{quote}

\end{fulllineitems}

\index{get\_solid\_phase() (fuel\_bundle.FuelBundle method)}

\begin{fulllineitems}
\phantomsection\label{\detokenize{support_rst/fuel_bundle:fuel_bundle.FuelBundle.get_solid_phase}}\pysiglinewithargsret{\sphinxbfcode{\sphinxupquote{get\_solid\_phase}}}{}{}
Returns the solid phase history associated with this fuel bundle.
\begin{quote}\begin{description}
\item[{Returns}] \leavevmode
\sphinxstylestrong{solidPhase}

\item[{Return type}] \leavevmode
dataFrame

\end{description}\end{quote}

\end{fulllineitems}

\index{heat\_pwr (fuel\_bundle.FuelBundle attribute)}

\begin{fulllineitems}
\phantomsection\label{\detokenize{support_rst/fuel_bundle:fuel_bundle.FuelBundle.heat_pwr}}\pysigline{\sphinxbfcode{\sphinxupquote{heat\_pwr}}}
Returns the total amount of heat produced by the fuel bundle, in watts.
\begin{quote}\begin{description}
\item[{Returns}] \leavevmode
\sphinxstylestrong{heat\_pwr}

\item[{Return type}] \leavevmode
\sphinxhref{https://docs.python.org/3/library/functions.html\#float}{float}

\end{description}\end{quote}

\end{fulllineitems}

\index{n\_fuel\_rods (fuel\_bundle.FuelBundle attribute)}

\begin{fulllineitems}
\phantomsection\label{\detokenize{support_rst/fuel_bundle:fuel_bundle.FuelBundle.n_fuel_rods}}\pysigline{\sphinxbfcode{\sphinxupquote{n\_fuel\_rods}}}
Returns the number of fuel rods in the bundle.
\begin{quote}\begin{description}
\item[{Returns}] \leavevmode
\sphinxstylestrong{n\_fuel\_rods}

\item[{Return type}] \leavevmode
\sphinxhref{https://docs.python.org/3/library/functions.html\#int}{int}

\end{description}\end{quote}

\end{fulllineitems}

\index{name (fuel\_bundle.FuelBundle attribute)}

\begin{fulllineitems}
\phantomsection\label{\detokenize{support_rst/fuel_bundle:fuel_bundle.FuelBundle.name}}\pysigline{\sphinxbfcode{\sphinxupquote{name}}}
Returns the name of the fuel bundle.
\begin{quote}\begin{description}
\item[{Returns}] \leavevmode
\sphinxstylestrong{name}

\item[{Return type}] \leavevmode
\sphinxhref{https://docs.python.org/3/library/stdtypes.html\#str}{str}

\end{description}\end{quote}

\end{fulllineitems}

\index{radioactivity (fuel\_bundle.FuelBundle attribute)}

\begin{fulllineitems}
\phantomsection\label{\detokenize{support_rst/fuel_bundle:fuel_bundle.FuelBundle.radioactivity}}\pysigline{\sphinxbfcode{\sphinxupquote{radioactivity}}}
Returns the total radioactivity of the fuel bundle, in curies.
\begin{quote}\begin{description}
\item[{Returns}] \leavevmode
\sphinxstylestrong{raduioactivity}

\item[{Return type}] \leavevmode
\sphinxhref{https://docs.python.org/3/library/functions.html\#float}{float}

\end{description}\end{quote}

\end{fulllineitems}

\index{set\_fuel\_pin\_length() (fuel\_bundle.FuelBundle method)}

\begin{fulllineitems}
\phantomsection\label{\detokenize{support_rst/fuel_bundle:fuel_bundle.FuelBundle.set_fuel_pin_length}}\pysiglinewithargsret{\sphinxbfcode{\sphinxupquote{set\_fuel\_pin\_length}}}{\emph{x}}{}
Sets the length of all fuel pins in the bundle to x.
\begin{quote}\begin{description}
\item[{Returns}] \leavevmode
\sphinxstylestrong{x}

\item[{Return type}] \leavevmode
\sphinxhref{https://docs.python.org/3/library/functions.html\#float}{float}

\end{description}\end{quote}

\end{fulllineitems}

\index{set\_gas\_phase() (fuel\_bundle.FuelBundle method)}

\begin{fulllineitems}
\phantomsection\label{\detokenize{support_rst/fuel_bundle:fuel_bundle.FuelBundle.set_gas_phase}}\pysiglinewithargsret{\sphinxbfcode{\sphinxupquote{set\_gas\_phase}}}{\emph{phase}}{}
Sets the gas phase history of the fuel equal to phase.
\begin{quote}\begin{description}
\item[{Parameters}] \leavevmode
\sphinxstyleliteralstrong{\sphinxupquote{phase}} (\sphinxstyleliteralemphasis{\sphinxupquote{dataFrame}}) \textendash{} 

\end{description}\end{quote}

\end{fulllineitems}

\index{set\_solid\_phase() (fuel\_bundle.FuelBundle method)}

\begin{fulllineitems}
\phantomsection\label{\detokenize{support_rst/fuel_bundle:fuel_bundle.FuelBundle.set_solid_phase}}\pysiglinewithargsret{\sphinxbfcode{\sphinxupquote{set\_solid\_phase}}}{\emph{phase}}{}
Sets the solid phase history of the fuel equal to phase.
\begin{quote}\begin{description}
\item[{Parameters}] \leavevmode
\sphinxstyleliteralstrong{\sphinxupquote{phase}} (\sphinxstyleliteralemphasis{\sphinxupquote{dataFrame}}) \textendash{} 

\end{description}\end{quote}

\end{fulllineitems}

\index{solid\_phase (fuel\_bundle.FuelBundle attribute)}

\begin{fulllineitems}
\phantomsection\label{\detokenize{support_rst/fuel_bundle:fuel_bundle.FuelBundle.solid_phase}}\pysigline{\sphinxbfcode{\sphinxupquote{solid\_phase}}}
Returns the solid phase history associated with this fuel bundle.
\begin{quote}\begin{description}
\item[{Returns}] \leavevmode
\sphinxstylestrong{solidPhase}

\item[{Return type}] \leavevmode
dataFrame

\end{description}\end{quote}

\end{fulllineitems}


\end{fulllineitems}



\section{fuel\_segment}
\label{\detokenize{support_rst/fuel_segment:module-fuel_segment}}\label{\detokenize{support_rst/fuel_segment:fuel-segment}}\label{\detokenize{support_rst/fuel_segment::doc}}\index{fuel\_segment (module)}
Fuel segment
Author: Valmor de Almeida \sphinxhref{mailto:dealmeidav@ornl.gov}{dealmeidav@ornl.gov}; vfda
Sat Jun 27 14:46:49 EDT 2015
\index{FuelSegment (class in fuel\_segment)}

\begin{fulllineitems}
\phantomsection\label{\detokenize{support_rst/fuel_segment:fuel_segment.FuelSegment}}\pysiglinewithargsret{\sphinxbfcode{\sphinxupquote{class }}\sphinxcode{\sphinxupquote{fuel\_segment.}}\sphinxbfcode{\sphinxupquote{FuelSegment}}}{\emph{geometry=Series({[}{]}}, \emph{dtype: float64)}, \emph{species={[}{]}}}{}
Bases: \sphinxhref{https://docs.python.org/3/library/functions.html\#object}{\sphinxcode{\sphinxupquote{object}}}
\index{\_\_repr\_\_() (fuel\_segment.FuelSegment method)}

\begin{fulllineitems}
\phantomsection\label{\detokenize{support_rst/fuel_segment:fuel_segment.FuelSegment.__repr__}}\pysiglinewithargsret{\sphinxbfcode{\sphinxupquote{\_\_repr\_\_}}}{}{}
Used to pront the geometry of the fuel segment and the species that it
consists of.
\begin{quote}\begin{description}
\item[{Returns}] \leavevmode
\sphinxstylestrong{s}

\item[{Return type}] \leavevmode
\sphinxhref{https://docs.python.org/3/library/stdtypes.html\#str}{str}

\end{description}\end{quote}

\end{fulllineitems}

\index{\_\_str\_\_() (fuel\_segment.FuelSegment method)}

\begin{fulllineitems}
\phantomsection\label{\detokenize{support_rst/fuel_segment:fuel_segment.FuelSegment.__str__}}\pysiglinewithargsret{\sphinxbfcode{\sphinxupquote{\_\_str\_\_}}}{}{}
Used to print the geometry of the fuel segment and the species that it
consists of.
\begin{quote}\begin{description}
\item[{Returns}] \leavevmode
\sphinxstylestrong{s}

\item[{Return type}] \leavevmode
\sphinxhref{https://docs.python.org/3/library/stdtypes.html\#str}{str}

\end{description}\end{quote}

\end{fulllineitems}

\index{geometry (fuel\_segment.FuelSegment attribute)}

\begin{fulllineitems}
\phantomsection\label{\detokenize{support_rst/fuel_segment:fuel_segment.FuelSegment.geometry}}\pysigline{\sphinxbfcode{\sphinxupquote{geometry}}}
Returns the geometry of the fuel bundle (cylindrical, hexoganol,
rectangular, etc).
\begin{quote}\begin{description}
\item[{Returns}] \leavevmode
\sphinxstylestrong{geometry}

\item[{Return type}] \leavevmode
\sphinxhref{https://docs.python.org/3/library/stdtypes.html\#str}{str}

\end{description}\end{quote}

\end{fulllineitems}

\index{get\_attribute() (fuel\_segment.FuelSegment method)}

\begin{fulllineitems}
\phantomsection\label{\detokenize{support_rst/fuel_segment:fuel_segment.FuelSegment.get_attribute}}\pysiglinewithargsret{\sphinxbfcode{\sphinxupquote{get\_attribute}}}{\emph{name}, \emph{nuclide=None}, \emph{series=None}}{}
Used to get stored fuel segment properties, either overall (as an
average), or on a nuclide basis. “name” in this case refers to the
attribute in question. At this point in time, series is not implemented
and passing it to this function will result in an error. Possible
attributes that may be retrieved with this function, as well as the
name to pass to this function to retrieve them are: number of segments
in the bundle (n-segments, always equal to 1), the id of the segment
that makes up the bundle (segment-id), the volume of the fuel in the
bundle  (fuel-volume), the total volume of the segment
(segment-volume), the diameter (fuel-diameter) and length
(fuel-length) of the segment, the mass or mass density of the segment
(mass or mass-cc, respectively), or the total or per-volume
radioactivity, gamma radiation density or heat density of the fuel
segment (radioactivity and radioactivityDens, gamma and gamma-dens,
and heat and heat-dens, respectively).

Finally, density or total mass of a specific nuclide can be determined
by passing a specific nuclide to the function, with a name value of
mass or mass-cc.
\begin{quote}\begin{description}
\item[{Parameters}] \leavevmode\begin{itemize}
\item {} 
\sphinxstyleliteralstrong{\sphinxupquote{name}} (\sphinxhref{https://docs.python.org/3/library/stdtypes.html\#str}{\sphinxstyleliteralemphasis{\sphinxupquote{str}}}) \textendash{} 

\item {} 
\sphinxstyleliteralstrong{\sphinxupquote{nuclide}} (\sphinxhref{https://docs.python.org/3/library/stdtypes.html\#str}{\sphinxstyleliteralemphasis{\sphinxupquote{str}}}) \textendash{} 

\end{itemize}

\item[{Returns}] \leavevmode


\item[{Return type}] \leavevmode
many types

\end{description}\end{quote}

\end{fulllineitems}

\index{get\_geometry() (fuel\_segment.FuelSegment method)}

\begin{fulllineitems}
\phantomsection\label{\detokenize{support_rst/fuel_segment:fuel_segment.FuelSegment.get_geometry}}\pysiglinewithargsret{\sphinxbfcode{\sphinxupquote{get\_geometry}}}{}{}
Returns the geometry of the fuel bundle (cylindrical, hexoganol,
rectangular, etc).
\begin{quote}\begin{description}
\item[{Returns}] \leavevmode
\sphinxstylestrong{geometry}

\item[{Return type}] \leavevmode
\sphinxhref{https://docs.python.org/3/library/stdtypes.html\#str}{str}

\end{description}\end{quote}

\end{fulllineitems}

\index{get\_specie() (fuel\_segment.FuelSegment method)}

\begin{fulllineitems}
\phantomsection\label{\detokenize{support_rst/fuel_segment:fuel_segment.FuelSegment.get_specie}}\pysiglinewithargsret{\sphinxbfcode{\sphinxupquote{get\_specie}}}{\emph{name}}{}
Returns a specie named {[}name{]} from the list of species making up the
fuel bundle. If no name is specified, this function will return None.
\begin{quote}\begin{description}
\item[{Parameters}] \leavevmode
\sphinxstyleliteralstrong{\sphinxupquote{name}} (\sphinxhref{https://docs.python.org/3/library/stdtypes.html\#str}{\sphinxstyleliteralemphasis{\sphinxupquote{str}}}) \textendash{} 

\item[{Returns}] \leavevmode
\sphinxstylestrong{specie}

\item[{Return type}] \leavevmode
obj

\end{description}\end{quote}

\end{fulllineitems}

\index{get\_species() (fuel\_segment.FuelSegment method)}

\begin{fulllineitems}
\phantomsection\label{\detokenize{support_rst/fuel_segment:fuel_segment.FuelSegment.get_species}}\pysiglinewithargsret{\sphinxbfcode{\sphinxupquote{get\_species}}}{}{}
Returns the species object which describes the composition of the fuel
bundle. The species encapsulates all chemical species present in the
fuel bundle.
\begin{quote}\begin{description}
\item[{Returns}] \leavevmode
\sphinxstylestrong{species}

\item[{Return type}] \leavevmode
\sphinxhref{https://docs.python.org/3/library/functions.html\#object}{object}

\end{description}\end{quote}

\end{fulllineitems}

\index{specie (fuel\_segment.FuelSegment attribute)}

\begin{fulllineitems}
\phantomsection\label{\detokenize{support_rst/fuel_segment:fuel_segment.FuelSegment.specie}}\pysigline{\sphinxbfcode{\sphinxupquote{specie}}}
Returns a specie named {[}name{]} from the list of species making up the
fuel bundle. If no name is specified, this function will return None.
\begin{quote}\begin{description}
\item[{Parameters}] \leavevmode
\sphinxstyleliteralstrong{\sphinxupquote{name}} (\sphinxhref{https://docs.python.org/3/library/stdtypes.html\#str}{\sphinxstyleliteralemphasis{\sphinxupquote{str}}}) \textendash{} 

\item[{Returns}] \leavevmode
\sphinxstylestrong{specie}

\item[{Return type}] \leavevmode
obj

\end{description}\end{quote}

\end{fulllineitems}

\index{species (fuel\_segment.FuelSegment attribute)}

\begin{fulllineitems}
\phantomsection\label{\detokenize{support_rst/fuel_segment:fuel_segment.FuelSegment.species}}\pysigline{\sphinxbfcode{\sphinxupquote{species}}}
Returns the species object which describes the composition of the fuel
bundle. The species encapsulates all chemical species present in the
fuel bundle.
\begin{quote}\begin{description}
\item[{Returns}] \leavevmode
\sphinxstylestrong{species}

\item[{Return type}] \leavevmode
\sphinxhref{https://docs.python.org/3/library/functions.html\#object}{object}

\end{description}\end{quote}

\end{fulllineitems}


\end{fulllineitems}



\section{fuelsegmentsgroups}
\label{\detokenize{support_rst/fuelsegmentsgroups:module-fuelsegmentsgroups}}\label{\detokenize{support_rst/fuelsegmentsgroups:fuelsegmentsgroups}}\label{\detokenize{support_rst/fuelsegmentsgroups::doc}}\index{fuelsegmentsgroups (module)}
Author: Valmor de Almeida \sphinxhref{mailto:dealmeidav@ornl.gov}{dealmeidav@ornl.gov}; vfda

Fuel segment

VFdALib support classes

Sat Jun 27 14:46:49 EDT 2015
\index{FuelSegmentsGroups (class in fuelsegmentsgroups)}

\begin{fulllineitems}
\phantomsection\label{\detokenize{support_rst/fuelsegmentsgroups:fuelsegmentsgroups.FuelSegmentsGroups}}\pysiglinewithargsret{\sphinxbfcode{\sphinxupquote{class }}\sphinxcode{\sphinxupquote{fuelsegmentsgroups.}}\sphinxbfcode{\sphinxupquote{FuelSegmentsGroups}}}{\emph{key=None}, \emph{fuelSegments=None}}{}
Bases: \sphinxhref{https://docs.python.org/3/library/functions.html\#object}{\sphinxcode{\sphinxupquote{object}}}

Creates a dictionary of lists of fuel segment objects, with the keys
typically being timestamps. Each fuel segment object has two data members,
a \sphinxtitleref{Pandas} Series for geometry spec and a panda DataFrame for property
density.
\index{AddGroup() (fuelsegmentsgroups.FuelSegmentsGroups method)}

\begin{fulllineitems}
\phantomsection\label{\detokenize{support_rst/fuelsegmentsgroups:fuelsegmentsgroups.FuelSegmentsGroups.AddGroup}}\pysiglinewithargsret{\sphinxbfcode{\sphinxupquote{AddGroup}}}{\emph{key}, \emph{fuelSegments=None}}{}
Appends the dictionary with a new key and associated list of
fuelSegments. If the specified key is already present in the
dictionary, then the specified list of fuel segments will be appended
to the list of fuel segments already associated with the specified key.
\begin{quote}\begin{description}
\item[{Parameters}] \leavevmode\begin{itemize}
\item {} 
\sphinxstyleliteralstrong{\sphinxupquote{key}} (\sphinxhref{https://docs.python.org/3/library/stdtypes.html\#str}{\sphinxstyleliteralemphasis{\sphinxupquote{str}}}) \textendash{} 

\item {} 
\sphinxstyleliteralstrong{\sphinxupquote{fuelSegments}} (\sphinxhref{https://docs.python.org/3/library/stdtypes.html\#list}{\sphinxstyleliteralemphasis{\sphinxupquote{list}}}) \textendash{} 

\end{itemize}

\end{description}\end{quote}

\end{fulllineitems}

\index{GetAttribute() (fuelsegmentsgroups.FuelSegmentsGroups method)}

\begin{fulllineitems}
\phantomsection\label{\detokenize{support_rst/fuelsegmentsgroups:fuelsegmentsgroups.FuelSegmentsGroups.GetAttribute}}\pysiglinewithargsret{\sphinxbfcode{\sphinxupquote{GetAttribute}}}{\emph{groupKey=None}, \emph{attributeName=None}, \emph{nuclideSymbol=None}, \emph{nuclideSeries=None}}{}
Returns the average value of an attribute amongst all elements in a
group (WARNING: keys with no values associated with them will lower
this average!). If groupKey is not specified, the function will return
the average attribute value of every fuel segment element in the
entire dictionary. If attribute is not specified, the function call
will fail. If the key value specified does not match any keys in the
dictionary, the function will return a value of 0.
\begin{quote}\begin{description}
\item[{Parameters}] \leavevmode\begin{itemize}
\item {} 
\sphinxstyleliteralstrong{\sphinxupquote{groupKey}} (\sphinxhref{https://docs.python.org/3/library/stdtypes.html\#str}{\sphinxstyleliteralemphasis{\sphinxupquote{str}}}) \textendash{} 

\item {} 
\sphinxstyleliteralstrong{\sphinxupquote{attributeName}} (\sphinxhref{https://docs.python.org/3/library/stdtypes.html\#str}{\sphinxstyleliteralemphasis{\sphinxupquote{str}}}) \textendash{} 

\item {} 
\sphinxstyleliteralstrong{\sphinxupquote{nuclideSymbol}} (\sphinxhref{https://docs.python.org/3/library/stdtypes.html\#str}{\sphinxstyleliteralemphasis{\sphinxupquote{str}}}) \textendash{} 

\item {} 
\sphinxstyleliteralstrong{\sphinxupquote{nuclideSeries}} (\sphinxhref{https://docs.python.org/3/library/stdtypes.html\#str}{\sphinxstyleliteralemphasis{\sphinxupquote{str}}}) \textendash{} 

\end{itemize}

\item[{Returns}] \leavevmode
\sphinxstylestrong{groupAttribute}

\item[{Return type}] \leavevmode
\sphinxhref{https://docs.python.org/3/library/functions.html\#float}{float}

\end{description}\end{quote}

\end{fulllineitems}

\index{GetFuelSegments() (fuelsegmentsgroups.FuelSegmentsGroups method)}

\begin{fulllineitems}
\phantomsection\label{\detokenize{support_rst/fuelsegmentsgroups:fuelsegmentsgroups.FuelSegmentsGroups.GetFuelSegments}}\pysiglinewithargsret{\sphinxbfcode{\sphinxupquote{GetFuelSegments}}}{\emph{groupKey=None}}{}
Returns a list of fuel segments associated with a specified groupkey.
If no group key is specified, then all elements in the dictionary
will be returned. If the specified group key does not exist, then the
function will return an empty list.
\begin{quote}\begin{description}
\item[{Parameters}] \leavevmode
\sphinxstyleliteralstrong{\sphinxupquote{groupKey}} (\sphinxhref{https://docs.python.org/3/library/stdtypes.html\#str}{\sphinxstyleliteralemphasis{\sphinxupquote{str}}}) \textendash{} 

\item[{Returns}] \leavevmode
\sphinxstylestrong{fuelSegments}

\item[{Return type}] \leavevmode
\sphinxhref{https://docs.python.org/3/library/stdtypes.html\#list}{list}

\end{description}\end{quote}

\end{fulllineitems}

\index{HasGroup() (fuelsegmentsgroups.FuelSegmentsGroups method)}

\begin{fulllineitems}
\phantomsection\label{\detokenize{support_rst/fuelsegmentsgroups:fuelsegmentsgroups.FuelSegmentsGroups.HasGroup}}\pysiglinewithargsret{\sphinxbfcode{\sphinxupquote{HasGroup}}}{\emph{key}}{}
Checks if the specified key has a group of fuel segments associated
with it.
\begin{quote}\begin{description}
\item[{Parameters}] \leavevmode
\sphinxstyleliteralstrong{\sphinxupquote{key}} (\sphinxhref{https://docs.python.org/3/library/stdtypes.html\#str}{\sphinxstyleliteralemphasis{\sphinxupquote{str}}}) \textendash{} 

\item[{Returns}] \leavevmode
\sphinxstylestrong{key}

\item[{Return type}] \leavevmode
\sphinxhref{https://docs.python.org/3/library/stdtypes.html\#str}{str}

\end{description}\end{quote}

\end{fulllineitems}

\index{RemoveFuelSegment() (fuelsegmentsgroups.FuelSegmentsGroups method)}

\begin{fulllineitems}
\phantomsection\label{\detokenize{support_rst/fuelsegmentsgroups:fuelsegmentsgroups.FuelSegmentsGroups.RemoveFuelSegment}}\pysiglinewithargsret{\sphinxbfcode{\sphinxupquote{RemoveFuelSegment}}}{\emph{groupKey}, \emph{fuelSegment}}{}
Removes a fuel segment from a list associated with a specified group
key. If the specified group key or fuel segment do not exist, the
function will fail.
\begin{quote}\begin{description}
\item[{Parameters}] \leavevmode\begin{itemize}
\item {} 
\sphinxstyleliteralstrong{\sphinxupquote{groupKey}} (\sphinxhref{https://docs.python.org/3/library/stdtypes.html\#str}{\sphinxstyleliteralemphasis{\sphinxupquote{str}}}) \textendash{} 

\item {} 
\sphinxstyleliteralstrong{\sphinxupquote{fuelSegment}} (\sphinxhref{https://docs.python.org/3/library/stdtypes.html\#str}{\sphinxstyleliteralemphasis{\sphinxupquote{str}}}) \textendash{} 

\end{itemize}

\item[{Returns}] \leavevmode


\item[{Return type}] \leavevmode
empty

\end{description}\end{quote}

\end{fulllineitems}


\end{fulllineitems}



\section{fuelslug}
\label{\detokenize{support_rst/fuelslug:module-fuelslug}}\label{\detokenize{support_rst/fuelslug:fuelslug}}\label{\detokenize{support_rst/fuelslug::doc}}\index{fuelslug (module)}
Author: Valmor de Almeida \sphinxhref{mailto:dealmeidav@ornl.gov}{dealmeidav@ornl.gov}; vfda

Fuel slug


\subsection{ATTENTION:}
\label{\detokenize{support_rst/fuelslug:attention}}
This container requires two Phase() containers which are by definition histories.
The history is not checked. Therefore any inconsistency will be propagated forward.
A fuel slug has two solid phases: cladding and fuel. The user will decide how to
best use the underlying history data in the Phase() container of each phase.

VFdALib support classes

Thu Dec 15 16:18:39 EST 2016
\index{FuelSlug (class in fuelslug)}

\begin{fulllineitems}
\phantomsection\label{\detokenize{support_rst/fuelslug:fuelslug.FuelSlug}}\pysiglinewithargsret{\sphinxbfcode{\sphinxupquote{class }}\sphinxcode{\sphinxupquote{fuelslug.}}\sphinxbfcode{\sphinxupquote{FuelSlug}}}{\emph{specs=Series({[}{]}}, \emph{dtype: float64)}, \emph{fuelPhase= 	 **Phase()**:  	 time unit: s 	 *quantities*: None 	 *species*: None 	 *history* \#time\_stamp=1 	 *history end* @0.0 Series({[}{]}}, \emph{Name: 0.0}, \emph{dtype: float64)}, \emph{claddingPhase= 	 **Phase()**:  	 time unit: s 	 *quantities*: None 	 *species*: None 	 *history* \#time\_stamp=1 	 *history end* @0.0 Series({[}{]}}, \emph{Name: 0.0}, \emph{dtype: float64)}}{}
Bases: \sphinxhref{https://docs.python.org/3/library/functions.html\#object}{\sphinxcode{\sphinxupquote{object}}}
\index{GetAttribute() (fuelslug.FuelSlug method)}

\begin{fulllineitems}
\phantomsection\label{\detokenize{support_rst/fuelslug:fuelslug.FuelSlug.GetAttribute}}\pysiglinewithargsret{\sphinxbfcode{\sphinxupquote{GetAttribute}}}{\emph{name}, \emph{phase=None}, \emph{symbol=None}, \emph{series=None}}{}
Returns the value of the specified attribute. Any attribute that is
specified in class construction can be retrieved using this function.
The attribute may also be retrived from a speciefic phase, a specific
nuclide OR a specific series.
\begin{quote}\begin{description}
\item[{Parameters}] \leavevmode\begin{itemize}
\item {} 
\sphinxstyleliteralstrong{\sphinxupquote{name}} (\sphinxhref{https://docs.python.org/3/library/stdtypes.html\#str}{\sphinxstyleliteralemphasis{\sphinxupquote{str}}}) \textendash{} 

\item {} 
\sphinxstyleliteralstrong{\sphinxupquote{phase}} (\sphinxhref{https://docs.python.org/3/library/stdtypes.html\#str}{\sphinxstyleliteralemphasis{\sphinxupquote{str}}}) \textendash{} 

\item {} 
\sphinxstyleliteralstrong{\sphinxupquote{symbol}} (\sphinxhref{https://docs.python.org/3/library/stdtypes.html\#str}{\sphinxstyleliteralemphasis{\sphinxupquote{str}}}) \textendash{} 

\item {} 
\sphinxstyleliteralstrong{\sphinxupquote{series}} (\sphinxhref{https://docs.python.org/3/library/stdtypes.html\#str}{\sphinxstyleliteralemphasis{\sphinxupquote{str}}}) \textendash{} 

\end{itemize}

\item[{Returns}] \leavevmode
\sphinxstylestrong{attribute}

\item[{Return type}] \leavevmode
\sphinxhref{https://docs.python.org/3/library/functions.html\#int}{int} or \sphinxhref{https://docs.python.org/3/library/functions.html\#float}{float}

\end{description}\end{quote}

\end{fulllineitems}

\index{GetCladdingPhase() (fuelslug.FuelSlug method)}

\begin{fulllineitems}
\phantomsection\label{\detokenize{support_rst/fuelslug:fuelslug.FuelSlug.GetCladdingPhase}}\pysiglinewithargsret{\sphinxbfcode{\sphinxupquote{GetCladdingPhase}}}{}{}
Returns the phase history of the cladding.
\begin{quote}\begin{description}
\item[{Returns}] \leavevmode
\sphinxstylestrong{claddingPhase}

\item[{Return type}] \leavevmode
dataFrame

\end{description}\end{quote}

\end{fulllineitems}

\index{GetFuelPhase() (fuelslug.FuelSlug method)}

\begin{fulllineitems}
\phantomsection\label{\detokenize{support_rst/fuelslug:fuelslug.FuelSlug.GetFuelPhase}}\pysiglinewithargsret{\sphinxbfcode{\sphinxupquote{GetFuelPhase}}}{}{}
Returns the phase history of the solid fuel.
\begin{quote}\begin{description}
\item[{Returns}] \leavevmode
\sphinxstylestrong{fuelPhase}

\item[{Return type}] \leavevmode
dataFrame

\end{description}\end{quote}

\end{fulllineitems}

\index{GetSpecs() (fuelslug.FuelSlug method)}

\begin{fulllineitems}
\phantomsection\label{\detokenize{support_rst/fuelslug:fuelslug.FuelSlug.GetSpecs}}\pysiglinewithargsret{\sphinxbfcode{\sphinxupquote{GetSpecs}}}{}{}
Returns the species associated with this fuel slug.
\begin{quote}\begin{description}
\item[{Returns}] \leavevmode
\sphinxstylestrong{specs}

\item[{Return type}] \leavevmode
\sphinxhref{https://docs.python.org/3/library/stdtypes.html\#str}{str}

\end{description}\end{quote}

\end{fulllineitems}

\index{ReduceCladdingVolume() (fuelslug.FuelSlug method)}

\begin{fulllineitems}
\phantomsection\label{\detokenize{support_rst/fuelslug:fuelslug.FuelSlug.ReduceCladdingVolume}}\pysiglinewithargsret{\sphinxbfcode{\sphinxupquote{ReduceCladdingVolume}}}{\emph{dissolvedVolume}}{}
Reduces the amount of cladding in the slug by dissolvedvolume.
This will also update the dimensions of the cladding walls and
end caps; volume will be taken from all sections equally such that the
relative dimensions stay the same.
\begin{quote}\begin{description}
\item[{Parameters}] \leavevmode
\sphinxstyleliteralstrong{\sphinxupquote{dissolvedVolume}} (\sphinxhref{https://docs.python.org/3/library/functions.html\#float}{\sphinxstyleliteralemphasis{\sphinxupquote{float}}}) \textendash{} 

\end{description}\end{quote}

\end{fulllineitems}

\index{ReduceFuelVolume() (fuelslug.FuelSlug method)}

\begin{fulllineitems}
\phantomsection\label{\detokenize{support_rst/fuelslug:fuelslug.FuelSlug.ReduceFuelVolume}}\pysiglinewithargsret{\sphinxbfcode{\sphinxupquote{ReduceFuelVolume}}}{\emph{dissolvedVolume}}{}
Reduces the amount of fuel in the slug by dissolvedVolume. This
will also update the dimensions of the fuel slug, mainly the thickness
of each fuel layers.
\begin{quote}\begin{description}
\item[{Parameters}] \leavevmode
\sphinxstyleliteralstrong{\sphinxupquote{dissolvedVolume}} (\sphinxhref{https://docs.python.org/3/library/functions.html\#float}{\sphinxstyleliteralemphasis{\sphinxupquote{float}}}) \textendash{} 

\end{description}\end{quote}

\end{fulllineitems}

\index{claddingPhase (fuelslug.FuelSlug attribute)}

\begin{fulllineitems}
\phantomsection\label{\detokenize{support_rst/fuelslug:fuelslug.FuelSlug.claddingPhase}}\pysigline{\sphinxbfcode{\sphinxupquote{claddingPhase}}}
Returns the phase history of the cladding.
\begin{quote}\begin{description}
\item[{Returns}] \leavevmode
\sphinxstylestrong{claddingPhase}

\item[{Return type}] \leavevmode
dataFrame

\end{description}\end{quote}

\end{fulllineitems}

\index{fuelPhase (fuelslug.FuelSlug attribute)}

\begin{fulllineitems}
\phantomsection\label{\detokenize{support_rst/fuelslug:fuelslug.FuelSlug.fuelPhase}}\pysigline{\sphinxbfcode{\sphinxupquote{fuelPhase}}}
Returns the phase history of the solid fuel.
\begin{quote}\begin{description}
\item[{Returns}] \leavevmode
\sphinxstylestrong{fuelPhase}

\item[{Return type}] \leavevmode
dataFrame

\end{description}\end{quote}

\end{fulllineitems}

\index{specs (fuelslug.FuelSlug attribute)}

\begin{fulllineitems}
\phantomsection\label{\detokenize{support_rst/fuelslug:fuelslug.FuelSlug.specs}}\pysigline{\sphinxbfcode{\sphinxupquote{specs}}}
Returns the species associated with this fuel slug.
\begin{quote}\begin{description}
\item[{Returns}] \leavevmode
\sphinxstylestrong{specs}

\item[{Return type}] \leavevmode
\sphinxhref{https://docs.python.org/3/library/stdtypes.html\#str}{str}

\end{description}\end{quote}

\end{fulllineitems}


\end{fulllineitems}



\section{nuclides}
\label{\detokenize{support_rst/nuclides:module-nuclides}}\label{\detokenize{support_rst/nuclides:nuclides}}\label{\detokenize{support_rst/nuclides::doc}}\index{nuclides (module)}
Author: Valmor de Almeida \sphinxhref{mailto:dealmeidav@ornl.gov}{dealmeidav@ornl.gov}; vfda

Nuclides container.
The purpose of the this container is to store and query a table of nuclides.
Typically the table is filled in with data from an ORIGEN calculation or
some other fission/transmutation code.

VFdALib support classes

Sat Jun 27 14:46:49 EDT 2015
\index{Nuclides (class in nuclides)}

\begin{fulllineitems}
\phantomsection\label{\detokenize{support_rst/nuclides:nuclides.Nuclides}}\pysiglinewithargsret{\sphinxbfcode{\sphinxupquote{class }}\sphinxcode{\sphinxupquote{nuclides.}}\sphinxbfcode{\sphinxupquote{Nuclides}}}{\emph{propertyDensities=Empty DataFrame Columns: {[}{]} Index: {[}{]}}}{}
Bases: \sphinxhref{https://docs.python.org/3/library/functions.html\#object}{\sphinxcode{\sphinxupquote{object}}}
\index{GetAttribute() (nuclides.Nuclides method)}

\begin{fulllineitems}
\phantomsection\label{\detokenize{support_rst/nuclides:nuclides.Nuclides.GetAttribute}}\pysiglinewithargsret{\sphinxbfcode{\sphinxupquote{GetAttribute}}}{\emph{name}, \emph{symbol=None}, \emph{series=None}}{}
\end{fulllineitems}


\end{fulllineitems}



\section{periodictable}
\label{\detokenize{support_rst/periodictable:module-periodictable}}\label{\detokenize{support_rst/periodictable:periodictable}}\label{\detokenize{support_rst/periodictable::doc}}\index{periodictable (module)}
Properties of the chemical elements.

Each chemical element is represented as an object instance. Physicochemical
and descriptive properties of the elements are stored as instance attributes.
\begin{quote}\begin{description}
\item[{Author}] \leavevmode
\sphinxhref{http://www.lfd.uci.edu/~gohlke/}{Christoph Gohlke}

\item[{Version}] \leavevmode
2015.01.29

\end{description}\end{quote}

Radiochemical data (isotopes) has been added to this table (2015-2016)
Origin: \sphinxurl{http://www.radiochemistry.org/}
Valmor F. de Almeida: \sphinxhref{mailto:dealmeidavf@gmail.com}{dealmeidavf@gmail.com}; \sphinxhref{mailto:dealmeidav@ornl.gov}{dealmeidav@ornl.gov}


\subsection{Requirements}
\label{\detokenize{support_rst/periodictable:requirements}}\begin{itemize}
\item {} 
\sphinxhref{http://www.python.org}{CPython 2.7 or 3.4}

\end{itemize}
\paragraph{References}
\begin{enumerate}
\item {} 
\sphinxurl{http://physics.nist.gov/PhysRefData/Compositions/}

\item {} 
\sphinxurl{http://physics.nist.gov/PhysRefData/IonEnergy/tblNew.html}

\item {} 
\sphinxurl{http://en.wikipedia.org/wiki/\%(element.name)s}

\item {} 
\sphinxurl{http://www.miranda.org/~jkominek/elements/elements.db}

\end{enumerate}
\paragraph{Examples}

\fvset{hllines={, ,}}%
\begin{sphinxVerbatim}[commandchars=\\\{\}]
\PYG{g+gp}{\PYGZgt{}\PYGZgt{}\PYGZgt{} }\PYG{k+kn}{from} \PYG{n+nn}{elements} \PYG{k}{import} \PYG{n}{ELEMENTS}
\PYG{g+gp}{\PYGZgt{}\PYGZgt{}\PYGZgt{} }\PYG{n+nb}{len}\PYG{p}{(}\PYG{n}{ELEMENTS}\PYG{p}{)}
\PYG{g+go}{109}
\PYG{g+gp}{\PYGZgt{}\PYGZgt{}\PYGZgt{} }\PYG{n+nb}{str}\PYG{p}{(}\PYG{n}{ELEMENTS}\PYG{p}{[}\PYG{l+m+mi}{109}\PYG{p}{]}\PYG{p}{)}
\PYG{g+go}{\PYGZsq{}Meitnerium\PYGZsq{}}
\PYG{g+gp}{\PYGZgt{}\PYGZgt{}\PYGZgt{} }\PYG{n}{ele} \PYG{o}{=} \PYG{n}{ELEMENTS}\PYG{p}{[}\PYG{l+s+s1}{\PYGZsq{}}\PYG{l+s+s1}{C}\PYG{l+s+s1}{\PYGZsq{}}\PYG{p}{]}
\PYG{g+gp}{\PYGZgt{}\PYGZgt{}\PYGZgt{} }\PYG{n}{ele}\PYG{o}{.}\PYG{n}{number}\PYG{p}{,} \PYG{n}{ele}\PYG{o}{.}\PYG{n}{symbol}\PYG{p}{,} \PYG{n}{ele}\PYG{o}{.}\PYG{n}{name}\PYG{p}{,} \PYG{n}{ele}\PYG{o}{.}\PYG{n}{eleconfig}
\PYG{g+go}{(6, \PYGZsq{}C\PYGZsq{}, \PYGZsq{}Carbon\PYGZsq{}, \PYGZsq{}[He] 2s2 2p2\PYGZsq{})}
\PYG{g+gp}{\PYGZgt{}\PYGZgt{}\PYGZgt{} }\PYG{n}{ele}\PYG{o}{.}\PYG{n}{eleconfig\PYGZus{}dict}
\PYG{g+go}{\PYGZob{}(1, \PYGZsq{}s\PYGZsq{}): 2, (2, \PYGZsq{}p\PYGZsq{}): 2, (2, \PYGZsq{}s\PYGZsq{}): 2\PYGZcb{}}
\PYG{g+gp}{\PYGZgt{}\PYGZgt{}\PYGZgt{} }\PYG{n+nb}{sum}\PYG{p}{(}\PYG{n}{ele}\PYG{o}{.}\PYG{n}{mass} \PYG{k}{for} \PYG{n}{ele} \PYG{o+ow}{in} \PYG{n}{ELEMENTS}\PYG{p}{)}
\PYG{g+go}{14659.1115599}
\PYG{g+gp}{\PYGZgt{}\PYGZgt{}\PYGZgt{} }\PYG{k}{for} \PYG{n}{ele} \PYG{o+ow}{in} \PYG{n}{ELEMENTS}\PYG{p}{:}
\PYG{g+gp}{... }    \PYG{n}{ele}\PYG{o}{.}\PYG{n}{validate}\PYG{p}{(}\PYG{p}{)}
\PYG{g+gp}{... }    \PYG{n}{ele} \PYG{o}{=} \PYG{n+nb}{eval}\PYG{p}{(}\PYG{n+nb}{repr}\PYG{p}{(}\PYG{n}{ele}\PYG{p}{)}\PYG{p}{)}
\end{sphinxVerbatim}


\section{phase}
\label{\detokenize{support_rst/phase:module-phase}}\label{\detokenize{support_rst/phase:phase}}\label{\detokenize{support_rst/phase::doc}}\index{phase (module)}
Phase \sphinxstyleemphasis{history} container. When you think of a phase value, think of that value at
a specific point in time. This container holds the historic data of a phase;
its species and quantities. This implementation treats access of time stamps within
a tolerance. All searches for time stamped values are subjected to an approximation
of the time stamp to avoid storing values too close to each other in time, and/or to
return the closest value in time searched or no value if none can be found according
to the tolerance.


\subsection{Background}
\label{\detokenize{support_rst/phase:background}}
TODO: ATTENTION:
The species (list of Specie) AND quantities (list of Quantity) data members
have ARBITRARY density values either at an arbitrary point in the history or at
no point in the history. This needs to be removed in the future to avoid confusion.

To obtain history values, associated to the phase, at a particular point in time,
use the GetValue() method to access the history data frame (pandas) via columns and
rows. ALERT: The corresponding values in species and quantities are OVERRIDEN and NOT to
be used through the phase interface.

Author: Valmor F. de Almeida \sphinxhref{mailto:dealmeidav@ornl.gov}{dealmeidav@ornl.gov}; vfda
Sat Sep  5 01:26:53 EDT 2015

Cortix: a program for system-level modules coupling, execution, and analysis.
\index{Phase (class in phase)}

\begin{fulllineitems}
\phantomsection\label{\detokenize{support_rst/phase:phase.Phase}}\pysiglinewithargsret{\sphinxbfcode{\sphinxupquote{class }}\sphinxcode{\sphinxupquote{phase.}}\sphinxbfcode{\sphinxupquote{Phase}}}{\emph{time\_stamp=None}, \emph{time\_unit=None}, \emph{species=None}, \emph{quantities=None}}{}
Bases: \sphinxhref{https://docs.python.org/3/library/functions.html\#object}{\sphinxcode{\sphinxupquote{object}}}

Phase \sphinxtitleref{history} container. A \sphinxtitleref{Phase} consists of \sphinxtitleref{Species} and \sphinxtitleref{Quantities}
varying with time. This container is meant to reproduce the basic idea of a
material phase.
\index{AddQuantity() (phase.Phase method)}

\begin{fulllineitems}
\phantomsection\label{\detokenize{support_rst/phase:phase.Phase.AddQuantity}}\pysiglinewithargsret{\sphinxbfcode{\sphinxupquote{AddQuantity}}}{\emph{newQuant}}{}
Adds a new quantity object to the dataframe. See quantity.py for more
details on the quantity class.
\begin{quote}\begin{description}
\item[{Parameters}] \leavevmode
\sphinxstyleliteralstrong{\sphinxupquote{newQuant}} (\sphinxhref{https://docs.python.org/3/library/functions.html\#object}{\sphinxstyleliteralemphasis{\sphinxupquote{object}}}) \textendash{} 

\end{description}\end{quote}

\end{fulllineitems}

\index{AddRow() (phase.Phase method)}

\begin{fulllineitems}
\phantomsection\label{\detokenize{support_rst/phase:phase.Phase.AddRow}}\pysiglinewithargsret{\sphinxbfcode{\sphinxupquote{AddRow}}}{\emph{try\_time\_stamp}, \emph{row\_values}}{}
Adds a row to the dataframe, with a timestamp of try\_time\_stamp and
row values equal to row\_values. Take care that the dimensions and order
of the data matches up!
\begin{quote}\begin{description}
\item[{Parameters}] \leavevmode\begin{itemize}
\item {} 
\sphinxstyleliteralstrong{\sphinxupquote{try\_time\_stamp}} (\sphinxhref{https://docs.python.org/3/library/functions.html\#float}{\sphinxstyleliteralemphasis{\sphinxupquote{float}}}) \textendash{} 

\item {} 
\sphinxstyleliteralstrong{\sphinxupquote{row\_values}} (\sphinxhref{https://docs.python.org/3/library/stdtypes.html\#list}{\sphinxstyleliteralemphasis{\sphinxupquote{list}}}) \textendash{} 

\end{itemize}

\end{description}\end{quote}

\end{fulllineitems}

\index{AddSpecie() (phase.Phase method)}

\begin{fulllineitems}
\phantomsection\label{\detokenize{support_rst/phase:phase.Phase.AddSpecie}}\pysiglinewithargsret{\sphinxbfcode{\sphinxupquote{AddSpecie}}}{\emph{new\_specie}}{}
Adds a new specie object to the phase history. See species.py for
more details on the specie class.
\begin{quote}\begin{description}
\item[{Parameters}] \leavevmode
\sphinxstyleliteralstrong{\sphinxupquote{new\_specie}} (\sphinxstyleliteralemphasis{\sphinxupquote{obj}}) \textendash{} 

\end{description}\end{quote}

\end{fulllineitems}

\index{ClearHistory() (phase.Phase method)}

\begin{fulllineitems}
\phantomsection\label{\detokenize{support_rst/phase:phase.Phase.ClearHistory}}\pysiglinewithargsret{\sphinxbfcode{\sphinxupquote{ClearHistory}}}{\emph{value=0.0}}{}
Set species and quantities of history to a given value
(default to zero value), all time stamps are preserved.
\begin{quote}\begin{description}
\item[{Parameters}] \leavevmode
\sphinxstyleliteralstrong{\sphinxupquote{value}} (\sphinxhref{https://docs.python.org/3/library/functions.html\#float}{\sphinxstyleliteralemphasis{\sphinxupquote{float}}}) \textendash{} 

\end{description}\end{quote}

\end{fulllineitems}

\index{GetActors() (phase.Phase method)}

\begin{fulllineitems}
\phantomsection\label{\detokenize{support_rst/phase:phase.Phase.GetActors}}\pysiglinewithargsret{\sphinxbfcode{\sphinxupquote{GetActors}}}{}{}
Returns a list of all the actors in the phase history.
\begin{quote}\begin{description}
\item[{Returns}] \leavevmode
\sphinxstylestrong{list(self.\_\_phase.colums)}

\item[{Return type}] \leavevmode
\sphinxhref{https://docs.python.org/3/library/stdtypes.html\#list}{list}

\end{description}\end{quote}

\end{fulllineitems}

\index{GetColumn() (phase.Phase method)}

\begin{fulllineitems}
\phantomsection\label{\detokenize{support_rst/phase:phase.Phase.GetColumn}}\pysiglinewithargsret{\sphinxbfcode{\sphinxupquote{GetColumn}}}{\emph{actor}}{}
Returns an entire column of data. A column is the entire history
of data associated with a specific actor.
\begin{quote}\begin{description}
\item[{Parameters}] \leavevmode
\sphinxstyleliteralstrong{\sphinxupquote{actor}} (\sphinxhref{https://docs.python.org/3/library/stdtypes.html\#str}{\sphinxstyleliteralemphasis{\sphinxupquote{str}}}) \textendash{} 

\item[{Returns}] \leavevmode
\sphinxstylestrong{list(self.\_\_phase.loc{[}}

\item[{Return type}] \leavevmode
, actor{]}): list

\end{description}\end{quote}

\end{fulllineitems}

\index{GetQuantities() (phase.Phase method)}

\begin{fulllineitems}
\phantomsection\label{\detokenize{support_rst/phase:phase.Phase.GetQuantities}}\pysiglinewithargsret{\sphinxbfcode{\sphinxupquote{GetQuantities}}}{}{}
Returns the list of \sphinxtitleref{Quantities}. The values in each \sphinxtitleref{Quantity} are
synchronized with the \sphinxtitleref{Phase} data frame.
\begin{quote}\begin{description}
\item[{Returns}] \leavevmode
\sphinxstylestrong{quantities}

\item[{Return type}] \leavevmode
\sphinxhref{https://docs.python.org/3/library/stdtypes.html\#list}{list}

\end{description}\end{quote}

\end{fulllineitems}

\index{GetQuantity() (phase.Phase method)}

\begin{fulllineitems}
\phantomsection\label{\detokenize{support_rst/phase:phase.Phase.GetQuantity}}\pysiglinewithargsret{\sphinxbfcode{\sphinxupquote{GetQuantity}}}{\emph{name}}{}
Returns the quantity evaluated at the last time step of the phase
history. This also updates the value of the quantity object. If the
quantity name does not exist the return is None.
\begin{quote}\begin{description}
\item[{Parameters}] \leavevmode
\sphinxstyleliteralstrong{\sphinxupquote{name}} (\sphinxhref{https://docs.python.org/3/library/stdtypes.html\#str}{\sphinxstyleliteralemphasis{\sphinxupquote{str}}}) \textendash{} 

\end{description}\end{quote}

\end{fulllineitems}

\index{GetRow() (phase.Phase method)}

\begin{fulllineitems}
\phantomsection\label{\detokenize{support_rst/phase:phase.Phase.GetRow}}\pysiglinewithargsret{\sphinxbfcode{\sphinxupquote{GetRow}}}{\emph{try\_time\_stamp=None}}{}
Returns an entire row of the phase dataframe. A row is a series of
values that are all at the same time stamp.
\begin{quote}\begin{description}
\item[{Parameters}] \leavevmode
\sphinxstyleliteralstrong{\sphinxupquote{try\_time\_stamp}} (\sphinxhref{https://docs.python.org/3/library/functions.html\#float}{\sphinxstyleliteralemphasis{\sphinxupquote{float}}}) \textendash{} 

\item[{Returns}] \leavevmode
\sphinxstylestrong{list(self.\_\_phase.loc{[}time\_stamp,}

\item[{Return type}] \leavevmode
{]}): list

\end{description}\end{quote}

\end{fulllineitems}

\index{GetSpecie() (phase.Phase method)}

\begin{fulllineitems}
\phantomsection\label{\detokenize{support_rst/phase:phase.Phase.GetSpecie}}\pysiglinewithargsret{\sphinxbfcode{\sphinxupquote{GetSpecie}}}{\emph{name}}{}
Returns the species specified by name if it exists, or none if it
doesn’t.
\begin{quote}\begin{description}
\item[{Parameters}] \leavevmode
\sphinxstyleliteralstrong{\sphinxupquote{name}} (\sphinxhref{https://docs.python.org/3/library/stdtypes.html\#str}{\sphinxstyleliteralemphasis{\sphinxupquote{str}}}) \textendash{} 

\item[{Returns}] \leavevmode
\sphinxstylestrong{specie}

\item[{Return type}] \leavevmode
\sphinxhref{https://docs.python.org/3/library/stdtypes.html\#str}{str}

\end{description}\end{quote}

\end{fulllineitems}

\index{GetSpecies() (phase.Phase method)}

\begin{fulllineitems}
\phantomsection\label{\detokenize{support_rst/phase:phase.Phase.GetSpecies}}\pysiglinewithargsret{\sphinxbfcode{\sphinxupquote{GetSpecies}}}{}{}
Returns every single species in the phase history.
\begin{quote}\begin{description}
\item[{Returns}] \leavevmode
\sphinxstylestrong{species}

\item[{Return type}] \leavevmode
\sphinxhref{https://docs.python.org/3/library/stdtypes.html\#list}{list}

\end{description}\end{quote}

\end{fulllineitems}

\index{GetTimeStamps() (phase.Phase method)}

\begin{fulllineitems}
\phantomsection\label{\detokenize{support_rst/phase:phase.Phase.GetTimeStamps}}\pysiglinewithargsret{\sphinxbfcode{\sphinxupquote{GetTimeStamps}}}{}{}
Returns a list of all the time stamps in the phase history.
\begin{quote}\begin{description}
\item[{Returns}] \leavevmode
\sphinxstylestrong{timeStamps}

\item[{Return type}] \leavevmode
\sphinxhref{https://docs.python.org/3/library/stdtypes.html\#list}{list}

\end{description}\end{quote}

\end{fulllineitems}

\index{GetValue() (phase.Phase method)}

\begin{fulllineitems}
\phantomsection\label{\detokenize{support_rst/phase:phase.Phase.GetValue}}\pysiglinewithargsret{\sphinxbfcode{\sphinxupquote{GetValue}}}{\emph{actor}, \emph{try\_time\_stamp=None}}{}
Deprecated: use get\_value()

\end{fulllineitems}

\index{ResetHistory() (phase.Phase method)}

\begin{fulllineitems}
\phantomsection\label{\detokenize{support_rst/phase:phase.Phase.ResetHistory}}\pysiglinewithargsret{\sphinxbfcode{\sphinxupquote{ResetHistory}}}{\emph{try\_time\_stamp=None}, \emph{value=None}}{}
Set species and quantities of history to a given value
(default to zero value) only one time stamp is preserved (default to
last time stamp).
\begin{quote}\begin{description}
\item[{Parameters}] \leavevmode\begin{itemize}
\item {} 
\sphinxstyleliteralstrong{\sphinxupquote{try\_time\_stamp}} (\sphinxhref{https://docs.python.org/3/library/functions.html\#float}{\sphinxstyleliteralemphasis{\sphinxupquote{float}}}) \textendash{} 

\item {} 
\sphinxstyleliteralstrong{\sphinxupquote{value}} (\sphinxhref{https://docs.python.org/3/library/functions.html\#float}{\sphinxstyleliteralemphasis{\sphinxupquote{float}}}) \textendash{} 

\end{itemize}

\end{description}\end{quote}

\end{fulllineitems}

\index{ScaleRow() (phase.Phase method)}

\begin{fulllineitems}
\phantomsection\label{\detokenize{support_rst/phase:phase.Phase.ScaleRow}}\pysiglinewithargsret{\sphinxbfcode{\sphinxupquote{ScaleRow}}}{\emph{try\_time\_stamp}, \emph{value}}{}
Multiplies all of the data in a row (except time stamp) by a scalar
value.
\begin{quote}\begin{description}
\item[{Parameters}] \leavevmode\begin{itemize}
\item {} 
\sphinxstyleliteralstrong{\sphinxupquote{try\_time\_stamp}} (\sphinxhref{https://docs.python.org/3/library/functions.html\#float}{\sphinxstyleliteralemphasis{\sphinxupquote{float}}}) \textendash{} 

\item {} 
\sphinxstyleliteralstrong{\sphinxupquote{value}} (\sphinxhref{https://docs.python.org/3/library/functions.html\#float}{\sphinxstyleliteralemphasis{\sphinxupquote{float}}}) \textendash{} 

\end{itemize}

\end{description}\end{quote}

\end{fulllineitems}

\index{SetSpecieId() (phase.Phase method)}

\begin{fulllineitems}
\phantomsection\label{\detokenize{support_rst/phase:phase.Phase.SetSpecieId}}\pysiglinewithargsret{\sphinxbfcode{\sphinxupquote{SetSpecieId}}}{\emph{name}, \emph{val}}{}
Sets the flag of a specie “name” equal to val.
\begin{quote}\begin{description}
\item[{Parameters}] \leavevmode\begin{itemize}
\item {} 
\sphinxstyleliteralstrong{\sphinxupquote{name}} (\sphinxhref{https://docs.python.org/3/library/stdtypes.html\#str}{\sphinxstyleliteralemphasis{\sphinxupquote{str}}}) \textendash{} 

\item {} 
\sphinxstyleliteralstrong{\sphinxupquote{val}} (\sphinxhref{https://docs.python.org/3/library/functions.html\#int}{\sphinxstyleliteralemphasis{\sphinxupquote{int}}}) \textendash{} 

\end{itemize}

\end{description}\end{quote}

\end{fulllineitems}

\index{SetValue() (phase.Phase method)}

\begin{fulllineitems}
\phantomsection\label{\detokenize{support_rst/phase:phase.Phase.SetValue}}\pysiglinewithargsret{\sphinxbfcode{\sphinxupquote{SetValue}}}{\emph{actor}, \emph{value}, \emph{try\_time\_stamp=None}}{}
For the record: old def SetValue(self, time\_stamp, actor, value):
\begin{quote}\begin{description}
\item[{Parameters}] \leavevmode\begin{itemize}
\item {} 
\sphinxstyleliteralstrong{\sphinxupquote{actor}} (\sphinxhref{https://docs.python.org/3/library/stdtypes.html\#str}{\sphinxstyleliteralemphasis{\sphinxupquote{str}}}) \textendash{} 

\item {} 
\sphinxstyleliteralstrong{\sphinxupquote{value}} (\sphinxhref{https://docs.python.org/3/library/functions.html\#float}{\sphinxstyleliteralemphasis{\sphinxupquote{float}}}) \textendash{} 

\item {} 
\sphinxstyleliteralstrong{\sphinxupquote{try\_time\_stamp}} (\sphinxhref{https://docs.python.org/3/library/functions.html\#float}{\sphinxstyleliteralemphasis{\sphinxupquote{float}}}) \textendash{} 

\end{itemize}

\end{description}\end{quote}

\end{fulllineitems}

\index{WriteHTML() (phase.Phase method)}

\begin{fulllineitems}
\phantomsection\label{\detokenize{support_rst/phase:phase.Phase.WriteHTML}}\pysiglinewithargsret{\sphinxbfcode{\sphinxupquote{WriteHTML}}}{\emph{fileName}}{}
Convert the \sphinxtitleref{Phase} container into an HTML file.
\begin{quote}\begin{description}
\item[{Parameters}] \leavevmode
\sphinxstyleliteralstrong{\sphinxupquote{fileName}} (\sphinxhref{https://docs.python.org/3/library/stdtypes.html\#str}{\sphinxstyleliteralemphasis{\sphinxupquote{str}}}) \textendash{} 

\end{description}\end{quote}

\end{fulllineitems}

\index{\_\_init\_\_() (phase.Phase method)}

\begin{fulllineitems}
\phantomsection\label{\detokenize{support_rst/phase:phase.Phase.__init__}}\pysiglinewithargsret{\sphinxbfcode{\sphinxupquote{\_\_init\_\_}}}{\emph{time\_stamp=None}, \emph{time\_unit=None}, \emph{species=None}, \emph{quantities=None}}{}
Sometimes an empty Phase object is created by user code. This case needs
adequate logic for None types.
Note on usage: when passing quantities, do set the value argument explicitly
to help define the type and avoid SetValue() errors with Pandas. This is
to be investigated later. Also, the usage of a DataFrame needs to be re-evaluated.
Maybe better to use a Quantity object and a Specie object with a Pandas Series
history as a value to avoid the existance of a value in Quantity and a value
in Phase that are not in sync.

\end{fulllineitems}

\index{get\_quantity() (phase.Phase method)}

\begin{fulllineitems}
\phantomsection\label{\detokenize{support_rst/phase:phase.Phase.get_quantity}}\pysiglinewithargsret{\sphinxbfcode{\sphinxupquote{get\_quantity}}}{\emph{name}, \emph{try\_time\_stamp=None}}{}
New version.
Get the quantity \sphinxtitleref{name} at a point in time closest to
\sphinxtitleref{try\_time\_stamp} up to a tolerance. If no time stamp is passed, the
whole history is returned.
\begin{quote}\begin{description}
\item[{Parameters}] \leavevmode\begin{itemize}
\item {} 
\sphinxstyleliteralstrong{\sphinxupquote{name}} (\sphinxhref{https://docs.python.org/3/library/stdtypes.html\#str}{\sphinxstyleliteralemphasis{\sphinxupquote{str}}}) \textendash{} 

\item {} 
\sphinxstyleliteralstrong{\sphinxupquote{try\_time\_stamp}} (\sphinxhref{https://docs.python.org/3/library/functions.html\#float}{\sphinxstyleliteralemphasis{\sphinxupquote{float}}}\sphinxstyleliteralemphasis{\sphinxupquote{, }}\sphinxhref{https://docs.python.org/3/library/functions.html\#int}{\sphinxstyleliteralemphasis{\sphinxupquote{int}}}\sphinxstyleliteralemphasis{\sphinxupquote{ or }}\sphinxhref{https://docs.python.org/3/library/constants.html\#None}{\sphinxstyleliteralemphasis{\sphinxupquote{None}}}) \textendash{} Time stamp of desired quantity value. Default: None returns the
whole quantity history.

\end{itemize}

\item[{Returns}] \leavevmode
\sphinxstylestrong{quant.value}

\item[{Return type}] \leavevmode
\sphinxhref{https://docs.python.org/3/library/functions.html\#float}{float} or \sphinxhref{https://docs.python.org/3/library/functions.html\#int}{int} or other

\end{description}\end{quote}

\end{fulllineitems}

\index{get\_quantity\_history() (phase.Phase method)}

\begin{fulllineitems}
\phantomsection\label{\detokenize{support_rst/phase:phase.Phase.get_quantity_history}}\pysiglinewithargsret{\sphinxbfcode{\sphinxupquote{get\_quantity\_history}}}{\emph{name}}{}
Create a Quantity \sphinxtitleref{name} history. This will create a fully qualified
Quantity object and return to the caller. The function is typically
needed for data output to a file through \sphinxtitleref{pickle}. Since the value
attribute of a quantity can be any data structure, a time-series is
built on the fly and stored in the value attribute. In addition the
time unit is added to the final return value as a tuple.
\begin{quote}\begin{description}
\item[{Parameters}] \leavevmode
\sphinxstyleliteralstrong{\sphinxupquote{name}} (\sphinxhref{https://docs.python.org/3/library/stdtypes.html\#str}{\sphinxstyleliteralemphasis{\sphinxupquote{str}}}) \textendash{} 

\item[{Returns}] \leavevmode
\sphinxstylestrong{quant\_history}

\item[{Return type}] \leavevmode
\sphinxhref{https://docs.python.org/3/library/stdtypes.html\#tuple}{tuple}({\hyperref[\detokenize{support_rst/quantity:quantity.Quantity}]{\sphinxcrossref{Quantity}}},\sphinxhref{https://docs.python.org/3/library/stdtypes.html\#str}{str})

\end{description}\end{quote}

\end{fulllineitems}

\index{get\_value() (phase.Phase method)}

\begin{fulllineitems}
\phantomsection\label{\detokenize{support_rst/phase:phase.Phase.get_value}}\pysiglinewithargsret{\sphinxbfcode{\sphinxupquote{get\_value}}}{\emph{actor}, \emph{try\_time\_stamp=None}}{}
Returns the value associated with a specified actor at a specified
time stamp.
\begin{quote}\begin{description}
\item[{Parameters}] \leavevmode\begin{itemize}
\item {} 
\sphinxstyleliteralstrong{\sphinxupquote{actor}} (\sphinxhref{https://docs.python.org/3/library/stdtypes.html\#str}{\sphinxstyleliteralemphasis{\sphinxupquote{str}}}) \textendash{} 

\item {} 
\sphinxstyleliteralstrong{\sphinxupquote{try\_time\_stamp}} (\sphinxhref{https://docs.python.org/3/library/functions.html\#float}{\sphinxstyleliteralemphasis{\sphinxupquote{float}}}) \textendash{} 

\end{itemize}

\item[{Returns}] \leavevmode
\sphinxstylestrong{self.\_\_phase.loc{[}time\_stamp, actor{]}}

\item[{Return type}] \leavevmode
\sphinxhref{https://docs.python.org/3/library/functions.html\#float}{float}

\end{description}\end{quote}

\end{fulllineitems}

\index{has\_time\_stamp() (phase.Phase method)}

\begin{fulllineitems}
\phantomsection\label{\detokenize{support_rst/phase:phase.Phase.has_time_stamp}}\pysiglinewithargsret{\sphinxbfcode{\sphinxupquote{has\_time\_stamp}}}{\emph{try\_time\_stamp}}{}
Checks to see if try\_time\_stamp exists in the phase history.
\begin{quote}\begin{description}
\item[{Parameters}] \leavevmode
\sphinxstyleliteralstrong{\sphinxupquote{try\_time\_stamp}} \textendash{} 

\end{description}\end{quote}

\end{fulllineitems}

\index{quantities (phase.Phase attribute)}

\begin{fulllineitems}
\phantomsection\label{\detokenize{support_rst/phase:phase.Phase.quantities}}\pysigline{\sphinxbfcode{\sphinxupquote{quantities}}}
Returns the list of \sphinxtitleref{Quantities}. The values in each \sphinxtitleref{Quantity} are
synchronized with the \sphinxtitleref{Phase} data frame.
\begin{quote}\begin{description}
\item[{Returns}] \leavevmode
\sphinxstylestrong{quantities}

\item[{Return type}] \leavevmode
\sphinxhref{https://docs.python.org/3/library/stdtypes.html\#list}{list}

\end{description}\end{quote}

\end{fulllineitems}

\index{set\_value() (phase.Phase method)}

\begin{fulllineitems}
\phantomsection\label{\detokenize{support_rst/phase:phase.Phase.set_value}}\pysiglinewithargsret{\sphinxbfcode{\sphinxupquote{set\_value}}}{\emph{actor}, \emph{value}, \emph{try\_time\_stamp=None}}{}
New version. Discontinue using SetValue()

\end{fulllineitems}

\index{species (phase.Phase attribute)}

\begin{fulllineitems}
\phantomsection\label{\detokenize{support_rst/phase:phase.Phase.species}}\pysigline{\sphinxbfcode{\sphinxupquote{species}}}
Returns every single species in the phase history.
\begin{quote}\begin{description}
\item[{Returns}] \leavevmode
\sphinxstylestrong{species}

\item[{Return type}] \leavevmode
\sphinxhref{https://docs.python.org/3/library/stdtypes.html\#list}{list}

\end{description}\end{quote}

\end{fulllineitems}

\index{timeStamps (phase.Phase attribute)}

\begin{fulllineitems}
\phantomsection\label{\detokenize{support_rst/phase:phase.Phase.timeStamps}}\pysigline{\sphinxbfcode{\sphinxupquote{timeStamps}}}
Returns a list of all the time stamps in the phase history.
\begin{quote}\begin{description}
\item[{Returns}] \leavevmode
\sphinxstylestrong{timeStamps}

\item[{Return type}] \leavevmode
\sphinxhref{https://docs.python.org/3/library/stdtypes.html\#list}{list}

\end{description}\end{quote}

\end{fulllineitems}

\index{time\_stamps (phase.Phase attribute)}

\begin{fulllineitems}
\phantomsection\label{\detokenize{support_rst/phase:phase.Phase.time_stamps}}\pysigline{\sphinxbfcode{\sphinxupquote{time\_stamps}}}
Get all time stamps in the index of the data frame.
\begin{quote}\begin{description}
\item[{Returns}] \leavevmode
\sphinxstylestrong{time\_stamps}

\item[{Return type}] \leavevmode
\sphinxhref{https://docs.python.org/3/library/stdtypes.html\#list}{list}

\end{description}\end{quote}

\end{fulllineitems}

\index{time\_unit (phase.Phase attribute)}

\begin{fulllineitems}
\phantomsection\label{\detokenize{support_rst/phase:phase.Phase.time_unit}}\pysigline{\sphinxbfcode{\sphinxupquote{time\_unit}}}
Returns the time unit of the \sphinxtitleref{Phase.}
\begin{quote}\begin{description}
\item[{Returns}] \leavevmode
\sphinxstylestrong{time\_unit}

\item[{Return type}] \leavevmode
\sphinxhref{https://docs.python.org/3/library/stdtypes.html\#str}{str}

\end{description}\end{quote}

\end{fulllineitems}


\end{fulllineitems}



\section{quantity}
\label{\detokenize{support_rst/quantity:module-quantity}}\label{\detokenize{support_rst/quantity:quantity}}\label{\detokenize{support_rst/quantity::doc}}\index{quantity (module)}
Author: Valmor de Almeida \sphinxhref{mailto:dealmeidav@ornl.gov}{dealmeidav@ornl.gov}; vfda

This Quantity class is to be used with other classes in plant-level process modules.
\begin{description}
\item[{For unit testing do at the linux command prompt:}] \leavevmode
python quantity.py

\end{description}

Sat Sep  5 12:51:34 EDT 2015
\index{Quantity (class in quantity)}

\begin{fulllineitems}
\phantomsection\label{\detokenize{support_rst/quantity:quantity.Quantity}}\pysiglinewithargsret{\sphinxbfcode{\sphinxupquote{class }}\sphinxcode{\sphinxupquote{quantity.}}\sphinxbfcode{\sphinxupquote{Quantity}}}{\emph{name='null-quantity'}, \emph{formalName='null-quantity'}, \emph{value=0.0}, \emph{unit='null-unit'}}{}
Bases: \sphinxhref{https://docs.python.org/3/library/functions.html\#object}{\sphinxcode{\sphinxupquote{object}}}
\begin{description}
\item[{todo: this probably should not have a “value” for the same reason as Specie.}] \leavevmode
this needs some thinking.

\end{description}

well not so fast. This can be used to build a quantity with anything as a
value. For instance a history of the quantity as a time series.
\index{GetFormalName() (quantity.Quantity method)}

\begin{fulllineitems}
\phantomsection\label{\detokenize{support_rst/quantity:quantity.Quantity.GetFormalName}}\pysiglinewithargsret{\sphinxbfcode{\sphinxupquote{GetFormalName}}}{}{}
Returns the formal name of the quantity.
\begin{quote}\begin{description}
\item[{Returns}] \leavevmode
\sphinxstylestrong{formalName}

\item[{Return type}] \leavevmode
\sphinxhref{https://docs.python.org/3/library/stdtypes.html\#str}{str}

\end{description}\end{quote}

\end{fulllineitems}

\index{GetUnit() (quantity.Quantity method)}

\begin{fulllineitems}
\phantomsection\label{\detokenize{support_rst/quantity:quantity.Quantity.GetUnit}}\pysiglinewithargsret{\sphinxbfcode{\sphinxupquote{GetUnit}}}{}{}
Returns the units of the quantity.
\begin{quote}\begin{description}
\item[{Returns}] \leavevmode
\sphinxstylestrong{unit}

\item[{Return type}] \leavevmode
\sphinxhref{https://docs.python.org/3/library/stdtypes.html\#str}{str}

\end{description}\end{quote}

\end{fulllineitems}

\index{GetValue() (quantity.Quantity method)}

\begin{fulllineitems}
\phantomsection\label{\detokenize{support_rst/quantity:quantity.Quantity.GetValue}}\pysiglinewithargsret{\sphinxbfcode{\sphinxupquote{GetValue}}}{}{}
Gets the numerical value of the quantity.
\begin{quote}\begin{description}
\item[{Returns}] \leavevmode
\sphinxstylestrong{value}

\item[{Return type}] \leavevmode
any type

\end{description}\end{quote}

\end{fulllineitems}

\index{SetFormalName() (quantity.Quantity method)}

\begin{fulllineitems}
\phantomsection\label{\detokenize{support_rst/quantity:quantity.Quantity.SetFormalName}}\pysiglinewithargsret{\sphinxbfcode{\sphinxupquote{SetFormalName}}}{\emph{fn}}{}
Sets the formal name of the property to fn.
\begin{quote}\begin{description}
\item[{Parameters}] \leavevmode
\sphinxstyleliteralstrong{\sphinxupquote{fn}} (\sphinxhref{https://docs.python.org/3/library/stdtypes.html\#str}{\sphinxstyleliteralemphasis{\sphinxupquote{str}}}) \textendash{} 

\end{description}\end{quote}

\end{fulllineitems}

\index{SetName() (quantity.Quantity method)}

\begin{fulllineitems}
\phantomsection\label{\detokenize{support_rst/quantity:quantity.Quantity.SetName}}\pysiglinewithargsret{\sphinxbfcode{\sphinxupquote{SetName}}}{\emph{n}}{}
Sets the name of the quantity in question to n.
\begin{quote}\begin{description}
\item[{Parameters}] \leavevmode
\sphinxstyleliteralstrong{\sphinxupquote{n}} (\sphinxhref{https://docs.python.org/3/library/stdtypes.html\#str}{\sphinxstyleliteralemphasis{\sphinxupquote{str}}}) \textendash{} 

\end{description}\end{quote}

\end{fulllineitems}

\index{SetUnit() (quantity.Quantity method)}

\begin{fulllineitems}
\phantomsection\label{\detokenize{support_rst/quantity:quantity.Quantity.SetUnit}}\pysiglinewithargsret{\sphinxbfcode{\sphinxupquote{SetUnit}}}{\emph{f}}{}
Sets the units of the quantity to f (for example, density would be in
units of g/cc.
\begin{quote}\begin{description}
\item[{Parameters}] \leavevmode
\sphinxstyleliteralstrong{\sphinxupquote{f}} (\sphinxhref{https://docs.python.org/3/library/stdtypes.html\#str}{\sphinxstyleliteralemphasis{\sphinxupquote{str}}}) \textendash{} 

\end{description}\end{quote}

\end{fulllineitems}

\index{SetValue() (quantity.Quantity method)}

\begin{fulllineitems}
\phantomsection\label{\detokenize{support_rst/quantity:quantity.Quantity.SetValue}}\pysiglinewithargsret{\sphinxbfcode{\sphinxupquote{SetValue}}}{\emph{v}}{}
Sets the numerical value of the quantity to v.
\begin{quote}\begin{description}
\item[{Parameters}] \leavevmode
\sphinxstyleliteralstrong{\sphinxupquote{v}} (\sphinxhref{https://docs.python.org/3/library/functions.html\#float}{\sphinxstyleliteralemphasis{\sphinxupquote{float}}}) \textendash{} 

\end{description}\end{quote}

\end{fulllineitems}

\index{\_\_repr\_\_() (quantity.Quantity method)}

\begin{fulllineitems}
\phantomsection\label{\detokenize{support_rst/quantity:quantity.Quantity.__repr__}}\pysiglinewithargsret{\sphinxbfcode{\sphinxupquote{\_\_repr\_\_}}}{}{}
Used to print the data stored by the quantity class. Will print out
name, formal name, the value of the quantity and its unit.
\begin{quote}\begin{description}
\item[{Returns}] \leavevmode
\sphinxstylestrong{s}

\item[{Return type}] \leavevmode
\sphinxhref{https://docs.python.org/3/library/stdtypes.html\#str}{str}

\end{description}\end{quote}

\end{fulllineitems}

\index{\_\_str\_\_() (quantity.Quantity method)}

\begin{fulllineitems}
\phantomsection\label{\detokenize{support_rst/quantity:quantity.Quantity.__str__}}\pysiglinewithargsret{\sphinxbfcode{\sphinxupquote{\_\_str\_\_}}}{}{}
Used to print the data stored by the quantity class. Will print out
name, formal name, the value of the quantity and its unit.
\begin{quote}\begin{description}
\item[{Returns}] \leavevmode
\sphinxstylestrong{s}

\item[{Return type}] \leavevmode
\sphinxhref{https://docs.python.org/3/library/stdtypes.html\#str}{str}

\end{description}\end{quote}

\end{fulllineitems}

\index{formalName (quantity.Quantity attribute)}

\begin{fulllineitems}
\phantomsection\label{\detokenize{support_rst/quantity:quantity.Quantity.formalName}}\pysigline{\sphinxbfcode{\sphinxupquote{formalName}}}
Returns the formal name of the quantity.
\begin{quote}\begin{description}
\item[{Returns}] \leavevmode
\sphinxstylestrong{formalName}

\item[{Return type}] \leavevmode
\sphinxhref{https://docs.python.org/3/library/stdtypes.html\#str}{str}

\end{description}\end{quote}

\end{fulllineitems}

\index{formal\_name (quantity.Quantity attribute)}

\begin{fulllineitems}
\phantomsection\label{\detokenize{support_rst/quantity:quantity.Quantity.formal_name}}\pysigline{\sphinxbfcode{\sphinxupquote{formal\_name}}}
Returns the formal name of the quantity.
\begin{quote}\begin{description}
\item[{Returns}] \leavevmode
\sphinxstylestrong{formalName}

\item[{Return type}] \leavevmode
\sphinxhref{https://docs.python.org/3/library/stdtypes.html\#str}{str}

\end{description}\end{quote}

\end{fulllineitems}

\index{get\_name() (quantity.Quantity method)}

\begin{fulllineitems}
\phantomsection\label{\detokenize{support_rst/quantity:quantity.Quantity.get_name}}\pysiglinewithargsret{\sphinxbfcode{\sphinxupquote{get\_name}}}{}{}
Returns the name of the quantity.
\begin{quote}\begin{description}
\item[{Returns}] \leavevmode
\sphinxstylestrong{name}

\item[{Return type}] \leavevmode
\sphinxhref{https://docs.python.org/3/library/stdtypes.html\#str}{str}

\end{description}\end{quote}

\end{fulllineitems}

\index{name (quantity.Quantity attribute)}

\begin{fulllineitems}
\phantomsection\label{\detokenize{support_rst/quantity:quantity.Quantity.name}}\pysigline{\sphinxbfcode{\sphinxupquote{name}}}
Returns the name of the quantity.
\begin{quote}\begin{description}
\item[{Returns}] \leavevmode
\sphinxstylestrong{name}

\item[{Return type}] \leavevmode
\sphinxhref{https://docs.python.org/3/library/stdtypes.html\#str}{str}

\end{description}\end{quote}

\end{fulllineitems}

\index{plot() (quantity.Quantity method)}

\begin{fulllineitems}
\phantomsection\label{\detokenize{support_rst/quantity:quantity.Quantity.plot}}\pysiglinewithargsret{\sphinxbfcode{\sphinxupquote{plot}}}{\emph{x\_scaling=1}, \emph{y\_scaling=1}, \emph{title=None}, \emph{x\_label='x'}, \emph{y\_label=None}, \emph{file\_name=None}, \emph{same\_axis=True}, \emph{dpi=300}}{}
This will support a few possibities for data storage in the self.\_\_value
member.

Pandas Series. If self.\_\_value is a Pandas Series, plot against the index.
However the type stored in the Series matter. Suppose it is a series
of a \sphinxtitleref{numpy} array. This must be of the same rank for every entry.
This plot method assumes it is an iterable type of the same length for every
entry in the series. A plot of all elements in the type against the index of
the series will be made. The plot may have all elements in one axis or
each element in its own axis.

\end{fulllineitems}

\index{unit (quantity.Quantity attribute)}

\begin{fulllineitems}
\phantomsection\label{\detokenize{support_rst/quantity:quantity.Quantity.unit}}\pysigline{\sphinxbfcode{\sphinxupquote{unit}}}
Returns the units of the quantity.
\begin{quote}\begin{description}
\item[{Returns}] \leavevmode
\sphinxstylestrong{unit}

\item[{Return type}] \leavevmode
\sphinxhref{https://docs.python.org/3/library/stdtypes.html\#str}{str}

\end{description}\end{quote}

\end{fulllineitems}

\index{value (quantity.Quantity attribute)}

\begin{fulllineitems}
\phantomsection\label{\detokenize{support_rst/quantity:quantity.Quantity.value}}\pysigline{\sphinxbfcode{\sphinxupquote{value}}}
Gets the numerical value of the quantity.
\begin{quote}\begin{description}
\item[{Returns}] \leavevmode
\sphinxstylestrong{value}

\item[{Return type}] \leavevmode
any type

\end{description}\end{quote}

\end{fulllineitems}


\end{fulllineitems}



\section{specie}
\label{\detokenize{support_rst/specie:module-specie}}\label{\detokenize{support_rst/specie:specie}}\label{\detokenize{support_rst/specie::doc}}\index{specie (module)}
Author: Valmor de Almeida \sphinxhref{mailto:dealmeidav@ornl.gov}{dealmeidav@ornl.gov}; vfda

This Specie class is to be used with other classes in plant-level process modules.
\begin{description}
\item[{NB: Species is always used either in singular or plural cases, the class}] \leavevmode
named here reflects one species. If many species are used in an external
context, the species object name can be used without conflict.

\item[{For unit testing do at the linux command prompt:}] \leavevmode
python specie.py

\item[{NB: The Specie() class encapsulates either the molecular or empirical chemical}] \leavevmode
formula of a compound.
This is done as follows. Say MAO2 is either a molecular or empirical chemical
formula of a ficticious compound denoting minor actinides dioxide. The list
of atoms is given as follows:

{[}‘0.49*Np-237’, ‘0.42*Am-241’, ‘0.08*Am-243’, ‘0.01*Cm-244’, ‘2.0*O-16’{]}

note the MA forming nuclides add to 1 = 0.49 + 0.42 + 0.08 + 0.01. Therefore
the number of atoms in this compound is 3. 1 MA “atom” and 2 O.
Note that the total number of “atoms” is obtained by summing all multipliers:
0.49 + 0.42 + 0.08 + 0.01 + 2.0.
The nuclide is indicated by the element symbol followed by a dash and the
atomic mass number. Here the number of nuclide types is 5 (self.\_nNuclideTypes).

The numbers preceeding the nuclide symbol before the * will be referred to as
multipliers. The sum of the multipliers will add to the number of “atoms” in
the formula. WARNING: a multiplier could be in the format 0.00e-00. In this
case a hiphen may appear twice, e.g.: 1.549e-09*U-233

Other forms can be used for common true species

{[}‘Np-237’, ‘2.0*O-16’{]} or {[}‘Np-237’, ‘O-16’, ‘O-16’{]} or {[} ‘2*H’, ‘O’ {]} or
{[} ‘H’, ‘O’, ‘H’ {]}  etc…

This code will calculate the molar mass of any species with a given valid
atom list using a provided periodic table of chemical elements. The user
can also reset the value of the molar mass with a setter method.

\end{description}

Sat May  9 21:40:48 EDT 2015 created; vfda
\index{Specie (class in specie)}

\begin{fulllineitems}
\phantomsection\label{\detokenize{support_rst/specie:specie.Specie}}\pysiglinewithargsret{\sphinxbfcode{\sphinxupquote{class }}\sphinxcode{\sphinxupquote{specie.}}\sphinxbfcode{\sphinxupquote{Specie}}}{\emph{name='null'}, \emph{formula\_name='null'}, \emph{phase='null'}, \emph{atoms={[}{]}}, \emph{molarCC=0.0}, \emph{massCC=0.0}, \emph{flag=None}}{}
Bases: \sphinxhref{https://docs.python.org/3/library/functions.html\#object}{\sphinxcode{\sphinxupquote{object}}}
\begin{description}
\item[{todo: phase should not be here; concentrations should not be here}] \leavevmode
only molar quantities should be here
see the Phase container

\end{description}
\index{GetAtoms() (specie.Specie method)}

\begin{fulllineitems}
\phantomsection\label{\detokenize{support_rst/specie:specie.Specie.GetAtoms}}\pysiglinewithargsret{\sphinxbfcode{\sphinxupquote{GetAtoms}}}{}{}
\end{fulllineitems}

\index{GetFlag() (specie.Specie method)}

\begin{fulllineitems}
\phantomsection\label{\detokenize{support_rst/specie:specie.Specie.GetFlag}}\pysiglinewithargsret{\sphinxbfcode{\sphinxupquote{GetFlag}}}{}{}
Returns the flag associated with the species.
\begin{quote}\begin{description}
\item[{Returns}] \leavevmode
\sphinxstylestrong{flag}

\item[{Return type}] \leavevmode
\sphinxhref{https://docs.python.org/3/library/stdtypes.html\#str}{str}

\end{description}\end{quote}

\end{fulllineitems}

\index{GetFormula() (specie.Specie method)}

\begin{fulllineitems}
\phantomsection\label{\detokenize{support_rst/specie:specie.Specie.GetFormula}}\pysiglinewithargsret{\sphinxbfcode{\sphinxupquote{GetFormula}}}{}{}
Returns the molecular or empirical formula of the species. It is
usually a list, for example, of the form {[}‘2*H’, ‘O’{]}.
\begin{quote}\begin{description}
\item[{Returns}] \leavevmode
\sphinxstylestrong{formula}

\item[{Return type}] \leavevmode
\sphinxhref{https://docs.python.org/3/library/stdtypes.html\#list}{list}

\end{description}\end{quote}

\end{fulllineitems}

\index{GetFormulaName() (specie.Specie method)}

\begin{fulllineitems}
\phantomsection\label{\detokenize{support_rst/specie:specie.Specie.GetFormulaName}}\pysiglinewithargsret{\sphinxbfcode{\sphinxupquote{GetFormulaName}}}{}{}
Returns the formulaic name of the compound. For example, “Dihydrogen
monoxide”.
\begin{quote}\begin{description}
\item[{Returns}] \leavevmode
\sphinxstylestrong{self.\_\_formula\_name}

\item[{Return type}] \leavevmode
\sphinxhref{https://docs.python.org/3/library/stdtypes.html\#str}{str}

\end{description}\end{quote}

\end{fulllineitems}

\index{GetMassCC() (specie.Specie method)}

\begin{fulllineitems}
\phantomsection\label{\detokenize{support_rst/specie:specie.Specie.GetMassCC}}\pysiglinewithargsret{\sphinxbfcode{\sphinxupquote{GetMassCC}}}{}{}
Returns the numerical value of the mass density of the species
(mass/volume).
\begin{quote}\begin{description}
\item[{Returns}] \leavevmode
\sphinxstylestrong{massCC}

\item[{Return type}] \leavevmode
\sphinxhref{https://docs.python.org/3/library/functions.html\#float}{float}

\end{description}\end{quote}

\end{fulllineitems}

\index{GetMassCCUnit() (specie.Specie method)}

\begin{fulllineitems}
\phantomsection\label{\detokenize{support_rst/specie:specie.Specie.GetMassCCUnit}}\pysiglinewithargsret{\sphinxbfcode{\sphinxupquote{GetMassCCUnit}}}{}{}
Returns the unit used to measure the mass density of the species.
\begin{quote}\begin{description}
\item[{Returns}] \leavevmode
\sphinxstylestrong{massCCUnit}

\item[{Return type}] \leavevmode
\sphinxhref{https://docs.python.org/3/library/stdtypes.html\#str}{str}

\end{description}\end{quote}

\end{fulllineitems}

\index{GetMolarCC() (specie.Specie method)}

\begin{fulllineitems}
\phantomsection\label{\detokenize{support_rst/specie:specie.Specie.GetMolarCC}}\pysiglinewithargsret{\sphinxbfcode{\sphinxupquote{GetMolarCC}}}{}{}
Returns the numerical value for the number (molar) density of the
species (moles/volume).
\begin{quote}\begin{description}
\item[{Returns}] \leavevmode
\sphinxstylestrong{molarCC}

\item[{Return type}] \leavevmode
\sphinxhref{https://docs.python.org/3/library/functions.html\#float}{float}

\end{description}\end{quote}

\end{fulllineitems}

\index{GetMolarCCUnit() (specie.Specie method)}

\begin{fulllineitems}
\phantomsection\label{\detokenize{support_rst/specie:specie.Specie.GetMolarCCUnit}}\pysiglinewithargsret{\sphinxbfcode{\sphinxupquote{GetMolarCCUnit}}}{}{}
Returns the unit used to measure molar density of the species.
\begin{quote}\begin{description}
\item[{Returns}] \leavevmode
\sphinxstylestrong{molarCCUnit}

\item[{Return type}] \leavevmode
\sphinxhref{https://docs.python.org/3/library/stdtypes.html\#str}{str}

\end{description}\end{quote}

\end{fulllineitems}

\index{GetMolarGammaPwr() (specie.Specie method)}

\begin{fulllineitems}
\phantomsection\label{\detokenize{support_rst/specie:specie.Specie.GetMolarGammaPwr}}\pysiglinewithargsret{\sphinxbfcode{\sphinxupquote{GetMolarGammaPwr}}}{}{}
Returns the amount of gamma radiation produced per mole of this species
(measured in units of power).
\begin{quote}\begin{description}
\item[{Returns}] \leavevmode
\sphinxstylestrong{molarGammaPwr}

\item[{Return type}] \leavevmode
\sphinxhref{https://docs.python.org/3/library/functions.html\#float}{float}

\end{description}\end{quote}

\end{fulllineitems}

\index{GetMolarGammaPwrUnit() (specie.Specie method)}

\begin{fulllineitems}
\phantomsection\label{\detokenize{support_rst/specie:specie.Specie.GetMolarGammaPwrUnit}}\pysiglinewithargsret{\sphinxbfcode{\sphinxupquote{GetMolarGammaPwrUnit}}}{}{}
Returns the unit used to measure the amount of gamma radiation produced
per mole of this species.
\begin{quote}\begin{description}
\item[{Returns}] \leavevmode
\sphinxstylestrong{molarGammaPwrUnit}

\item[{Return type}] \leavevmode
\sphinxhref{https://docs.python.org/3/library/stdtypes.html\#str}{str}

\end{description}\end{quote}

\end{fulllineitems}

\index{GetMolarHeatPwr() (specie.Specie method)}

\begin{fulllineitems}
\phantomsection\label{\detokenize{support_rst/specie:specie.Specie.GetMolarHeatPwr}}\pysiglinewithargsret{\sphinxbfcode{\sphinxupquote{GetMolarHeatPwr}}}{}{}
Returns the amount of heat generated per mole of this species.
\begin{quote}\begin{description}
\item[{Returns}] \leavevmode
\sphinxstylestrong{molarHeatPwr}

\item[{Return type}] \leavevmode
\sphinxhref{https://docs.python.org/3/library/functions.html\#float}{float}

\end{description}\end{quote}

\end{fulllineitems}

\index{GetMolarHeatPwrUnit() (specie.Specie method)}

\begin{fulllineitems}
\phantomsection\label{\detokenize{support_rst/specie:specie.Specie.GetMolarHeatPwrUnit}}\pysiglinewithargsret{\sphinxbfcode{\sphinxupquote{GetMolarHeatPwrUnit}}}{}{}
Returns the unit used to measure the amount of heat generated per mole
of this species.
\begin{quote}\begin{description}
\item[{Returns}] \leavevmode
\sphinxstylestrong{molarHeatPwrUnit}

\item[{Return type}] \leavevmode
\sphinxhref{https://docs.python.org/3/library/stdtypes.html\#str}{str}

\end{description}\end{quote}

\end{fulllineitems}

\index{GetMolarMass() (specie.Specie method)}

\begin{fulllineitems}
\phantomsection\label{\detokenize{support_rst/specie:specie.Specie.GetMolarMass}}\pysiglinewithargsret{\sphinxbfcode{\sphinxupquote{GetMolarMass}}}{}{}
Returns the numerical value for the molar mass of the species. Units
are given by molarMassUnit.
\begin{quote}\begin{description}
\item[{Returns}] \leavevmode
\sphinxstylestrong{molarMass}

\item[{Return type}] \leavevmode
\sphinxhref{https://docs.python.org/3/library/functions.html\#float}{float}

\end{description}\end{quote}

\end{fulllineitems}

\index{GetMolarMassUnit() (specie.Specie method)}

\begin{fulllineitems}
\phantomsection\label{\detokenize{support_rst/specie:specie.Specie.GetMolarMassUnit}}\pysiglinewithargsret{\sphinxbfcode{\sphinxupquote{GetMolarMassUnit}}}{}{}
Returns the unit used to measure the molar mass of the species.
\begin{quote}\begin{description}
\item[{Returns}] \leavevmode
\sphinxstylestrong{molarMassUnit}

\item[{Return type}] \leavevmode
\sphinxhref{https://docs.python.org/3/library/stdtypes.html\#str}{str}

\end{description}\end{quote}

\end{fulllineitems}

\index{GetMolarRadioactivity() (specie.Specie method)}

\begin{fulllineitems}
\phantomsection\label{\detokenize{support_rst/specie:specie.Specie.GetMolarRadioactivity}}\pysiglinewithargsret{\sphinxbfcode{\sphinxupquote{GetMolarRadioactivity}}}{}{}
Returns the numerical value for molar radioactivity of the species.
\begin{quote}\begin{description}
\item[{Returns}] \leavevmode
\sphinxstylestrong{molarRadioactivity}

\item[{Return type}] \leavevmode
\sphinxhref{https://docs.python.org/3/library/functions.html\#float}{float}

\end{description}\end{quote}

\end{fulllineitems}

\index{GetMolarRadioactivityFractions() (specie.Specie method)}

\begin{fulllineitems}
\phantomsection\label{\detokenize{support_rst/specie:specie.Specie.GetMolarRadioactivityFractions}}\pysiglinewithargsret{\sphinxbfcode{\sphinxupquote{GetMolarRadioactivityFractions}}}{}{}
Returns a list of numbers that speciefies the \% of molar reactivity
that comes from each type of atom in the species. For example, a
molarRadioactivityFraction of {[}0.65, 0.35{]} for water means that 65\%
of the molar radioactivity comes from the hydrogen atoms and 35\% comes
from the oxygen atom.
\begin{quote}\begin{description}
\item[{Returns}] \leavevmode
\sphinxstylestrong{molarRadioactivityFractions}

\item[{Return type}] \leavevmode
\sphinxhref{https://docs.python.org/3/library/stdtypes.html\#list}{list}

\end{description}\end{quote}

\end{fulllineitems}

\index{GetMolarRadioactivityUnit() (specie.Specie method)}

\begin{fulllineitems}
\phantomsection\label{\detokenize{support_rst/specie:specie.Specie.GetMolarRadioactivityUnit}}\pysiglinewithargsret{\sphinxbfcode{\sphinxupquote{GetMolarRadioactivityUnit}}}{}{}
Returns the unit used to measure molar radioactivity.
\begin{quote}\begin{description}
\item[{Returns}] \leavevmode
\sphinxstylestrong{molarRadioactivityUnit}

\item[{Return type}] \leavevmode
\sphinxhref{https://docs.python.org/3/library/stdtypes.html\#str}{str}

\end{description}\end{quote}

\end{fulllineitems}

\index{GetNAtoms() (specie.Specie method)}

\begin{fulllineitems}
\phantomsection\label{\detokenize{support_rst/specie:specie.Specie.GetNAtoms}}\pysiglinewithargsret{\sphinxbfcode{\sphinxupquote{GetNAtoms}}}{}{}
Returns the total number of atoms comprising the species. For example,
water is comprised of three atoms.
\begin{quote}\begin{description}
\item[{Returns}] \leavevmode
\sphinxstylestrong{nAtoms}

\item[{Return type}] \leavevmode
\sphinxhref{https://docs.python.org/3/library/functions.html\#int}{int}

\end{description}\end{quote}

\end{fulllineitems}

\index{GetNNuclideTypes() (specie.Specie method)}

\begin{fulllineitems}
\phantomsection\label{\detokenize{support_rst/specie:specie.Specie.GetNNuclideTypes}}\pysiglinewithargsret{\sphinxbfcode{\sphinxupquote{GetNNuclideTypes}}}{}{}
Returns the number of different types of atoms comprising the species.
For example, water is composed of two different types of atoms,
hydrogen and oxygen.
\begin{quote}\begin{description}
\item[{Returns}] \leavevmode
\sphinxstylestrong{nNuclideTypes}

\item[{Return type}] \leavevmode
\sphinxhref{https://docs.python.org/3/library/functions.html\#int}{int}

\end{description}\end{quote}

\end{fulllineitems}

\index{GetName() (specie.Specie method)}

\begin{fulllineitems}
\phantomsection\label{\detokenize{support_rst/specie:specie.Specie.GetName}}\pysiglinewithargsret{\sphinxbfcode{\sphinxupquote{GetName}}}{}{}
Returns the empirical name of the species. For example, “water”.
\begin{quote}\begin{description}
\item[{Returns}] \leavevmode
\sphinxstylestrong{name}

\item[{Return type}] \leavevmode
\sphinxhref{https://docs.python.org/3/library/stdtypes.html\#str}{str}

\end{description}\end{quote}

\end{fulllineitems}

\index{GetPhase() (specie.Specie method)}

\begin{fulllineitems}
\phantomsection\label{\detokenize{support_rst/specie:specie.Specie.GetPhase}}\pysiglinewithargsret{\sphinxbfcode{\sphinxupquote{GetPhase}}}{}{}
Returns the phase history of the species.
\begin{quote}\begin{description}
\item[{Returns}] \leavevmode
\sphinxstylestrong{phase}

\item[{Return type}] \leavevmode
dataFrame

\end{description}\end{quote}

\end{fulllineitems}

\index{SetAtoms() (specie.Specie method)}

\begin{fulllineitems}
\phantomsection\label{\detokenize{support_rst/specie:specie.Specie.SetAtoms}}\pysiglinewithargsret{\sphinxbfcode{\sphinxupquote{SetAtoms}}}{\emph{atoms}}{}
\end{fulllineitems}

\index{SetFlag() (specie.Specie method)}

\begin{fulllineitems}
\phantomsection\label{\detokenize{support_rst/specie:specie.Specie.SetFlag}}\pysiglinewithargsret{\sphinxbfcode{\sphinxupquote{SetFlag}}}{\emph{f}}{}
Sets the flag associated with the species to f.
\begin{quote}\begin{description}
\item[{Parameters}] \leavevmode
\sphinxstyleliteralstrong{\sphinxupquote{f}} (\sphinxhref{https://docs.python.org/3/library/stdtypes.html\#str}{\sphinxstyleliteralemphasis{\sphinxupquote{str}}}) \textendash{} 

\end{description}\end{quote}

\end{fulllineitems}

\index{SetFormula() (specie.Specie method)}

\begin{fulllineitems}
\phantomsection\label{\detokenize{support_rst/specie:specie.Specie.SetFormula}}\pysiglinewithargsret{\sphinxbfcode{\sphinxupquote{SetFormula}}}{\emph{atoms}}{}
Sets the species’ formula equal to atoms. Will automatically update
the molar mass of the species, and will also fail if atoms is not a
list of strings.
\begin{quote}\begin{description}
\item[{Parameters}] \leavevmode
\sphinxstyleliteralstrong{\sphinxupquote{atoms}} (\sphinxhref{https://docs.python.org/3/library/stdtypes.html\#list}{\sphinxstyleliteralemphasis{\sphinxupquote{list}}}) \textendash{} 

\end{description}\end{quote}

\end{fulllineitems}

\index{SetFormulaName() (specie.Specie method)}

\begin{fulllineitems}
\phantomsection\label{\detokenize{support_rst/specie:specie.Specie.SetFormulaName}}\pysiglinewithargsret{\sphinxbfcode{\sphinxupquote{SetFormulaName}}}{\emph{f}}{}
Sets the formulaic name to f.
\begin{quote}\begin{description}
\item[{Returns}] \leavevmode
\sphinxstylestrong{self.\_\_formula\_name}

\item[{Return type}] \leavevmode
\sphinxhref{https://docs.python.org/3/library/stdtypes.html\#str}{str}

\end{description}\end{quote}

\end{fulllineitems}

\index{SetMassCC() (specie.Specie method)}

\begin{fulllineitems}
\phantomsection\label{\detokenize{support_rst/specie:specie.Specie.SetMassCC}}\pysiglinewithargsret{\sphinxbfcode{\sphinxupquote{SetMassCC}}}{\emph{v}}{}
Sets the numerical value of the mass density equal to v.
\begin{quote}\begin{description}
\item[{Parameters}] \leavevmode
\sphinxstyleliteralstrong{\sphinxupquote{v}} (\sphinxhref{https://docs.python.org/3/library/functions.html\#float}{\sphinxstyleliteralemphasis{\sphinxupquote{float}}}) \textendash{} 

\end{description}\end{quote}

\end{fulllineitems}

\index{SetMassCCUnit() (specie.Specie method)}

\begin{fulllineitems}
\phantomsection\label{\detokenize{support_rst/specie:specie.Specie.SetMassCCUnit}}\pysiglinewithargsret{\sphinxbfcode{\sphinxupquote{SetMassCCUnit}}}{\emph{v}}{}
Sets the units used to measure mass density to v.
\begin{quote}\begin{description}
\item[{Parameters}] \leavevmode
\sphinxstyleliteralstrong{\sphinxupquote{v}} (\sphinxhref{https://docs.python.org/3/library/stdtypes.html\#str}{\sphinxstyleliteralemphasis{\sphinxupquote{str}}}) \textendash{} 

\end{description}\end{quote}

\end{fulllineitems}

\index{SetMolarCC() (specie.Specie method)}

\begin{fulllineitems}
\phantomsection\label{\detokenize{support_rst/specie:specie.Specie.SetMolarCC}}\pysiglinewithargsret{\sphinxbfcode{\sphinxupquote{SetMolarCC}}}{\emph{v}}{}
Sets the numerical value for the molar density of the species to v.
\begin{quote}\begin{description}
\item[{Parameters}] \leavevmode
\sphinxstyleliteralstrong{\sphinxupquote{v}} (\sphinxhref{https://docs.python.org/3/library/functions.html\#float}{\sphinxstyleliteralemphasis{\sphinxupquote{float}}}) \textendash{} 

\end{description}\end{quote}

\end{fulllineitems}

\index{SetMolarCCUnit() (specie.Specie method)}

\begin{fulllineitems}
\phantomsection\label{\detokenize{support_rst/specie:specie.Specie.SetMolarCCUnit}}\pysiglinewithargsret{\sphinxbfcode{\sphinxupquote{SetMolarCCUnit}}}{\emph{v}}{}
Sets the unit used to measure the molar density of the species to v.
\begin{quote}\begin{description}
\item[{Parameters}] \leavevmode
\sphinxstyleliteralstrong{\sphinxupquote{v}} (\sphinxhref{https://docs.python.org/3/library/stdtypes.html\#str}{\sphinxstyleliteralemphasis{\sphinxupquote{str}}}) \textendash{} 

\end{description}\end{quote}

\end{fulllineitems}

\index{SetMolarGammaPwr() (specie.Specie method)}

\begin{fulllineitems}
\phantomsection\label{\detokenize{support_rst/specie:specie.Specie.SetMolarGammaPwr}}\pysiglinewithargsret{\sphinxbfcode{\sphinxupquote{SetMolarGammaPwr}}}{\emph{v}}{}
Sets the amount of gamma radiation produced per mole of this species to
v.
\begin{quote}\begin{description}
\item[{Parameters}] \leavevmode
\sphinxstyleliteralstrong{\sphinxupquote{v}} (\sphinxhref{https://docs.python.org/3/library/functions.html\#float}{\sphinxstyleliteralemphasis{\sphinxupquote{float}}}) \textendash{} 

\end{description}\end{quote}

\end{fulllineitems}

\index{SetMolarGammaPwrUnit() (specie.Specie method)}

\begin{fulllineitems}
\phantomsection\label{\detokenize{support_rst/specie:specie.Specie.SetMolarGammaPwrUnit}}\pysiglinewithargsret{\sphinxbfcode{\sphinxupquote{SetMolarGammaPwrUnit}}}{\emph{v}}{}
Sets the unit used to measure the amount of gamma radiation produced
per mole of this species to v.
\begin{quote}\begin{description}
\item[{Parameters}] \leavevmode
\sphinxstyleliteralstrong{\sphinxupquote{v}} (\sphinxhref{https://docs.python.org/3/library/stdtypes.html\#str}{\sphinxstyleliteralemphasis{\sphinxupquote{str}}}) \textendash{} 

\end{description}\end{quote}

\end{fulllineitems}

\index{SetMolarHeatPwr() (specie.Specie method)}

\begin{fulllineitems}
\phantomsection\label{\detokenize{support_rst/specie:specie.Specie.SetMolarHeatPwr}}\pysiglinewithargsret{\sphinxbfcode{\sphinxupquote{SetMolarHeatPwr}}}{\emph{v}}{}
Sets the amount of heat generated per mole of this species to v.
\begin{quote}\begin{description}
\item[{Parameters}] \leavevmode
\sphinxstyleliteralstrong{\sphinxupquote{v}} (\sphinxhref{https://docs.python.org/3/library/functions.html\#float}{\sphinxstyleliteralemphasis{\sphinxupquote{float}}}) \textendash{} 

\end{description}\end{quote}

\end{fulllineitems}

\index{SetMolarHeatPwrUnit() (specie.Specie method)}

\begin{fulllineitems}
\phantomsection\label{\detokenize{support_rst/specie:specie.Specie.SetMolarHeatPwrUnit}}\pysiglinewithargsret{\sphinxbfcode{\sphinxupquote{SetMolarHeatPwrUnit}}}{\emph{v}}{}
Sets the unit used to measure the amount of heat generated per mole of
this species to v.
\begin{quote}\begin{description}
\item[{Parameters}] \leavevmode
\sphinxstyleliteralstrong{\sphinxupquote{v}} (\sphinxhref{https://docs.python.org/3/library/stdtypes.html\#str}{\sphinxstyleliteralemphasis{\sphinxupquote{str}}}) \textendash{} 

\end{description}\end{quote}

\end{fulllineitems}

\index{SetMolarMass() (specie.Specie method)}

\begin{fulllineitems}
\phantomsection\label{\detokenize{support_rst/specie:specie.Specie.SetMolarMass}}\pysiglinewithargsret{\sphinxbfcode{\sphinxupquote{SetMolarMass}}}{\emph{v}}{}
Sets the molar mass of the species equal to v.
\begin{quote}\begin{description}
\item[{Parameters}] \leavevmode
\sphinxstyleliteralstrong{\sphinxupquote{v}} (\sphinxhref{https://docs.python.org/3/library/functions.html\#float}{\sphinxstyleliteralemphasis{\sphinxupquote{float}}}) \textendash{} 

\end{description}\end{quote}

\end{fulllineitems}

\index{SetMolarMassUnit() (specie.Specie method)}

\begin{fulllineitems}
\phantomsection\label{\detokenize{support_rst/specie:specie.Specie.SetMolarMassUnit}}\pysiglinewithargsret{\sphinxbfcode{\sphinxupquote{SetMolarMassUnit}}}{\emph{v}}{}
Sets the unit used to measure the molar mass of the species to v.
\begin{quote}\begin{description}
\item[{Parameters}] \leavevmode
\sphinxstyleliteralstrong{\sphinxupquote{v}} (\sphinxhref{https://docs.python.org/3/library/stdtypes.html\#str}{\sphinxstyleliteralemphasis{\sphinxupquote{str}}}) \textendash{} 

\end{description}\end{quote}

\end{fulllineitems}

\index{SetMolarRadioactivity() (specie.Specie method)}

\begin{fulllineitems}
\phantomsection\label{\detokenize{support_rst/specie:specie.Specie.SetMolarRadioactivity}}\pysiglinewithargsret{\sphinxbfcode{\sphinxupquote{SetMolarRadioactivity}}}{\emph{v}}{}
Sets the molar radioactivity of the species equal to v.
\begin{quote}\begin{description}
\item[{Parameters}] \leavevmode
\sphinxstyleliteralstrong{\sphinxupquote{v}} (\sphinxhref{https://docs.python.org/3/library/functions.html\#float}{\sphinxstyleliteralemphasis{\sphinxupquote{float}}}) \textendash{} 

\end{description}\end{quote}

\end{fulllineitems}

\index{SetMolarRadioactivityFractions() (specie.Specie method)}

\begin{fulllineitems}
\phantomsection\label{\detokenize{support_rst/specie:specie.Specie.SetMolarRadioactivityFractions}}\pysiglinewithargsret{\sphinxbfcode{\sphinxupquote{SetMolarRadioactivityFractions}}}{\emph{fracs}}{}
Sets molarRadioactivityFractions equal to fracs. Fracs must be a list
of floatswith the same length as there are different atoms in the
species, or the function call will fail. (e.g. self.\_atoms and fracs
must be of the same length). Take care to ensure that the elements of
fracs match with the elements of self.\_atoms! (65\% is in the same
position in fracs as hydrogen is in self.\_atoms, following the above
example).
\begin{quote}\begin{description}
\item[{Parameters}] \leavevmode
\sphinxstyleliteralstrong{\sphinxupquote{fracs}} (\sphinxhref{https://docs.python.org/3/library/stdtypes.html\#list}{\sphinxstyleliteralemphasis{\sphinxupquote{list}}}) \textendash{} 

\end{description}\end{quote}

\end{fulllineitems}

\index{SetMolarRadioactivityUnit() (specie.Specie method)}

\begin{fulllineitems}
\phantomsection\label{\detokenize{support_rst/specie:specie.Specie.SetMolarRadioactivityUnit}}\pysiglinewithargsret{\sphinxbfcode{\sphinxupquote{SetMolarRadioactivityUnit}}}{\emph{v}}{}
Sets the unit used to measure molar radioactivity to v.
\begin{quote}\begin{description}
\item[{Parameters}] \leavevmode
\sphinxstyleliteralstrong{\sphinxupquote{v}} (\sphinxhref{https://docs.python.org/3/library/stdtypes.html\#str}{\sphinxstyleliteralemphasis{\sphinxupquote{str}}}) \textendash{} 

\end{description}\end{quote}

\end{fulllineitems}

\index{SetName() (specie.Specie method)}

\begin{fulllineitems}
\phantomsection\label{\detokenize{support_rst/specie:specie.Specie.SetName}}\pysiglinewithargsret{\sphinxbfcode{\sphinxupquote{SetName}}}{\emph{n}}{}
Sets the empirical  name of the species to n.
\begin{quote}\begin{description}
\item[{Parameters}] \leavevmode
\sphinxstyleliteralstrong{\sphinxupquote{n}} (\sphinxhref{https://docs.python.org/3/library/stdtypes.html\#str}{\sphinxstyleliteralemphasis{\sphinxupquote{str}}}) \textendash{} 

\end{description}\end{quote}

\end{fulllineitems}

\index{SetPhase() (specie.Specie method)}

\begin{fulllineitems}
\phantomsection\label{\detokenize{support_rst/specie:specie.Specie.SetPhase}}\pysiglinewithargsret{\sphinxbfcode{\sphinxupquote{SetPhase}}}{\emph{p}}{}
Sets the phase history to p.
\begin{quote}\begin{description}
\item[{Parameters}] \leavevmode
\sphinxstyleliteralstrong{\sphinxupquote{p}} (\sphinxstyleliteralemphasis{\sphinxupquote{dataFrame}}) \textendash{} 

\end{description}\end{quote}

\end{fulllineitems}

\index{atoms (specie.Specie attribute)}

\begin{fulllineitems}
\phantomsection\label{\detokenize{support_rst/specie:specie.Specie.atoms}}\pysigline{\sphinxbfcode{\sphinxupquote{atoms}}}
\end{fulllineitems}

\index{flag (specie.Specie attribute)}

\begin{fulllineitems}
\phantomsection\label{\detokenize{support_rst/specie:specie.Specie.flag}}\pysigline{\sphinxbfcode{\sphinxupquote{flag}}}
Returns the flag associated with the species.
\begin{quote}\begin{description}
\item[{Returns}] \leavevmode
\sphinxstylestrong{flag}

\item[{Return type}] \leavevmode
\sphinxhref{https://docs.python.org/3/library/stdtypes.html\#str}{str}

\end{description}\end{quote}

\end{fulllineitems}

\index{formula (specie.Specie attribute)}

\begin{fulllineitems}
\phantomsection\label{\detokenize{support_rst/specie:specie.Specie.formula}}\pysigline{\sphinxbfcode{\sphinxupquote{formula}}}
Returns the molecular or empirical formula of the species. It is
usually a list, for example, of the form {[}‘2*H’, ‘O’{]}.
\begin{quote}\begin{description}
\item[{Returns}] \leavevmode
\sphinxstylestrong{formula}

\item[{Return type}] \leavevmode
\sphinxhref{https://docs.python.org/3/library/stdtypes.html\#list}{list}

\end{description}\end{quote}

\end{fulllineitems}

\index{formula\_name (specie.Specie attribute)}

\begin{fulllineitems}
\phantomsection\label{\detokenize{support_rst/specie:specie.Specie.formula_name}}\pysigline{\sphinxbfcode{\sphinxupquote{formula\_name}}}
Returns the formulaic name of the compound. For example, “Dihydrogen
monoxide”.
\begin{quote}\begin{description}
\item[{Returns}] \leavevmode
\sphinxstylestrong{self.\_\_formula\_name}

\item[{Return type}] \leavevmode
\sphinxhref{https://docs.python.org/3/library/stdtypes.html\#str}{str}

\end{description}\end{quote}

\end{fulllineitems}

\index{massCC (specie.Specie attribute)}

\begin{fulllineitems}
\phantomsection\label{\detokenize{support_rst/specie:specie.Specie.massCC}}\pysigline{\sphinxbfcode{\sphinxupquote{massCC}}}
Returns the numerical value of the mass density of the species
(mass/volume).
\begin{quote}\begin{description}
\item[{Returns}] \leavevmode
\sphinxstylestrong{massCC}

\item[{Return type}] \leavevmode
\sphinxhref{https://docs.python.org/3/library/functions.html\#float}{float}

\end{description}\end{quote}

\end{fulllineitems}

\index{massCCUnit (specie.Specie attribute)}

\begin{fulllineitems}
\phantomsection\label{\detokenize{support_rst/specie:specie.Specie.massCCUnit}}\pysigline{\sphinxbfcode{\sphinxupquote{massCCUnit}}}
Returns the unit used to measure the mass density of the species.
\begin{quote}\begin{description}
\item[{Returns}] \leavevmode
\sphinxstylestrong{massCCUnit}

\item[{Return type}] \leavevmode
\sphinxhref{https://docs.python.org/3/library/stdtypes.html\#str}{str}

\end{description}\end{quote}

\end{fulllineitems}

\index{molarCC (specie.Specie attribute)}

\begin{fulllineitems}
\phantomsection\label{\detokenize{support_rst/specie:specie.Specie.molarCC}}\pysigline{\sphinxbfcode{\sphinxupquote{molarCC}}}
Returns the numerical value for the number (molar) density of the
species (moles/volume).
\begin{quote}\begin{description}
\item[{Returns}] \leavevmode
\sphinxstylestrong{molarCC}

\item[{Return type}] \leavevmode
\sphinxhref{https://docs.python.org/3/library/functions.html\#float}{float}

\end{description}\end{quote}

\end{fulllineitems}

\index{molarCCUnit (specie.Specie attribute)}

\begin{fulllineitems}
\phantomsection\label{\detokenize{support_rst/specie:specie.Specie.molarCCUnit}}\pysigline{\sphinxbfcode{\sphinxupquote{molarCCUnit}}}
Returns the unit used to measure molar density of the species.
\begin{quote}\begin{description}
\item[{Returns}] \leavevmode
\sphinxstylestrong{molarCCUnit}

\item[{Return type}] \leavevmode
\sphinxhref{https://docs.python.org/3/library/stdtypes.html\#str}{str}

\end{description}\end{quote}

\end{fulllineitems}

\index{molarGammaPwr (specie.Specie attribute)}

\begin{fulllineitems}
\phantomsection\label{\detokenize{support_rst/specie:specie.Specie.molarGammaPwr}}\pysigline{\sphinxbfcode{\sphinxupquote{molarGammaPwr}}}
Returns the amount of gamma radiation produced per mole of this species
(measured in units of power).
\begin{quote}\begin{description}
\item[{Returns}] \leavevmode
\sphinxstylestrong{molarGammaPwr}

\item[{Return type}] \leavevmode
\sphinxhref{https://docs.python.org/3/library/functions.html\#float}{float}

\end{description}\end{quote}

\end{fulllineitems}

\index{molarGammaPwrUnit (specie.Specie attribute)}

\begin{fulllineitems}
\phantomsection\label{\detokenize{support_rst/specie:specie.Specie.molarGammaPwrUnit}}\pysigline{\sphinxbfcode{\sphinxupquote{molarGammaPwrUnit}}}
Returns the unit used to measure the amount of gamma radiation produced
per mole of this species.
\begin{quote}\begin{description}
\item[{Returns}] \leavevmode
\sphinxstylestrong{molarGammaPwrUnit}

\item[{Return type}] \leavevmode
\sphinxhref{https://docs.python.org/3/library/stdtypes.html\#str}{str}

\end{description}\end{quote}

\end{fulllineitems}

\index{molarHeatPwr (specie.Specie attribute)}

\begin{fulllineitems}
\phantomsection\label{\detokenize{support_rst/specie:specie.Specie.molarHeatPwr}}\pysigline{\sphinxbfcode{\sphinxupquote{molarHeatPwr}}}
Returns the amount of heat generated per mole of this species.
\begin{quote}\begin{description}
\item[{Returns}] \leavevmode
\sphinxstylestrong{molarHeatPwr}

\item[{Return type}] \leavevmode
\sphinxhref{https://docs.python.org/3/library/functions.html\#float}{float}

\end{description}\end{quote}

\end{fulllineitems}

\index{molarHeatPwrUnit (specie.Specie attribute)}

\begin{fulllineitems}
\phantomsection\label{\detokenize{support_rst/specie:specie.Specie.molarHeatPwrUnit}}\pysigline{\sphinxbfcode{\sphinxupquote{molarHeatPwrUnit}}}
Returns the unit used to measure the amount of heat generated per mole
of this species.
\begin{quote}\begin{description}
\item[{Returns}] \leavevmode
\sphinxstylestrong{molarHeatPwrUnit}

\item[{Return type}] \leavevmode
\sphinxhref{https://docs.python.org/3/library/stdtypes.html\#str}{str}

\end{description}\end{quote}

\end{fulllineitems}

\index{molarMass (specie.Specie attribute)}

\begin{fulllineitems}
\phantomsection\label{\detokenize{support_rst/specie:specie.Specie.molarMass}}\pysigline{\sphinxbfcode{\sphinxupquote{molarMass}}}
Returns the numerical value for the molar mass of the species. Units
are given by molarMassUnit.
\begin{quote}\begin{description}
\item[{Returns}] \leavevmode
\sphinxstylestrong{molarMass}

\item[{Return type}] \leavevmode
\sphinxhref{https://docs.python.org/3/library/functions.html\#float}{float}

\end{description}\end{quote}

\end{fulllineitems}

\index{molarMassUnit (specie.Specie attribute)}

\begin{fulllineitems}
\phantomsection\label{\detokenize{support_rst/specie:specie.Specie.molarMassUnit}}\pysigline{\sphinxbfcode{\sphinxupquote{molarMassUnit}}}
Returns the unit used to measure the molar mass of the species.
\begin{quote}\begin{description}
\item[{Returns}] \leavevmode
\sphinxstylestrong{molarMassUnit}

\item[{Return type}] \leavevmode
\sphinxhref{https://docs.python.org/3/library/stdtypes.html\#str}{str}

\end{description}\end{quote}

\end{fulllineitems}

\index{molarRadioactivity (specie.Specie attribute)}

\begin{fulllineitems}
\phantomsection\label{\detokenize{support_rst/specie:specie.Specie.molarRadioactivity}}\pysigline{\sphinxbfcode{\sphinxupquote{molarRadioactivity}}}
Returns the numerical value for molar radioactivity of the species.
\begin{quote}\begin{description}
\item[{Returns}] \leavevmode
\sphinxstylestrong{molarRadioactivity}

\item[{Return type}] \leavevmode
\sphinxhref{https://docs.python.org/3/library/functions.html\#float}{float}

\end{description}\end{quote}

\end{fulllineitems}

\index{molarRadioactivityFractions (specie.Specie attribute)}

\begin{fulllineitems}
\phantomsection\label{\detokenize{support_rst/specie:specie.Specie.molarRadioactivityFractions}}\pysigline{\sphinxbfcode{\sphinxupquote{molarRadioactivityFractions}}}
Returns a list of numbers that speciefies the \% of molar reactivity
that comes from each type of atom in the species. For example, a
molarRadioactivityFraction of {[}0.65, 0.35{]} for water means that 65\%
of the molar radioactivity comes from the hydrogen atoms and 35\% comes
from the oxygen atom.
\begin{quote}\begin{description}
\item[{Returns}] \leavevmode
\sphinxstylestrong{molarRadioactivityFractions}

\item[{Return type}] \leavevmode
\sphinxhref{https://docs.python.org/3/library/stdtypes.html\#list}{list}

\end{description}\end{quote}

\end{fulllineitems}

\index{molarRadioactivityUnit (specie.Specie attribute)}

\begin{fulllineitems}
\phantomsection\label{\detokenize{support_rst/specie:specie.Specie.molarRadioactivityUnit}}\pysigline{\sphinxbfcode{\sphinxupquote{molarRadioactivityUnit}}}
Returns the unit used to measure molar radioactivity.
\begin{quote}\begin{description}
\item[{Returns}] \leavevmode
\sphinxstylestrong{molarRadioactivityUnit}

\item[{Return type}] \leavevmode
\sphinxhref{https://docs.python.org/3/library/stdtypes.html\#str}{str}

\end{description}\end{quote}

\end{fulllineitems}

\index{nAtoms (specie.Specie attribute)}

\begin{fulllineitems}
\phantomsection\label{\detokenize{support_rst/specie:specie.Specie.nAtoms}}\pysigline{\sphinxbfcode{\sphinxupquote{nAtoms}}}
Returns the total number of atoms comprising the species. For example,
water is comprised of three atoms.
\begin{quote}\begin{description}
\item[{Returns}] \leavevmode
\sphinxstylestrong{nAtoms}

\item[{Return type}] \leavevmode
\sphinxhref{https://docs.python.org/3/library/functions.html\#int}{int}

\end{description}\end{quote}

\end{fulllineitems}

\index{nNuclideTypes (specie.Specie attribute)}

\begin{fulllineitems}
\phantomsection\label{\detokenize{support_rst/specie:specie.Specie.nNuclideTypes}}\pysigline{\sphinxbfcode{\sphinxupquote{nNuclideTypes}}}
Returns the number of different types of atoms comprising the species.
For example, water is composed of two different types of atoms,
hydrogen and oxygen.
\begin{quote}\begin{description}
\item[{Returns}] \leavevmode
\sphinxstylestrong{nNuclideTypes}

\item[{Return type}] \leavevmode
\sphinxhref{https://docs.python.org/3/library/functions.html\#int}{int}

\end{description}\end{quote}

\end{fulllineitems}

\index{name (specie.Specie attribute)}

\begin{fulllineitems}
\phantomsection\label{\detokenize{support_rst/specie:specie.Specie.name}}\pysigline{\sphinxbfcode{\sphinxupquote{name}}}
Returns the empirical name of the species. For example, “water”.
\begin{quote}\begin{description}
\item[{Returns}] \leavevmode
\sphinxstylestrong{name}

\item[{Return type}] \leavevmode
\sphinxhref{https://docs.python.org/3/library/stdtypes.html\#str}{str}

\end{description}\end{quote}

\end{fulllineitems}

\index{phase (specie.Specie attribute)}

\begin{fulllineitems}
\phantomsection\label{\detokenize{support_rst/specie:specie.Specie.phase}}\pysigline{\sphinxbfcode{\sphinxupquote{phase}}}
Returns the phase history of the species.
\begin{quote}\begin{description}
\item[{Returns}] \leavevmode
\sphinxstylestrong{phase}

\item[{Return type}] \leavevmode
dataFrame

\end{description}\end{quote}

\end{fulllineitems}


\end{fulllineitems}



\section{stream}
\label{\detokenize{support_rst/stream:module-stream}}\label{\detokenize{support_rst/stream:stream}}\label{\detokenize{support_rst/stream::doc}}\index{stream (module)}
Author: Valmor F. de Almeida \sphinxhref{mailto:dealmeidav@ornl.gov}{dealmeidav@ornl.gov}; vfda

Stream container

VFdALib support classes

Sat Aug 15 17:24:02 EDT 2015
\index{Stream (class in stream)}

\begin{fulllineitems}
\phantomsection\label{\detokenize{support_rst/stream:stream.Stream}}\pysiglinewithargsret{\sphinxbfcode{\sphinxupquote{class }}\sphinxcode{\sphinxupquote{stream.}}\sphinxbfcode{\sphinxupquote{Stream}}}{\emph{timeStamp}, \emph{species=None}, \emph{quantities=None}, \emph{values=0.0}}{}
Bases: \sphinxhref{https://docs.python.org/3/library/functions.html\#object}{\sphinxcode{\sphinxupquote{object}}}
\index{GetActors() (stream.Stream method)}

\begin{fulllineitems}
\phantomsection\label{\detokenize{support_rst/stream:stream.Stream.GetActors}}\pysiglinewithargsret{\sphinxbfcode{\sphinxupquote{GetActors}}}{}{}
Returns the actors present in the stream of data.
\begin{quote}\begin{description}
\item[{Returns}] \leavevmode
\sphinxstylestrong{list(self.stream.columns)}

\item[{Return type}] \leavevmode
\sphinxhref{https://docs.python.org/3/library/stdtypes.html\#list}{list}

\end{description}\end{quote}

\end{fulllineitems}

\index{GetQuantities() (stream.Stream method)}

\begin{fulllineitems}
\phantomsection\label{\detokenize{support_rst/stream:stream.Stream.GetQuantities}}\pysiglinewithargsret{\sphinxbfcode{\sphinxupquote{GetQuantities}}}{}{}
Returns all the quantities given by the stream.
\begin{quote}\begin{description}
\item[{Returns}] \leavevmode
\sphinxstylestrong{self.quantities}

\item[{Return type}] \leavevmode
\sphinxhref{https://docs.python.org/3/library/stdtypes.html\#list}{list}

\end{description}\end{quote}

\end{fulllineitems}

\index{GetQuantity() (stream.Stream method)}

\begin{fulllineitems}
\phantomsection\label{\detokenize{support_rst/stream:stream.Stream.GetQuantity}}\pysiglinewithargsret{\sphinxbfcode{\sphinxupquote{GetQuantity}}}{\emph{name}}{}
Returns the specified quantity called “name” from the stream, or none
if the specified name does not exist.
\begin{quote}\begin{description}
\item[{Parameters}] \leavevmode
\sphinxstyleliteralstrong{\sphinxupquote{name}} (\sphinxhref{https://docs.python.org/3/library/stdtypes.html\#str}{\sphinxstyleliteralemphasis{\sphinxupquote{str}}}) \textendash{} 

\item[{Returns}] \leavevmode
\sphinxstylestrong{quant}

\item[{Return type}] \leavevmode
\sphinxhref{https://docs.python.org/3/library/functions.html\#float}{float}

\end{description}\end{quote}

\end{fulllineitems}

\index{GetRow() (stream.Stream method)}

\begin{fulllineitems}
\phantomsection\label{\detokenize{support_rst/stream:stream.Stream.GetRow}}\pysiglinewithargsret{\sphinxbfcode{\sphinxupquote{GetRow}}}{\emph{timeStamp=None}}{}
Returns an entire row of data from the stream. A row of data is all
the data in a dataframe at a specified time stamp, given by timeStamp.
If timeStamp is not specified, this function will return the entire
stream dataframe.
\begin{quote}\begin{description}
\item[{Parameters}] \leavevmode
\sphinxstyleliteralstrong{\sphinxupquote{timeStamp}} (\sphinxhref{https://docs.python.org/3/library/functions.html\#float}{\sphinxstyleliteralemphasis{\sphinxupquote{float}}}) \textendash{} 

\item[{Returns}] \leavevmode
\begin{itemize}
\item {} 
\sphinxstylestrong{self.stream.loc{[}self.timestamp,} (\sphinxstyleemphasis{{]}) or self.stream.loc{[}timeStamp, :{]}):})

\item {} 
\sphinxstyleemphasis{list}

\end{itemize}


\end{description}\end{quote}

\end{fulllineitems}

\index{GetSpecie() (stream.Stream method)}

\begin{fulllineitems}
\phantomsection\label{\detokenize{support_rst/stream:stream.Stream.GetSpecie}}\pysiglinewithargsret{\sphinxbfcode{\sphinxupquote{GetSpecie}}}{\emph{name}}{}
Returns a specie named “name” from the stream.
\begin{quote}\begin{description}
\item[{Parameters}] \leavevmode
\sphinxstyleliteralstrong{\sphinxupquote{name}} (\sphinxhref{https://docs.python.org/3/library/stdtypes.html\#str}{\sphinxstyleliteralemphasis{\sphinxupquote{str}}}) \textendash{} 

\item[{Returns}] \leavevmode
\sphinxstylestrong{specie}

\item[{Return type}] \leavevmode
obj

\end{description}\end{quote}

\end{fulllineitems}

\index{GetSpecies() (stream.Stream method)}

\begin{fulllineitems}
\phantomsection\label{\detokenize{support_rst/stream:stream.Stream.GetSpecies}}\pysiglinewithargsret{\sphinxbfcode{\sphinxupquote{GetSpecies}}}{}{}
Returns a list of all species in the stream.
\begin{quote}\begin{description}
\item[{Returns}] \leavevmode
\sphinxstylestrong{self.species}

\item[{Return type}] \leavevmode
\sphinxhref{https://docs.python.org/3/library/stdtypes.html\#list}{list}

\end{description}\end{quote}

\end{fulllineitems}

\index{GetTimeStamp() (stream.Stream method)}

\begin{fulllineitems}
\phantomsection\label{\detokenize{support_rst/stream:stream.Stream.GetTimeStamp}}\pysiglinewithargsret{\sphinxbfcode{\sphinxupquote{GetTimeStamp}}}{}{}
Returns the time stamp of the stream.
\begin{quote}\begin{description}
\item[{Returns}] \leavevmode
\sphinxstylestrong{self.timeStamp}

\item[{Return type}] \leavevmode
\sphinxhref{https://docs.python.org/3/library/functions.html\#float}{float}

\end{description}\end{quote}

\end{fulllineitems}

\index{GetValue() (stream.Stream method)}

\begin{fulllineitems}
\phantomsection\label{\detokenize{support_rst/stream:stream.Stream.GetValue}}\pysiglinewithargsret{\sphinxbfcode{\sphinxupquote{GetValue}}}{\emph{actor}, \emph{timeStamp=None}}{}
Returns the value associated with a specified “actor” at a specified
“timeStamp”. If no timeStamp is specified, then the function will
return all values associated with the specified actor at all time
stamps.
\begin{quote}\begin{description}
\item[{Parameters}] \leavevmode\begin{itemize}
\item {} 
\sphinxstyleliteralstrong{\sphinxupquote{actor}} (\sphinxhref{https://docs.python.org/3/library/stdtypes.html\#str}{\sphinxstyleliteralemphasis{\sphinxupquote{str}}}) \textendash{} 

\item {} 
\sphinxstyleliteralstrong{\sphinxupquote{timeStamp}} (\sphinxhref{https://docs.python.org/3/library/functions.html\#float}{\sphinxstyleliteralemphasis{\sphinxupquote{float}}}) \textendash{} 

\end{itemize}

\item[{Returns}] \leavevmode
\begin{itemize}
\item {} 
\sphinxstyleemphasis{self.stream.loc{[}self.timeStamp, actor{]} or self.stream.loc{[}timeStamp,}

\item {} 
\sphinxstylestrong{actor{]}} (\sphinxstyleemphasis{list or float, respectively.})

\end{itemize}


\end{description}\end{quote}

\end{fulllineitems}

\index{SetSpecieId() (stream.Stream method)}

\begin{fulllineitems}
\phantomsection\label{\detokenize{support_rst/stream:stream.Stream.SetSpecieId}}\pysiglinewithargsret{\sphinxbfcode{\sphinxupquote{SetSpecieId}}}{\emph{name}, \emph{val}}{}
Sets the numerical id of the specie of name “name” to val.
\begin{quote}\begin{description}
\item[{Parameters}] \leavevmode\begin{itemize}
\item {} 
\sphinxstyleliteralstrong{\sphinxupquote{name}} (\sphinxhref{https://docs.python.org/3/library/stdtypes.html\#str}{\sphinxstyleliteralemphasis{\sphinxupquote{str}}}) \textendash{} 

\item {} 
\sphinxstyleliteralstrong{\sphinxupquote{val}} (\sphinxhref{https://docs.python.org/3/library/functions.html\#int}{\sphinxstyleliteralemphasis{\sphinxupquote{int}}}) \textendash{} 

\end{itemize}

\end{description}\end{quote}

\end{fulllineitems}

\index{SetValue() (stream.Stream method)}

\begin{fulllineitems}
\phantomsection\label{\detokenize{support_rst/stream:stream.Stream.SetValue}}\pysiglinewithargsret{\sphinxbfcode{\sphinxupquote{SetValue}}}{\emph{actor}, \emph{value=None}, \emph{timeStamp=None}}{}
Sets the value associated with a specified actor at a specified
timeStamp to “value”. If no value is specified, the value will default
to 0.0. If no timeStamp is specified, it will set all values associated
with actor to the specified value (or 0.0 if value = None).
\begin{quote}\begin{description}
\item[{Parameters}] \leavevmode\begin{itemize}
\item {} 
\sphinxstyleliteralstrong{\sphinxupquote{actor}} (\sphinxhref{https://docs.python.org/3/library/stdtypes.html\#str}{\sphinxstyleliteralemphasis{\sphinxupquote{str}}}) \textendash{} 

\item {} 
\sphinxstyleliteralstrong{\sphinxupquote{value}} (\sphinxhref{https://docs.python.org/3/library/functions.html\#float}{\sphinxstyleliteralemphasis{\sphinxupquote{float}}}) \textendash{} 

\item {} 
\sphinxstyleliteralstrong{\sphinxupquote{timeStamp}} (\sphinxhref{https://docs.python.org/3/library/functions.html\#float}{\sphinxstyleliteralemphasis{\sphinxupquote{float}}}) \textendash{} 

\end{itemize}

\end{description}\end{quote}

\end{fulllineitems}


\end{fulllineitems}



\renewcommand{\indexname}{Python Module Index}
\begin{sphinxtheindex}
\def\bigletter#1{{\Large\sffamily#1}\nopagebreak\vspace{1mm}}
\bigletter{a}
\item {\sphinxstyleindexentry{actor}}\sphinxstyleindexpageref{support_rst/actor:\detokenize{module-actor}}
\item {\sphinxstyleindexentry{adjudication}}\sphinxstyleindexpageref{examples_rst/adjudication:\detokenize{module-adjudication}}
\item {\sphinxstyleindexentry{arrested}}\sphinxstyleindexpageref{examples_rst/arrested:\detokenize{module-arrested}}
\indexspace
\bigletter{b}
\item {\sphinxstyleindexentry{body}}\sphinxstyleindexpageref{examples_rst/body:\detokenize{module-body}}
\indexspace
\bigletter{c}
\item {\sphinxstyleindexentry{community}}\sphinxstyleindexpageref{examples_rst/community:\detokenize{module-community}}
\item {\sphinxstyleindexentry{cortix\_main}}\sphinxstyleindexpageref{src_rst/cortix_main:\detokenize{module-cortix_main}}
\indexspace
\bigletter{d}
\item {\sphinxstyleindexentry{dataplot}}\sphinxstyleindexpageref{examples_rst/dataplot:\detokenize{module-dataplot}}
\item {\sphinxstyleindexentry{droplet}}\sphinxstyleindexpageref{examples_rst/droplet:\detokenize{module-droplet}}
\item {\sphinxstyleindexentry{dummy\_module}}\sphinxstyleindexpageref{examples_rst/dummy_module:\detokenize{module-dummy_module}}
\indexspace
\bigletter{f}
\item {\sphinxstyleindexentry{fuel\_bucket}}\sphinxstyleindexpageref{support_rst/fuel_bucket:\detokenize{module-fuel_bucket}}
\item {\sphinxstyleindexentry{fuel\_bundle}}\sphinxstyleindexpageref{support_rst/fuel_bundle:\detokenize{module-fuel_bundle}}
\item {\sphinxstyleindexentry{fuel\_segment}}\sphinxstyleindexpageref{support_rst/fuel_segment:\detokenize{module-fuel_segment}}
\item {\sphinxstyleindexentry{fuelsegmentsgroups}}\sphinxstyleindexpageref{support_rst/fuelsegmentsgroups:\detokenize{module-fuelsegmentsgroups}}
\item {\sphinxstyleindexentry{fuelslug}}\sphinxstyleindexpageref{support_rst/fuelslug:\detokenize{module-fuelslug}}
\indexspace
\bigletter{j}
\item {\sphinxstyleindexentry{jail}}\sphinxstyleindexpageref{examples_rst/jail:\detokenize{module-jail}}
\indexspace
\bigletter{m}
\item {\sphinxstyleindexentry{module}}\sphinxstyleindexpageref{src_rst/module:\detokenize{module-module}}
\indexspace
\bigletter{n}
\item {\sphinxstyleindexentry{network}}\sphinxstyleindexpageref{src_rst/network:\detokenize{module-network}}
\item {\sphinxstyleindexentry{nuclides}}\sphinxstyleindexpageref{support_rst/nuclides:\detokenize{module-nuclides}}
\indexspace
\bigletter{p}
\item {\sphinxstyleindexentry{parole}}\sphinxstyleindexpageref{examples_rst/parole:\detokenize{module-parole}}
\item {\sphinxstyleindexentry{periodictable}}\sphinxstyleindexpageref{support_rst/periodictable:\detokenize{module-periodictable}}
\item {\sphinxstyleindexentry{phase}}\sphinxstyleindexpageref{support_rst/phase:\detokenize{module-phase}}
\item {\sphinxstyleindexentry{plot\_data}}\sphinxstyleindexpageref{examples_rst/plot_data:\detokenize{module-plot_data}}
\item {\sphinxstyleindexentry{port}}\sphinxstyleindexpageref{src_rst/port:\detokenize{module-port}}
\item {\sphinxstyleindexentry{prison}}\sphinxstyleindexpageref{examples_rst/prison:\detokenize{module-prison}}
\item {\sphinxstyleindexentry{probation}}\sphinxstyleindexpageref{examples_rst/probation:\detokenize{module-probation}}
\indexspace
\bigletter{q}
\item {\sphinxstyleindexentry{quantity}}\sphinxstyleindexpageref{support_rst/quantity:\detokenize{module-quantity}}
\indexspace
\bigletter{r}
\item {\sphinxstyleindexentry{run\_city\_justice}}\sphinxstyleindexpageref{examples_rst/run_city_justice:\detokenize{module-run_city_justice}}
\item {\sphinxstyleindexentry{run\_droplet\_swirl}}\sphinxstyleindexpageref{examples_rst/run_droplet_swirl:\detokenize{module-run_droplet_swirl}}
\item {\sphinxstyleindexentry{run\_droplet\_test}}\sphinxstyleindexpageref{examples_rst/run_droplet_test:\detokenize{module-run_droplet_test}}
\item {\sphinxstyleindexentry{run\_planets}}\sphinxstyleindexpageref{examples_rst/run_planets:\detokenize{module-run_planets}}
\item {\sphinxstyleindexentry{run\_region\_justice}}\sphinxstyleindexpageref{examples_rst/run_region_justice:\detokenize{module-run_region_justice}}
\indexspace
\bigletter{s}
\item {\sphinxstyleindexentry{specie}}\sphinxstyleindexpageref{support_rst/specie:\detokenize{module-specie}}
\item {\sphinxstyleindexentry{state}}\sphinxstyleindexpageref{examples_rst/state:\detokenize{module-state}}
\item {\sphinxstyleindexentry{stream}}\sphinxstyleindexpageref{support_rst/stream:\detokenize{module-stream}}
\indexspace
\bigletter{v}
\item {\sphinxstyleindexentry{vortex}}\sphinxstyleindexpageref{examples_rst/vortex:\detokenize{module-vortex}}
\end{sphinxtheindex}

\renewcommand{\indexname}{Index}
\printindex
\end{document}